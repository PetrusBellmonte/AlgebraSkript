\documentclass[../main.tex]{subfiles}
\graphicspath{{\subfix{../images/}}}

\begin{document}
\begin{flushright}
VL vom 26.10.2023:
\end{flushright}
\addtocounter{theorem}{-1}
\begin{reminder}
    Ein \emph{Normalteiler} $N$ einer Gruppe $(G,*)$ ist eine Untergruppe mit der Eigenschaft  $\forall g\in G: gNg^{-1} = N$. Man definiert auf den Nebenklassen $G/N = \{gN\mid g\in G\}$ eine Verknüpfung $g_1N \cdot g_2N = g_1g_2N$, die aus $G/N$ eine Gruppe macht. Weiter ist $G\rightarrow G/N, g\mapsto gN$ ein Homomorphismus. Notation $N\nteq G$.

    Z.B. Die alternierende Grruppe $A_n$ ist ein Normalteiler der symmetrischen Gruppe $S_3$.

\end{reminder}
Ansatz: Verstehe eine Gruppe $G$, indem man Normalteiler $\{e\} \neq N \nt G$ und dann $G/N$ studiert.
\subsection{Einfache Gruppen}
\begin{definition}[Einfache Gruppe]
Eine Gruppe $(G,*)$ heißt \emph{einfach}, wenn $G \neq \{1\}$ und die trivialen Normalteiler $\{1\}, G$ die einzigen Normalteiler von $G$ sind.
\end{definition}
\begin{example}
    $\Z/n\Z$ ist einfach gdw. $n$ prim ist. Andernfalls folgt mit dem chinesischen Restsatz für alle $d \mid n$, dass $\mathbb{Z}/d\mathbb{Z} \trianglelefteq \mathbb{Z}/n\mathbb{Z}$.
\end{example}
Wir verfolgen das Ziel, Gruppen zu verstehen, indem wir sie in einfache Normalteiler zerlegen und diese sowie deren Quotientengruppen separat untersuchen, welche hoffentlich eine simplere Struktur haben. Für endliche Gruppen haben die nichttrivialen Normalteiler bspw. echt kleinere Kardinalität.
\begin{theorem}[$A_5$]
Die alternierende Gruppe $A_n$ ist einfach für $n \geq 5$.
\end{theorem}
\begin{proof}
Wir wissen, dass $A_n$ von 3-Zykeln erzeugt wird (\cref{lem:found:1}).
Weiterhin sind alle 3-Zykel in $A_n$ konjugiert zueinander, d.h. für jeden 3-Zykel $\sigma \in A_n$ existiert $\tau \in A_n$ (nicht unbedingt ein 3-Zykel) mit $\tau\sigma\tau^{-1} = (1\ 2\ 3)$\footnote{Es gibt $\tau_0\in S_n$ mit $\tau_0\sigma\tau_0^{-1}=(1\ 2\ 3)$. Falls $\tau_0 \in A_n$ \checkmark, sonst betrachte $\tau = (4\ 5)\tau$: $\tau \sigma \tau^{-1} = (4\ 5)\tau_0\sigma\tau_0^{-1} =(4\ 5)(1\ 2\ 3)(5\ 4) = (4\ 5)(5\ 4)(1\ 2\ 3)=(1\ 2\ 3)$}.
Sei $N\nteq A_n$ ein Normalteiler mit $N\neq \{1\}$. z.z. $N=A$. Das erreichen wir indem wir zeigen, dass $N$ enthält einen $3$-Zykel, da alle $3$-Zyklen zueinander konjugiert sind und $A_n$ erzeugen, wodurch $N=A_n$ gelten müsste.
Wähle $\sigma \in N\setminus\{1\}$:
\begin{itemize}[font=\itshape,align= left]
    \item[Fall 1: $\sigma$ enthält einen Zyklus der Länge $\geq 4$] \obda $\sigma=(1\ 2\ \dots\ r) \rho$, $\forall i\in\{1,2,3\}:\rho(i)=i$. Dann $\sigma^{-1}(1\ 3\ 2)\sigma(1\ 2\ 3) = (2\ 3\ r)\in N$.
    \item[Fall 2: $\sigma$ hat als längsten Zykel einen 3-Zykel (aber ist keiner)] \obda $\sigma = (1\ 2\ 3)\rho$ mit $\forall i\in\{1,2,3\}:\rho(i)=i$ und $\rho(4)\neq 4$. dann besitzt $N\ni \sigma^{-1}(2\ 3\ 4)\sigma (2\ 4\ 3) = (1\ 2\ 4\ 3\ \dots)$ ($\Rightarrow$ Fall 1)
    \item[Fall 3: $\sigma$ besteht nur aus Transpositonen(aber gerade Anzahl)] \obda $\sigma = (1\ 2)(3\ 4)\rho$, $\forall i\in\{1,2,34,\}:\rho(i)=i$. Dann ist $\sigma^{-1}(1\ 3\ 2)\sigma(1\ 2\ 3)=(1\ 4)(2\ 3)\in N$. Weiter gilt: Alle Elemente in $A_n$ von diesen Zykeltyp sind in $A_n$ zueinander konjugiert (vgl. oben mit $\tau=(1\ 2)\tau_0$). Also liegt auch $(1\ 2)(3\ 4)(2\ 5) (3\ 4) = (1\ 2\ 5) \in N$ ($\Rightarrow$ Fall 2)
\end{itemize}

\end{proof}

\subsection{Normal- und Kompositionsreihen}

\begin{definition}[Normalreihe, Kompositionsreihe]
Sei $G$ eine Gruppe. Eine \emph{Normalreihe} ist eine aufsteigende Folge von Untergruppen $\trivGO=G_0\nt G_1\nt \dots \nt G_n = G$ sodass $G_i$ normal in $G_{i+1}$ ist. Die Quotienten $G_{i+1}/G_i$ heißen \emph{Faktoren} der Reihe $\mathcal{G}$.

Man sagt, dass eine Normalreihe $\mathcal{H}$ von $G$ eine Normalreihe $\mathcal{G}$ \emph{verfeinert}, wenn $\mathcal{H}$ aus $\mathcal{G}$ durch hinzufügen von Termen hervorgeht.

Man sagt, dass $\mathcal{G}$ und $\mathcal{H}$ äquivalent sind, wenn sie die gleiche Länge haben und es eine Permutation $\sigma \in S_n$ gibt mit $H_{i+1}/H_i \cong G_{\sigma(i)+1}/G_{\sigma(i)}$.

Eine Normalreihe, die keine echte Verfeinerung besitzt, heißt Kompositionsreihe.
\end{definition}

\begin{flushright}
VL vom 27.10.2023:
\end{flushright}

\begin{example}[Kompositionsreihen von $(G := (\Z,+)$]
    Alle Untergruppen von $G$ sind Normalteiler, da es sich um eine abelsche Gruppe handelt (\cref{lem:found:2}). Weiterhin haben alle Untergruppen von $G$ die Form $n\mathbb{Z}$ für ein $n \in \mathbb{N}_0$.
    Sei nun $n \in \mathbb{N}$. Dann ist
    $$\mathcal{G}: \{0\} = G_0 \; \triangleleftneq \; n\mathbb{Z} \; \triangleleftneq \; \mathbb{Z} = G$$ eine Normalreihe in $G$ mit den Faktoren $$G_1/G_0 = \{\{k\} \mid k \in n\mathbb{Z}\} \; \cong \; n\mathbb{Z} \; \cong \; \mathbb{Z}, \; G/G_1 = \mathbb{Z}/n\mathbb{Z}$$
    $G$ besitzt allerdings keine Kompositionsreihe, denn für jede Normalreihe $$\{0\} \; \triangleleftneq \; n\mathbb{Z} \; \triangleleftneq \; ... \; \triangleleftneq \; \mathbb{Z} = G$$ ist für alle $1 < k \in \mathbb{N}$ eine echte Verfeinerung gegeben durch $$\{0\} \; \triangleleftneq \; (kn)\mathbb{Z} \; \triangleleftneq \; n\mathbb{Z} \; \triangleleftneq \; \; ... \; \triangleleftneq \; \mathbb{Z} = G$$
    wobei $n\mathbb{Z}/(kn)\mathbb{Z} \; \cong \; k\mathbb{Z}$ \TODO.
\end{example}
\begin{theorem} Es gelten die folgenden Charakterisierungen von Kompositionsreihen:
    \begin{enumerate}[label=(\alph*)]
        \item Eine Normalreihe ist genau dann eine Kompositionsreihe, wenn alle Faktoren einfach sind.
        \item Jede endliche Gruppe besitzt eine Kompositionsreihe.
    \end{enumerate}
\end{theorem}
\begin{proof}
    Zu (a): 
    \begin{itemize}
        \item[$\Rightarrow$] Der Beweis erfolgt durch Kontraposition. Sei dazu $$\mathcal{G}: \{1\} = G_0 \; \triangleleftneq \; G_1 \; \triangleleftneq \; ... \; \triangleleftneq \; G_i \; \triangleleftneq \; ... \; \triangleleftneq \; G$$ so dass $G_i/G_{i-1}$ nicht einfach ist. Sei weiterhin $\pi_i: G_i \rightarrow G_i/G_{i-1}$ die kanonische Projektion. Per Definition existiert dann ein nichttrivialer Normalteiler $N$ von $G_i/G_{i-1}$, also $(\{G_{i-1}\} =) \{1_{G_i/G_{i-1}}\} \; \neq N \; \triangleleftneq \; G_i/G_{i-1}$. Dann ist mit \cref{lem:found:4} (beachte, dass die kanon. Projektion surjektiv ist und $\pi^{-1}(\{G_{i-1}\}) = G_{i-1}, \; \pi^{-1}(G_i/G_{i-1}) = G_i$) $$\{1\} = G_0 \; \triangleleftneq \; G_1 \; \triangleleftneq \; ... \; \triangleleftneq G_{i-1} \; \triangleleftneq \; \pi_i^{-1}(N) \; \triangleleftneq \; G_i \; \triangleleftneq \; ... \; \triangleleftneq \; G$$ eine echte Verfeinerung von $\mathcal{G}$, also ist $\mathcal{G}$ keine Kompositionsreihe.
        \item[$\Leftarrow$] Sei $\mathcal{G}$ eine Normalreihe mit einfachen Faktoren und $\mathcal{H}$ eine Verfeinerung. Z.z. $\mathcal{H} = \mathcal{G}$, d.h. $G_i = H_i$ für alle $i$. Beiweis durch Induktion.

        IA  $i=0$: $G_0=\trivG=H_0$ \checkmark

        IS: Es existiert $j>i$ mit $H_i = G_{i+1}$. $G_i \subseteq H_{j-1}\nt H_j = G_{i+1} \overset{\rightarrow}{\pi_i} G_{i+1}/=G_i$ einfach. Da surjektive Homomorphismen Normalteiler erhalten gilt $\pi_i(H_{j-1})\subseteq G_{i+1}/G_i$. Wegen "Einfachheit" $\pi_i(H_j)=\trivG$. $\Rightarrow H_{j-1} = G_i = H_i$.
    \end{itemize}
    Zu (b):
    Induktion über die Mächtigkeit dei Gruppe $\card{G}$. IA $\card{G}=1$: $G=\trivG$ \checkmark.
    IS: Wähle maximalen Normalteiler $N\nt G$. Dann ist $G/N$ einfach. Wende nun IA auf $N$ (um die Kette weiter aufzubauen) an.$\Rightarrow$ Es entsteht eine Reihe mit einfachen Faktoren, also eine Kompositionsreihe.
\end{proof}

Erinnerung: Sei $G$ Gruppe, $N \nteq G$ und $U\leq G$, dann ist $UN=NU$ Untergruppe von $G$. 
\begin{lemma}[Schmetterlingslemma von Zassenhaus]
Sei $G$ Gruppe, $A,B <G$ Untergruppen und $A_0 \nteq A$, $B_0 \nteq B$. Dann
\begin{enumerate}[label=(\alph*)]
        \item $A_0(A\cap B_0)\nteq A_0(A\cap B)$ und $B_0(A_0 \cap B) \nteq B_0(A\cap B)$
        \item $\frac{A_0(A\cap B)}{A_0(A\cap B_0)}\cong\frac{(A\cap B)}{(A_0\cap B)(A\cap B_0)}\cong \frac{B_0(A\cap B)}{B_0(A_0\cap B)}$
    \end{enumerate}
\end{lemma}
\begin{proof}
    \TODO
\end{proof}

\begin{theorem}
    Sind $\mathcal{G}$, $\mathcal{H}$ Normalreihen von $G$, dann gibt es Verfeinerung $\Tilde{\mathcal{G}}$, $\Tilde{\mathcal{H}}$ von $\mathcal{G}$, $\mathcal{H}$, sodass sie äquivalent sind.
\end{theorem}
\begin{proof}
    \TODO
\end{proof}

\begin{theorem}[Jordan Hölder]
    Je zwei Kompositionsreihen einer Gruppe sind äquivalent.
\end{theorem}
\begin{proof}
    Kompositionsreihen haben keine Verfeinerung \& 1.8
\end{proof}
\begin{remark*}
    Gleiche Kompositionsreihen $\notin$ gleiche Gruppe.
\end{remark*}
\begin{example*}
    \begin{itemize}
        \item $\Z/14\Z \cong \Z/2\Z \times \Z/7\Z$ hat Kompositionsreihen $\trivGZ \nt (\Z/2\Z \times \trivGZ) \nt \Z/14\Z$ und $\trivGZ \nt (\trivGZ \times \Z/2\Z) \nt \Z/14\Z$, aber immer die gleichen Faktoren in unterschiedlicher Reihenfolge.
        \item Zu $G=\Z/9\Z$ und $H=\Z/3\Z\times\Z/3\Z$ haben die Kompositionsreihen $\mathcal{G}: \trivGZ \nt 3 \Z/9\Z \nt G$ und $\mathcal{H}: \trivGZ \nt \Z/3\Z \nt H$ die gleichen Faktoren (zweimal $\Z/3\Z$ und damit äquivalent, aber $G\neq H$).
    \end{itemize}
\end{example*}
    

\subsection{Auflösbare Gruppen}
\begin{definition}
    Eine Gruppe $(G,*)$ heißt auflösbar, wenn sie eine Normalreihe besitzt, deren Faktoren alle abelsch sind.
\end{definition}
\begin{flushright}
VL vom 02.11.2023:
\end{flushright}
\begin{example}
    a) Insbesondere ist jede abelsche Gruppe auflösbar: Für die triviale Gruppe $\{1\}$ existieren keine Faktoren, ansonsten setze $$G_0 := \{1\} \; \triangleleftneq \; G_1 := G$$

    b) Sei weiterhin $\mathbb{K}$ ein Körper. Die Matrixgruppe $$(B = \{\begin{pmatrix} a & c \\ 0 & b \end{pmatrix} \in GL_2(\mathbb{K}) \mid a,b,c \in \mathbb{K}\}, \cdot)$$ ist nicht abelsch, aber dennoch auflösbar. Sei dafür $$\mathbb{K}^* = \mathbb{K}\setminus\{0\}$$ und $$\phi: B \rightarrow \mathbb{K}^* \times \mathbb{K}^*, \begin{pmatrix} a & c \\ 0 & b
    \end{pmatrix} \mapsto (a,b)$$ Dann ist $N = ker(\phi) = \{\begin{pmatrix}
        1 & c \\ 0 & 1
    \end{pmatrix} \mid c \in \mathbb{K}\} \cong (\mathbb{K},+)$ Somit ist $$\{id\} \triangleleftneq N \triangleleftneq B$$ eine Normalreihe mit abelschen Faktoren. (Geht auch für $GL_n(K)$, Beweis komplizierter)

    c) (Semidirektes Produkt von Gruppen) \TODO
\end{example}
\begin{theorem}[Untergruppen und Faktorgruppen auflösbarer Gruppen]\label{thm:ug:aufloesbar}\label{theo:1.12}
    Untergruppen und Faktorgruppen auflösbarer Gruppen sind auflösbar.
\end{theorem}
\begin{proof}
    Sei G auflösbar mit abelscher Normalreihe (abelsche Faktoren) 
    $$G_0 := \{1\} \; \triangleleftneq \; ... \; \triangleleftneq \; G_n := G$$
    Sei $H < G$ Untergruppe. Dann ist $$H_0 := \{1\} \; \triangleleft \; ... \; \triangleleft \; H_n := H$$ mit $H_i := G_i \cap H$ eine abelsche Normalreihe von $H$ (nach Streichen gleicher Elemente):
    $$H_{i+1}/_{H_i} < G_{i+1}/_{G_i}$$
\end{proof}
\begin{theorem}\label{theo:1.13}
    Sei $G$ eine endliche Gruppe und $\mathcal{G}$ eine Kompositionsreihe von $G$. Dann ist $G$ auflösbar gdw. jeder Faktor von $\mathcal{G}$ zyklisch von Primzahlordnung ist.
\end{theorem}
\begin{proof}
    "$\Rightarrow$": Sei $G$ auflösbar. Nach Def. von Kompositionsreihen sind dann die $G_i/_{G_{i-1}}$ einfach und nach \cref{thm:ug:aufloesbar} auflösbar. Insbesondere existiert also eine Normalreihe von  $G_i/_{G_{i-1}}$. Dies impliziert, dass $G_i/_{G_{i-1}}$ abelsch ist, da $G_i/_{G_{i-1}}$ in einer solchen Normalreihe vorkommt (Begründung aus VL: da  $\{1\} \triangleleftneq G_i/_{G_{i-1}}$ die einzige Normalreihe ist - wird aber nicht benötigt?)
    $G_i/_{G_{i-1}}$ ist als endliche abelsche Gruppe isomorph zu
    $$\mathbb{Z}/_{p_1^{\alpha_1}\mathbb{Z}} \times ... \times \mathbb{Z}/_{p_m^{\alpha_m}\mathbb{Z}}, \alpha_t \geq 1$$
    Wegen Einfachheit gilt $m = 1$ (ansonsten $\mathbb{Z}/_{p_1^{\alpha_1}\mathbb{Z}} \times \{0\} \times ... \times \{0\}$ nichttrivialer NT). Also $G_i/_{G_{i-1}} \cong \mathbb{Z}/_{p^\alpha\mathbb{Z}}$. Wieder wegen Einfachheit ist $\alpha = 1$ (ansonsten $p \mathbb{Z}/_{p^\alpha\mathbb{Z}}$ nichttrivialer NT).
    "$\Leftarrow$": Offensichtlich.
\end{proof}
\begin{definition}[Kommutator]
    Sei G eine Gruppe. Für $x,y \in G$ heißt $x^{-1}y^{-1}xy = [x,y]$ \emph{Kommutator} von x und y. Die Kommutatoruntergruppe von G ist $$D(G) = <[x,y] \mid x,y \in G>$$
    Alternativnotation: $[G,G] := D(G)$.
\end{definition}
\begin{lemma}
    Die Kommutatoruntergruppe D(G) einer Gruppe G ist ein NT von G. Die Faktorgruppe $G^m = G/_{D(G)}$ ist abelsch und heißt \emph{Abelisierung} von G.
    Ist $N \trianglelefteq G$ und $G/_N$ abelsch, dann ist $D(G) \subseteq N$ (d.h. $G/_{D(G)} ->> G/_N$ ist surjektiv)
\end{lemma}
\begin{proof}
    Es ist $g^{-1}[x,y]g = g^{-1}x^{-1}y^{-1}xyg = (g^{-1}x^{-1}g)(g^{-1}yg)(g^{-1}xg)(g^{-1}yg) = [g^{-1}xg, g^{-1}yg]$. Somit ist $g^{-1}D(G)g = <[g^{-1}xg, g^{-1}yg] \mid x,g \in G> = D(G)$.
    
    Seien $xD(F), yD(G) \in G/D(G)$. Es ist dann $xyD(G) = xy[y,x]D(G) = xyy^{-1}x^{-1}yxD(G) = yxD(G)$. 
    
    Betrachte die kanonische Projektion $\pi: G ->> G/N$. Dann ist $\pi([x,y]) = \pi(x^{-1}y^{-1}xy) = \pi(x)^{-1}\pi(y)^{-1}\pi(x)\pi(y) = [\pi(x), \pi(y)] = N = \in G/N$ abelsch.
    Also $D(G) \subseteq N$.
\end{proof}
\begin{definition}
    Setze $D^0(G) = G$ und dann induktiv $D^{i+1}(G) := D(D^i(G))$. Die Reihe 
    $$... \triangleleft D^2(G) \triangleleft D^1(G) \triangleleft D^0(G) = G$$
    (mit abelschen Faktoren nach dem vorhergehenden Lemma) heißt \emph{abgeleitete Reihe} von G.
\end{definition}
\begin{theorem}
    Eine Gruppe G ist auflösbar gdw. es ein $m \in \mathbb{N}$ gibt mit $D^m(G) = \{1\}$. (Dies ist nicht immer erfüllt, da es Gruppen gibt mit $D(G) = G$, so dass die abgeleitete Reihe konstant G ist.)
\end{theorem}
\begin{proof}
    "$\Leftarrow$": ist klar (good one).

    "$\Rightarrow$": Sei G auflösbar. Dann gibt es eine abelsche Normalreihe
    $$\{1\} = G_0 \triangleleftneq ... \triangleleftneq G_n = G$$
    Wir zeigen induktiv über die Länge $n$ der Normalreihe: $D^j(G) \subseteq G_{n-j}, j \in [n]_0$.
    IA: $n=0$. Klar: $D^0(G) = G = G_0 = G_n$
    ISt: Angenommen $D^j(G) \subseteq G_{n-j}$. Dann $D^{j+1}(G) = D(D^j(G)) \subseteq D(G_{n-j}) \subseteq G_{n-j-1}$. Die letzte Inklusion folgt aus TODO
\end{proof}
\begin{example}
    Sei G eine Gruppe und p eine Primzahl. Ist $|G| = p^n$, dann ist G auflösbar (und sogar nilpotent).
\end{example}
\end{document}