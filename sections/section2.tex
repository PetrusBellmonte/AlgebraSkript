\documentclass[../main.tex]{subfiles}
\graphicspath{{\subfix{../images/}}}

\begin{document}
\subsection{Irreduzible Polynome}
\begin{definition}[Wiederholung aus EAZ]
    Sei $K$ im Folgenden ein Körper. Der Polynomring $K[X]$ ist ein Hauptidealring\footnote{nullteilerfrei, kommutativ und jedes Ideal ist Hauptideal}. Das Ideal $(f)$ ist Primideal gdw. $f = 0$ oder $f$ irreduzibles Polynom ist, d.h. $f = gh \Rightarrow g \in K^\times \vee h \in K^\times$.
\end{definition}
\begin{lemma} \label{theo:2.2}
    Ist f irreduzibel, dann ist (f) ein maximales Ideal\footnote{Es existiert kein ("größeres") Ideal $I\neq K[X]$ mit $(f)\subset I$}.
\end{lemma}
\begin{proof}
    Angenommen $(f) \subseteq (g)$. Dann $f = hg$ für ein $h \in K[X]$ nach Definition von $(g)$. Dann gilt
    \begin{equation*}
        \begin{cases}
            g \in K^\times \Rightarrow (g) = K[X] \\
            oder \\
            h \in K^\times \Rightarrow (f) = (g)
        \end{cases}
    \end{equation*}
\end{proof}
\begin{theorem}[Eisensteinkriterium]
    Sei A ein kommutativer Ring und $P \subseteq A$ ein Primideal\footnote{e.g. $A = \mathbb{Z}$, $P = \{pn \mid n \in \mathbb{N}\}$, $p$ prim}. Sei $f \in A[X]$ mit $f = \sum_{0 \leq i \leq n} a_iX^i$ mit drei Eigenschaften:
    
    1) $a_n \notin P$

    2) $a_i \in P \quad \forall 0 \leq i \leq n-1$

    3) $a_0$ ist kein Produkt von zwei Elementen in P.

    Dann lässt sich $f$ nicht als Produkt zweier Polynome in $A[X]$ vom Grad $< n$ schreiben.
\end{theorem}
\begin{proof}
    EAZ Kühnlein
\end{proof}
\begin{definition}[Inhalt]
    Sei A ein Hauptidealring und $K = Quot(A)$. Sei $f = \sum_{0 \leq i \leq n} a_i X^i \in A[X]$. Definiere $\Tilde{c}(f) \in A$ als einen Erzeuger des Ideals $(a_0, ..., a_n) \subset A$ (eindeutig bis auf Multiplikation mit Einheiten/inv. Elemente in A). Die Assoziiertenklasse $c(f) = \Tilde{c}(f)A^\times \subset A/A^\times$ [$A^\times$ invertierbare Elemente in A] heißt \emph{Inhalt} von f.\\
    Sei $f \in K[X]$. Wähle $a \in A \setminus \{0\}$ mit $af \in A[X]$. Dann definiere $\Tilde{c}(f) = c(af)a^{-1} \in K$ und $c(f) = \Tilde{c}(f)A^\times \in K/A^\times$ (Übung: Def. unabh. von der Wahl von a).
\end{definition}
\begin{example*}
    $A = \mathbb{Z}, K = \mathbb{Q}$. Dann
    $$f(x) := \frac{2}{5}x^7 - 2x^3 + \frac{8}{3} = \frac{1}{15}(6x^7 - 30x^3 + 40)$$
    also $\Tilde{c}(f) = \frac{ggT(6,30,40)}{15} = \frac{2}{15}$.
\end{example*}
\begin{lemma}\label{theo:2.5}
    Seien $A, K$ wie oben und $f,g \in K[X]$. Dann gilt
    
    a) Für $f \neq 0$ gilt $\Tilde{c}(f)^{-1}f \in A[X]$.

    b) $c(fg) = c(f)c(g)$ (Gauß)
\end{lemma}
\begin{proof}
    Kühnlein EAZ
\end{proof}
\begin{lemma}
    A Hauptidealring, $K = Quot(A)$, $f \in A[X]$ nicht-konstantes Polynom. Falls f sich nicht als Produkt $f = gh$ mit $g,h \in A[X], \deg(g), \deg(h) < \deg(f)$ schreiben lässen, dann ist auch $f \in K[X]$ irreduzibel.
\end{lemma}
\begin{proof}
    Ang. $f = g_0h_0$ mit $g_0, h_0 \in K[X]$. Setze 
    $$g = \Tilde{c}(g_0)^{-1}g_0 \in A[X]$$ und 
    $$h := \Tilde{c}(g_0)\Tilde{c}(h_0)\Tilde{c}(h_0)^{-1}h_0 \overset{\text{\ref{theo:2.5} b}}{=} \underbrace{\Tilde{c}(\overbrace{g_0h_0}^{=f})}_{\in A}a\ \underbrace{\Tilde{c}(h_0)^{-1}h_0}_{\in A[X]}\in A[X]$$ für geeignetes $a \in A^\times$. Somit ist auch $h \in A[X]$. Weiter ist $f = gh$. Dann ist $\deg g_0 = \deg g = \deg f$ oder $\deg h_0 = \deg h = \deg f$.
\end{proof}
\begin{theorem}[Eisensteinkriterium für Irreduzibilität]
    Sei A ein Hauptidealring, $P \subset A$ Primideal und $f \in A[X]$. Erfüllt f die Bedingungen i), ii), iii) des Eisensteinkriteriums, dann ist f irreduzibel in K[X] ($K = Quot(A)$).
\end{theorem}
\begin{example}
    $A = \mathbb{Z}, K = \mathbb{Q}$.

\begin{enumerate}
    \item $f = X^m - a$, $a \in \mathbb{Z}$. Falls $a = \prod_{1 \leq i \leq r} p_i^{\alpha_i}$ mit verschiedenen Primzahlen $p_i$ und es ein $j \in \{1, ..., r\}$ mit $\alpha_i = 1$ gibt, dann ist f irreduzibel in K[X].
    \item Sei p eine Primzahl. Das Polynom $\Phi_p = X^{p-1} + X^{p-2} + ... + X + 1$ heißt das p-te Kreisteilungspolynom. Setze $g(X) := \Phi_p(X+1)$ (dann impliziert insb. Irreduzibilität von g auch Irreduzibilität von $\Phi_p$; "Substitutionstrick" falls Eisensteinkriterium für Irreduzibilität nicht direkt anwendbar). Es ist $(X-1)\Phi_p(X) = X^p - 1$ und daher 
    $$g(X) = \frac{(X+1)^p - 1}{X} = \sum_{1 \leq j \leq p}\binom{p}{j}X^{j-1}$$
    Eisenstein für p liefert nun Irreduzibilität von g (beachte $p \mid \binom{p}{j}$ für $j = 1,...,p-1$).
    Dann $\Phi_p$ irreduzibel. Die Nullstellen von $\Phi_p$ sind gerade die primitiven p-ten Einheitswurzeln.
\end{enumerate}
\end{example}

\subsection{Körpererweiterungen}
\begin{definition}[Körpererweiterung]
    Sei $(L, +, \cdot)$ ein Körper. Sei K ein Teilkörper von L, d.h. $K \subseteq L$ und $(K,+|_K,\cdot|_K)$ ist selbst Körper. Dann bezeichnet man L als Erweiterungskörper von K. Man sagt, dass L über K eine Körpererweiterung (oder auch "K-Erweiterung") ist. 

    Notation: $L | K$, $(L - K)^T$ $\begin{tikzcd}[row sep=0.2cm]
L \arrow[d, no head] \\
K          
\end{tikzcd}$
\end{definition}
\begin{example*} $\mathbb{R}|\mathbb{Q}, \mathbb{C}|\mathbb{R},\mathbb{C}|\mathbb{Q},\mathbb{Q}(\sqrt{2}) | \mathbb{Q}, \mathbb{C}(\mathbb{Z}) | \mathbb{C}$ \end{example*}
\begin{definition}[Endliche Erweiterung]
    Sei $L | K$ eine Erweiterung. Dann ist L insb. ein $K-VR$. Die Dimension über K von L $dim_K(L) =: [L:K]$ heißt der \emph{Grad der Körpererweiterung} $L | K$. Die Erweiterung heißt \emph{endlich}, wenn $[L:K] < \infty$.
\end{definition}
\begin{lemma}[Grad ist multiplikativ]
    Sei $L | K$ eine K-Erweiterung und sei V ein L-Vektorraum mit L-Basis $(v_i)_{i \in I} \subset V$. Sei $(e_j)_{j \in J} \subset L$ eine K-Basis von L. Dann ist $(e_j \cdot v_i)_{i \in I, j \in J}$ (VR-Multiplikation von Skalaren aus L mit Vektoren aus V) eine K-Basis von V.
\end{lemma}
\begin{proof}
$(v_i)$ L-Basis $\Rightarrow$ für jedes $i \in I$ ist $\sum_j c_{ij}e_j = 0$. $(c_j)$ K-Basis $\Rightarrow$ $c_{ij} = 0$ für alle $i \in I, j \in J$
Erzeugendensystem: Sei $v \in V$.
Dann ist $v = \sum_{i \in I} \lambda_iv_i$ mit gewissen $\lambda_i \in L$. Für jedes $i \in I$ ist $\lambda_i = \sum_{j \in J} b_ije_j$ mit gewissen $b_ij \in K$. Also $v = \sum_{i,j} b_{ij}(e_j \cdot v_i)$.
\end{proof}
\begin{lemma}[Korollar] \label{theo:2.12}
    Sind $M | L, L | K$ Körpererweiterungen, dann gilt $[M:K] = [M:L]\cdot[L:K]$ (mit den üblichen Konventionen $\infty \cdot \infty = \infty$).
\end{lemma}
\begin{definition}[Adjungieren]
    Sei $L | K$ eine Körpererweiterung. Sei $S \subset L$ eine Teilmenge. Dann bezeichnet $K(S)$ den kleinsten Teilkörper von L, der K und S enthält. Für $S=\{\alpha\}$ schreibt man $K(\alpha) = K(\{\alpha\})$ (gesprochen "K adjungiert $\alpha$"). Man nennt eine Körpererweiterung $K(\alpha) | K$ \emph{einfach}.
\end{definition}
\begin{definition}[Algebraisch vs. Transzendent]
    Sei $L | K$ eine Körpererweiterung. Sei $\alpha \in L$. Betrachte die Evaluationsabbildung $ev_\alpha: K[X] \rightarrow L, f \mapsto f(\alpha)$. 
    
    Falls $ev_\alpha$ injektiv ist, so nennt man $\alpha$ \emph{transzendent} (dann ist $f(\alpha) \equiv 0 \Rightarrow f \equiv 0$, also ist $\alpha$ nicht Nullstelle eines nichttrivialen Polynoms). Es gilt $Bild(ev_\alpha) \cong K[X]$ und $K(X) \cong K(\alpha)$. Insb. $[K(\alpha):K] = \infty$.
    
    Falls dagegen $ev_\alpha$ nicht injektiv ist, nennt man $\alpha$ \emph{algebraisch}. Dann ist $ker(ev_\alpha) = (m_{\alpha, K})$ Hauptideal. O.B.d.A. sei $m_{\alpha, K}$ normiert (Leitkoeffizient = 1). Wir nennen $m_{\alpha, K}$ das \emph{Minimalpolynom} von $\alpha$ über K. Dann $L \supset Bild(ev_\alpha) \cong K[X]/(m_{\alpha, K})$ nullteilerfrei. Folglich ist $m_{\alpha, K}$ irreduzibel. Damit ist $(m_{\alpha, K})$ sogar ein maximales Ideal in K[X], also ist $Bild(ev_\alpha) \cong K[X]/(m_{\alpha, K})$ sogar ein Körper, also $Bild(ev_\alpha) = K(\alpha)$. (Schreibe auch $K[\alpha] := Bild(ev_\alpha)$.) Es ist $[K(\alpha):K] = deg \; m_{\alpha, K} < \infty$.
\end{definition}
\begin{example}
    $\pi \in \mathbb{C}$ ist transzendent über $\mathbb{Q}$ (Lindemann 1882). Es gibt in $\mathbb{C}$ nur abzählbar viele algebraische Zahlen über $\mathbb{Q}$, also überabzählbar viele transzendente Zahlen.

    $d \in \mathbb{Z}$ sei quadratfrei und $d \neq 1$. Die Zahl $\sqrt{d} \in \mathbb{C}$ ist algebraisch mit Minimalpolynom $m_{\sqrt{d}, \mathbb{Q}} = X^2 - d \in \mathbb{Q}[X]$. Weiter ist dann $\mathbb{Q}(\sqrt{d}) = \mathbb{Q}[\sqrt{d}] = \{a+b\sqrt{d} \mid a,b \in \mathbb{Q}\}$.
\end{example}
\begin{definition}[Algebraische K-Erweiterung]
    Eine K-Erweiterung $L \mid K$ ist algebraisch, wenn jedes $\alpha \in L$ algebraisch über K ist.
\end{definition}

\subsection{Algebraische Körpererweiterungen}
\begin{theorem}
    \begin{enumerate}
        \item Ist $L \mid K$ endlich, dann ist $L \mid K$ algebraisch.
        \item Ist $L \mid K$ algebraisch und endlich erzeugt (d.h. $L = K(\alpha_1, ..., \alpha_n)$ für geeignete $\alpha_i$), dann ist $L \mid K$ endlich.
        \item Sind $M \mid L$ und $L \mid K$ algebraisch, so ist auch $M \mid K$ algebraisch.
    \end{enumerate}
\end{theorem}
\begin{proof}
    \begin{itemize}
        \item[Zu a)] Sei $\alpha \in L$. Dann gilt $[K(\alpha):K] \leq [L:K] < \infty \Rightarrow \alpha$ algebraisch.
        \item[Zu b)] Betrachte die Zwischenkörper $L =: L_n - ... - L_2 := L_1(\alpha_2) = K(\alpha_1, \alpha_2) - L_1 := L_0(\alpha_1) = K(\alpha_1) - L_0 := K$ (also $L_i := L_{i-1}(\alpha_i)$). Da $\alpha_i$ algebraisch über K ist, ist $\alpha_i$ insb. algebraisch über $L_{i-1}$. Somit $[L_i, L_{i-1}] = [L_{i-1}(\alpha_i), L_{i-1}] < \infty \Rightarrow [L:K] = [L_n:L_{n-1}]...[L_1:L_0] < \infty$.
        \item[Zu c)] Sei $\alpha \in M$. Betrachte $m_{\alpha,L} = \sum_{i=0}^n c_iX^i, \; c_i \in L$, Minimalpolynom von $\alpha$ über L. Definiere $K_0 := K(c_0,...,c_n) \subseteq L$. Wegen b) ist $[K_0:K] < \infty$. Wegen $m_\alpha,L \in K_0[X]$, ist $\alpha$ algebraisch über $K_0$. Dann $[K(\alpha):K] \leq [K_0(\alpha):K] = [K_0(\alpha):K_0][K_0:K] < \infty$, also $\alpha$ algebraisch über K.
    \end{itemize}
\end{proof}
\begin{remark}
Sei $L \mid K$ eine K-Erweiterung. Die Menge $L_{alg, K} := \{\alpha \in L \mid \alpha \text{ algebraisch über K}\}$ ist ein Zwischenkörper $L-L_{alg, K}-K$. Diesen nennt man den \emph{algebraischen Abschluss} von K in L.

Begründung: Seien $\alpha, \beta \in L_{alg,K}$. $$[K(\alpha,\beta):K]=[K(\alpha,\beta):K(\alpha)][K(\alpha):K] \leq [K(\beta):K][K(\alpha):K] < \infty$$ (Frage aus VL: Wie würde das Minimalpolynom von $\alpha + \beta$ über K aussehen?)
\end{remark}
Wir stellen uns nun die folgenden Fragen: Sei K ein Körper und $f \in K[X]$ nicht-konstant. Gibt es einen Erweiterungskörper $L \mid K$ so, dass
\begin{itemize}
    \item f eine Nullstelle in L hat?
    \item f in L[X] in Linearfaktoren zerfällt?
    \item jedes Polynom in L[X] in Linearfaktoren zerfällt?
\end{itemize}
\begin{theorem} \label{theo:2.19}
    Seien K ein Körper und $f \in K[X]$ irreduzibel. Dann gibt es eine algebraische K-Erweiterung mit $[L:K] = deg(f)$, in der f eine Nullstelle besitzt.
\end{theorem}
\begin{proof}

    Setze $L := K[X]/(f)$. Nach \cref{theo:2.2} ist L ein Körper. Betrachte die Quotientenabbildung $\pi: K[X] \rightarrow L$ (Homomorphismus!). Setze $\alpha := \pi(X) \in L$. Dann ist 
    $$f(\alpha) = f\left(\pi(X)\right) = \sum_{i=0}^n c_i\pi(X)^i = \pi\left(\sum_{i=0}^n c_iX^i\right) = \pi(f) = 0 \in L$$
    Also $[L:K] = [K(\alpha):K] = deg(f)$, weil $f$ Minimalpolynom von $\alpha$ ist.
\end{proof}
\begin{corollary}\label{theo:2.20}
Seien K ein Körper und $f \in K[X]$ ein nicht-konstantes Polynom\footnote{irreduzible Polynome sind per Definition nicht-konstant}. Es gibt eine endliche K-Erweiterung $L \mid K$, so dass f über L in Linearfaktoren zerfällt.
\end{corollary}
\begin{proof}
    Induktion über $deg(f) =: d$:
    \begin{itemize}
        \item[IA] $d = 1$: Klar, da $f$ selbst Linearfaktor ist.
        \item[ISchritt] Sei $g \in K[X]$ ein irreduzibles Polynom, das $f$ teilt. Nach dem vorhergehenden Satz (\cref{theo:2.19}) gibt es $L_1 \mid K$, in der $g$ eine Nullstelle $\alpha \in L_1$ besitzt. Schreibe $f = (X-\alpha)\Tilde{f}$ mit $\Tilde{f} \in L_1[X]$. Nach I-Annahme existiert $L \mid L_1$ endlich, in der $\Tilde{f}$ in Linearfaktoren zerfällt. Weiter $[L:K] = [L:L_1][L_1:L] < \infty$.
    \end{itemize}
\end{proof}
\begin{definition}
    Ein Körper $K$ ist \emph{algebraisch abgeschlossen}, falls jedes nicht-konstante Polynom $f \in K[X]$ eine Nullstelle in $K$ besitzt.
\end{definition}
\begin{remark}\label{theo:2.22}
    Ist K algebraisch abgeschlossen, dann gilt:
    \begin{itemize}
        \item Jedes $f \in K[X]\setminus K$ zerfällt in Linearfaktoren.
        \item Die irreduziblen Polynome über K sind von Grad 1.
    \end{itemize}
    Weiter gilt: Ist $L \mid K$ algebraisch, dann gilt bereits $L = K$, denn: Das Minimalpolynom $m_\beta$ von $\beta \in L$ ist linear, d.h. $m_\beta = X - \alpha, \alpha \in K \Rightarrow \beta = \alpha \in K$.
\end{remark}
\begin{example*}
    $\mathbb{C}$ ist algebraisch abgeschlossen (Fundamentalsatz der Algebra).
\end{example*}
\begin{lemma}
    Sei A ein kommutativer Ring und $I \subset A$ ein Ideal. Dann ex. ein maximales Ideal $m \subset A$, das I enthält (Beweisidee: Lemma von Zorn).
\end{lemma}
\begin{theorem}\label{theo:2.24}
    Für jeden Körper gibt es einen algebraisch abgeschlossenen Erweiterungskörper.
\end{theorem}
Auswahlaxiom in Algebra: nur hier;
Funktionale Analysis: überall/Hahn-Banach;
Lineare Algebra: Existenz von Basen unendlich dim VR
\begin{proof}
1) Konstruiere $L_1 \mid K$, so dass jedes $f \in K[X]\setminus K =: \Omega$ eine Nullstelle in $L_1$ hat. Definiere $A = K[(X_f)_{f \in \Omega}]$ (Polynomring in unendlich vielen Variablen). Setze $I := (\{f(X_f) \mid f \in \Omega\}) \subseteq A$. Dann ist $I \subsetneq A$. Angenommen $I = A$, also $1 \in I$. Dann $1 = \sum_{i = 1}^r g_if_i(X_{f_i})$ für $g_i \in A, f_i \in \Omega$. Sei $f = f_1...f_r$ Nach \ref{theo:2.20} ex. $F \mid K$, so dass f in Linearfaktoren zerfällt. Insbesondere hat jedes $f_i$ eine Nullstelle $z_i \in F$. Definiere $\phi: A \rightarrow F$ durch $\phi|_K = \text{Inklusion } K -> F$, $\phi(X_f) = 0$ für $f \in \Omega\setminus\{f_1,...,f_r\}$, $\phi(X_{f_i}) = z_i$. Es gilt: $$1 = \phi(1) = \sum_{i = 1}^r \phi(g_i)\phi(f_i(X_{f_i})) = \sum_{i = 1}^r \phi(g_i)f_i(\phi(X_{f_i})) = 0 \in F$$
Nach \ref{theo:2.24} ex. ein max. Ideal $m \subsetneq A$ mit $I \subseteq m$. Setze $L_1 := A/m$.
Dann ist $K -> A ->> A/m = L_1$ eine Einbettung (setze $\pi: A \rightarrow A/m$). Sei $f \in K[X]\setminus K$. Setze $\gamma_f = \pi(X_f) \subseteq L_!$. Dann $f(\gamma_f) = f(\pi(X_f)) = \pi(f(X_f)) = 0$ da $f(X_f) \in I \subseteq m$.

2) Mit Schritt 1) erhalten wir eine Folge von Körpererweiterungen $$K \subseteq L_1 \subseteq L_2 \subseteq ...$$ mit der Eigenschaft, dass jedes $f \in L_j[X]\setminus L_j$ eine Nullstelle in $L_{j+1}$ hat. Definiere $L = \bigcup_{j \geq 1} L_j$. Sei $g \in L[X]\setminus L$. Dann hat g endlich viele Koeffizienten, die alle in einem $L_m$ (m groß genug) liegen. Damit hat g eine Nullstelle in $L_{m+1} \subseteq L$. [Frage aus VL: Warum existiert $L = \bigcup_{j \geq 1} L_j$?]
\end{proof}
\begin{definition}
    Sei K ein Körper. Es gibt einen algebraisch abgeschlossenen Körper $\overline{K}$ so, dass $K \subseteq \overline{K}$ und $\overline{K}\setminus K$ algebraisch ist. Man nennt $\overline{K}$ einen \emph{algebraischen Abschluss}. 
\end{definition}
\begin{proof}
    Sei $L|K$ eine alg. abgeschlossene Erweiterung. Sei $\overline{K} = L_{alg, K} = \{\alpha \in L \mid \alpha \text{ algebraisch über K}\}$. Z.z. $\overline{K}$ ist alg. abgeschlossen. Sei dafür $f \in \overline{K}[X]\setminus\overline{K}$. Es gibt eine Nullstelle $\alpha \in L$ von f. Dann ist $\alpha$ algebraisch über $\overline{K}$. Da $\overline{K}$ alg. über K, ist $\alpha$ alg. über K, also $\alpha \in \overline{K}$.
\end{proof}

\subsection{$\overline{K}$-Homomorphismen}
\begin{definition}
    Seien $L_1| K$ und $L_2|K$ $K$-Erweiterungen. Ein Homomorphismus $f: L_1 \rightarrow L_2$ heißt \emph{$K$-Homomorphismus}, falls $f(k) = k$ für alle $k \in K$. Ein K-Isomorphismus ist ein bijektiver K-Homomorphismus. Definiere ${Aut}(L_1|K) = \{f: L_1 \rightarrow L_1 \mid f \; K-Isom.\}$ (Gruppe der K-Automorphismen mit Verknüpfung als Operation).
\end{definition}
\begin{remark}[Beobachtung]\label{theo:2.27}
    Sei $\phi: L_1 \rightarrow L_2$ K-Hom. Sei $f \in K[X]$, $\alpha \in L_1$ mit $f(\alpha) = 0$. Dann ist $f(\phi(\alpha)) = \phi(f(\alpha)) = \phi(0) = 0$. 
    Folgerungen: $\alpha$ transzendent, $\phi$ K-Isom. $\Rightarrow \phi(\alpha)$ transzendent; für algebraisches $\alpha$ ist $m_{\alpha, K} = m_{\phi(\alpha), K}$.
\end{remark}
\begin{example*}
    $Aut(\mathbb{C}\mid\mathbb{R}) = \{id, \tau\}$ mit $\tau$ komplexe Konjugation. Denn: $\mathbb{C} = \mathbb{R}[i], m_{i, \mathbb{R}} = X^2 + 1$ mit Nullstellen $i$, $-i$. Jeder $\mathbb{R}-Aut.$ bildet $i$ auf $i$ (id) oder $-i$ ($\tau$) ab.
\end{example*}
\begin{lemma} \label{theo:2.28}
    Seien $K$, $K'$ zwei Körper und $\sigma: K \rightarrow K'$ ein Isomorphismus. Sei $K(\alpha) \mid K$ eine einfache algebraische $K$-Erweiterung. Sei $L' \mid K'$ eine $K'$-Erweiterung. Für jede Nullstelle $\alpha' \in L'$ von $\sigma_*(m_{\alpha, K}) \in K'[X]$ ($\sigma_*$ wendet $\sigma$ auf die Koeffizienten an) gibt es genau einen Homomorphismus $\phi: K(\alpha) \rightarrow L'$ mit $\phi_{|K} = \sigma$ und $\phi(\alpha) = \alpha'$. Dann ist $\phi$ Isomorphismus zwischen $K(\alpha)$ und $K'(\alpha')$.
\end{lemma}
\begin{proof}
    Kommutatives Diagramm
\end{proof}
\begin{remark}
    Die Anzahl der Homomorphismen $\phi$ wie im vorherigen Lemma ist genau die Anzahl der Nullstellen von $\sigma_*(m_{\alpha, K})$ in $L'$.
\end{remark}
\begin{example}
    Sei $d \neq 1$ eine quadratfreie ganze Zahl. $Aut(\mathbb{Q}(\sqrt{d}) \mid \mathbb{Q}) = \{id, \sigma\}$, $m_{\sqrt{d},\mathbb{Q}} = X^2 - d$ mit $\sigma(\sqrt{d}) = -\sqrt{d}$.
\end{example}
\begin{theorem}[Fortsetzungssatz] \label{theo:2.31} $ $
    \begin{enumerate}[label=(\alph*)]
        \item Sei $L \mid K$ eine alg. $K$-Erweiterung, $M$ ein alg. abgeschlossener Körper und $\sigma: K \rightarrow M$ ein Homomorphismus. Dann existiert $\phi: L \rightarrow M$ mit $\phi_{|K} = \sigma$.
        \item Sei $\sigma: K \rightarrow K'$ ein Isomorphismus von Körpern. Seien $\overline{K}, \overline{K'}$ alg. Abschlüsse von $K$ bzw. $K'$. Dann ex. ein Isomorphismus $\phi: \overline{K} \rightarrow \overline{K'}$ mit $\phi_{|K} = \sigma$. (Je zwei algebraische Abschlüsse eines Körpers K sind isomorph.)
    \end{enumerate}
\end{theorem}
\begin{proof} $ $
    \begin{enumerate}[label=(\alph*)]
        \item 
        Sei $$\mathcal{U} = \{(F,\tau) \mid K \subseteq F \subseteq L \text{ Zwischenkörper und } \tau: F \rightarrow M \text{ Hom. mit } \tau|_K = \sigma\}$$ Die Menge $\mathcal{U}$ ist partiell geordnet via 
        $$(F_1, \tau_1) \leq (F_2, \tau_2) :\Leftrightarrow F_1 \subseteq F_2 \text{ und } \tau_2|_{F_1} = \tau_1$$
        $(\mathcal{U},\leq)$ ist induktiv, d.h. jede Kette in $(\mathcal{U},\leq)$ (total geordnete Teilmenge) besitzt eine obere Schranke:
        Sei $C$ eine Kette. Setze $F_0 := \bigcup_{(F,\tau) \in C} F \subseteq L$. Für $x \in F_0$ definiere man $\tau_0(x) := \tau_F(x)$, falls $x \in F$.
        Da C eine Kette ist, ist die Definition von $\tau_0(x)$ unabhängig von der konkreten Wahl von $F$, also wohldefiniert.
        Dann ist $(F_0,\tau_0)$ eine obere Schranke von C. Lemma von Zorn\footnote{Eine halbgeordnete Menge, in der jede Kette eine obere Schranke hat, enthält mindestens ein maximales Element.} impliziert die Existenz eines max. Elementes $(F_1, \tau_1) \in \mathcal{U}$.
        Wir behaupten nun, dass $F_1 = L$. Falls nicht, also $F_1 \subset L$, sei $\alpha \in L\setminus F_1$. Dann ist $\alpha$ algebraisch über K, insb. algebraisch über $F_1$. Definiere $F_2 := F_1(\alpha)$.
        Nach \ref{theo:2.28} ex. $\tau_2: F_2 \rightarrow M$ mit $\tau_2|_{F_1} = \tau_1$, also $(F_2, \tau_2) > (F_1, \tau_1)$ - Widerspruch.
        \item 
        Nach a) gibt es $\phi: \overline{K} \rightarrow \overline{K'}$ mit $\phi_{|K} = \sigma$. Z.z. $\phi(\overline{K}) = K'$.

        Das Bild $\phi(\overline{K})$ ist ein alg. abgeschlossener Körper, da $\phi$ injektiv und damit $\phi(\overline{K})$ isomorph zu $\overline{K}$.
        $$% https://tikzcd.yichuanshen.de/#N4Igdg9gJgpgziAXAbVABwnAlgFyxMJZABgBpiBdUkANwEMAbAVxiRAB12IaYAnBrGBjAA0gHIAviAml0mXPkIoyARiq1GLNpzQALLAApO3PgKGiJASmmyQGbHgJEyAJnX1mrRCHE25DxSIVcndNLw52HBgADxxgRgBzADoAAjoAIwT4AGNdBkw4GDAJKRl-BScUYLVqDy1vTijY+IZktMycvIKi0vUYKCyEFFAAM14IAFskF2ocCCQVMpAxyYXZ+cQyDU9tSJi4xKSpagYMmAYABXlHJRBeLATdHD9l8anN9enqdKKoJABaADMWzq4Ua+xayWOIFOP0u10C3nuj2eEgoEiAA
    \begin{tikzcd}[column sep=0cm]
    \overline{K'} \arrow[dd, "\text{alg.}"', bend right=50, no head] & \text{alg. abgeschlossen} \\
    \phi(\overline{K}) \arrow[u, "\text{alg.}"', no head]         & \text{alg. abgeschlossen}  \\
    K' \arrow[u, no head]                                         &                           
    \end{tikzcd}
    $$
        Wegen \cref{theo:2.22} und der Algebraizität von $\overline{K'}|\phi(\overline{K})$ folgt $\phi(\overline{K}) = \overline{K'}$.
    \end{enumerate}
\end{proof}
\begin{flushright}
VL vom 13.11.2023:
\end{flushright}
\subsection{Zerfallskörper}

\begin{definition}
    Sei $K$ ein Körper und $\Omega \subseteq K[X]$ eine Teilmenge von nicht konstanten Polynomen. Ein Erweiterungskörper $L$ von $K$ heißt \emph{Zerfällungskörper} von $\Omega$, falls gilt
    \begin{enumerate}
        \item Jedes $f\in \Omega$ zerfällt in $L[X]$ in Linearfaktoren
        \item $L=K(S)$, wobei $S=\{x\in L \mid \exists f\in \Omega: f(x)=0\}$
    \end{enumerate}
    Eine Körpererweiterung $L|K$ heißt \emph{normal}, falls sie ein Zerfällungskörper für eine Menge $\Omega \subseteq K[X]\setminus K$ ist.
\end{definition}

\begin{remark}
    Ist $L$ ein Zerfällungskörper von $f\in K[X]\setminus K$ mit $\deg(f)=m$.
    Dann ist $L=K(\alpha_1, \dots, \alpha_r)$, wobei $\alpha_1,\dots,\alpha_r$ Nullstellen von $f$ in $L$ sind ($r\leq m$).
    Es gilt $[L:K]< m^r$.
    Tatsächlich gilt $[L:K] \leq m!$ (Übung).
    Natürlich
    \begin{align*}
        L&=K(\alpha_1,\dots,\alpha_r)\\
        |&\\
        K&(\alpha_1,\dots,\alpha_{r-1})\\
        \vdots&\\
        K&
    \end{align*}
    und $[K(\alpha_1,\dots,\alpha_i):K(\alpha_1,\dots,\alpha_{i-1})] \leq [K(\alpha_i):K] \leq m$.
\end{remark}

\begin{example}$ $
    \begin{enumerate}[label=\alph*)]
        \item $f=X^4 -2 \in \Q[X]$ irreduzibel nach Eisensteinkriterium mit $p=2$ und Nullstellen $\sqrt[4]{2},-\sqrt[4]{2},i\sqrt[4]{2}$,$-i\sqrt[4]{2}$
        \begin{align*}
            L=\Q&(\sqrt[4]{2}) \cong \Q[X]/(f) \text{hat Grad 4 über }\Q\\
            &\shortparallel\\
            \C\subseteq\Q&[\sqrt[4]{2}] = \{\sum_{j=0}^{3} a_j (\sqrt[4]{2})^j\mid a_j \in \Q\} \not\owns \pm i \sqrt[4]{2}
        \end{align*}
        daher ist $L$ kein Zerfällungskörper von $f$.

        Dagegen ist $\Q(\sqrt{2})$ ein Zerfällungskörper von $X^2 \in \Q[X]$.
        \item $f=X^4-2 \in \mathbb{F}_5[X]$ irreduzibel.
        \begin{proof}
            Angenommen $f=gh$: 
            Wenn $\deg(g)=1$, hat $f$ eine Nullstelle in $\mathbb{F}_5$, aber $X^4 \equiv 1 \mod{5}$ nach Satz von Euler\footnote{Satz von Euler: $X^{p-1} \equiv 1 \mod{p}$} und damit $\forall x\in \F_5\setminus\trivGZ\colon f(x)=X^4-2=1-2=4\neq0$.
            Wenn $\deg(g)= 2 = \deg(f)$ \TODO[was soll das für eine begründung sein?]
        \end{proof}
        Sei $\alpha\in \overline{\mathbb{F}}_5$ eine Nullstelle von $f$. $\overline{\mathbb{F}}_5$ algebraischer Abschuss von $\mathbb{F}_5$.
        Dann ist $E = \mathbb{F}_5(\alpha) \cong \mathbb{F}_5[X]/(f)$ ein Zerfällungskörper von $f$, weil:
        $f$ hat die Nullstellen $\alpha, 2\alpha, 3\alpha, 4\alpha \in E$, da $b^4\equiv 1 \mod{5}$ nach Euler für $b=2,3,4$ und damit $(b\alpha)^4 = b^4\alpha^4=\alpha^4$.

        In $E[X]$ gilt: $f = \Pi_{i=1}^4 ( X-i\alpha) \in E[X]$.
    \end{enumerate}
\end{example}

\begin{theorem} \label{theo:2.35}
    Sei $K$ Körper und $\Omega\subseteq K[X]\setminus K$
    \begin{enumerate}[label=\alph*)]
        \item Jeder alg. Abschluss von $K$ enhält genau einen Zerfällungskörper von $\Omega$.
        \item Je zwei Zerfällungskörper von $\Omega$ sind $K$-Isomorph
    \end{enumerate}
\end{theorem}
\begin{proof}
    \TODO
\end{proof}

\begin{theorem}[Charakterisierung von normalen Erweiterungen] \label{theo:2.36}
    Sei $K$ Körper mit alg. Abschluss $\overline{K}$. Für einen Zwischenkörper $K\subseteq L \subseteq \overline{K}$ sind folgende Aussagen äquivalent:
    \begin{enumerate}
        \item $L|K$ ist normal
        \item Ist $\phi: L\rightarrow \overline{K}$ ein $K$-Homomorphismus, dann ist $\phi(L)=L$
        \item Jedes irreduzible $f\in K[X]$, dass in $L$ eine Nullstelle besitzt, zerfällt in $L[X]$ in Linearfaktoren
    \end{enumerate}
\end{theorem}
\begin{proof}
    \TODO
\end{proof}

\begin{theorem} \label{theo:2.37}
    Sei $L|K$ eine normale $K$-Erweiterung
    \begin{enumerate}[label=\alph*)]
        \item Für jeden Zwischenkörper $M$ gilt $L|M$ ist normal.
        \item Sind $\alpha, \beta \in L$, dann gibt es $\sigma\in Aut(L|K)$ mit $\sigma(\alpha)=\beta$ gdw $m_{\alpha, K} = m_{\beta, K}$
    \end{enumerate}
\end{theorem}
\begin{proof}
    \TODO
\end{proof}

\begin{flushright}
VL vom 13.11.2023:
\end{flushright}

\subsection{Serperable Erweiterungen}
\begin{definition}
    Ein irreduzibles Polynom in $K[X]$ heißt \emph{separabel}, wenn es im $\overline{K}$ nur einfache Nullstellen hat. (allgemein heißt ein Polynom separabel, wenn alle irreduzieblen Faktoren separabel sind)
\end{definition}

\begin{definition}
    Die $K$-lineare Abbildung $D:K[X]\rightarrow K[X], \sum a_ix^i \mapsto \sum ia_iX^{i-1}$ heißt \emph{(formal) Ableitung}.
    Es gilt die Leibnizregel $D(fg) = D(f)g+fD(g)$.\footnote{Präzieser $\sum a_ix^i \mapsto \sum \pi(i)a_iX^{i-1}$ mit $\pi\colon \Z\rightarrow K$ definiert durch $1\mapsto 1_K$. Daher ist die $Char(K)$ auch relevant.}
\end{definition}
\begin{theorem}
    Ein irreduzibles Polynom $f$ ist genau dann separabel, wenn $D(f)\neq 0$.
\end{theorem}
(Die mehrfachen Nullstellen eines beliebigen Polynoms $f$ sind die gemeinsamen Nullstellen von $f$ und $D(f)$)
\begin{example}
    $K = \F_p(T) = Quot(\F_p[T])$ (rationaler Funktionenkörper)
    Betrachte $f=X^p-T\in K[X]$.
    Nach Eisenstein ist $f$ irreduzibel.
    Es ist $D(f)=pX^{p-1} = 0$.
    Also ist $f$ nicht separabel.
    
    Sei $a=\sqrt[p]{T}\in \overline{K}$ eine Nullstelle von $f$.
    Dann gilt $(X-a)^p =f$.
\end{example}
\begin{proof}[Beweis zu 2.40]
    \begin{remark*}
        Schreibe $f=c\Pi_{j=1}^d (X-a_j)\in \overline{K}[X]$ mit $0\neq c$, $a_1,\dots, a_d\in \overline{K}$.
        Dann ist $D(f) = c\sum_{j=1}^d \Pi_{i\neq j} (X-a_i)$ (Leibnizregel).
    \end{remark*}
    Damit folgt $D(f)(a_k)=c\Pi_{i\neq k} (a_k-a_i)$.

    \begin{itemize}
        \item["'$\Rightarrow$"'] Durch Kontraposition: Sei $D(f)=0$. Dann $D(f)(a_1)=0$. Dann $\exists i\neq1: a_i=a_1$, was \Lightning sep.
        \item["'$\Leftarrow$"'] Sei $D(f)\neq 0$.
        Wegen $\deg(D(f)) < \deg(f)$ und $f$ irreduzibel, sind $D(f)$ und $f$ teilerfremd.
        Es gibt also $g,k\in K[X]$ mit $1=gf+hD(f)$.
    
        Sei $a_k$ eine der Nullstellen von $f$ in $\overline{K}$. Dann ist
        $$1=\equalto{g(a_k) f(a_k)}{0} + \notequalto{h(a_k) D(f)(a_k)}{ 0}$$
        Wegen (*) muss die Nullstelle $a_k$ einfach sein.
    \end{itemize}
\end{proof}

\begin{remark}
    Ist $Char(K)=0$, dann ist jedes irreduzible Polynom separabel.\footnote{Editor's remark: $Char(K)=0$ verhindert, dass die Faktor $i$ in der Ableitung $0$ werden kann und damit $D(f)=0$ werden könnte.}
\end{remark}

\begin{definition} \label{theo:2.43}
    Sei $L|K$ eine algebraische $K$-Erweiterung.
    Ein Element $a\in L$ ist separabel, wenn $m_{a,K}$ separabel ist.
    Sind alle $a\in L$ separabel, dann nennt man $L|K$ separabel.
\end{definition}

\begin{lemma} \label{theo:2.44}
    Ist $L|K$ separabel und $K\subseteq M \subseteq L$ ein Zwischenkörper, dann ist $L|M$ und $M|K$ separabel.
\end{lemma}
\begin{proof} $ $
    \begin{enumerate}
        \item[$M|K$:] Sei $a\in M$. Die Minimalpolynome über $K$ von $a$ als Element von $M$ und $L$ sind gleich. Also ist $a$ separabel.
        \item[$L|M$:] Sei $a\in L$. Dann ist $m_{a,M}$ ein Teiler von $m_{a,K}$ in $\overline{K}[X]$.
        daher hat auch $m_{a,M}$ nur einfache Nullstellen.
    \end{enumerate}
\end{proof}

\begin{definition}
    Sei $L|K$ eine alg. $K$-Erweiterung.
    Der \emph{Separabilitätsgrad} $[L:K]_S$ über $K$ ist definiert als $\card{Hom_K(L,\overline{K})}$.
    Ist $L|K$ normal, dann ist $[L:K]_s = \card{Aut(L|K)}$.\footnote{Nach \cref{theo:2.35} b)} 
\end{definition}

\begin{lemma}\label{theo:2.46}$ $
    \begin{enumerate}[label=\alph*)]
        \item Sind $M|L$ und $L|K$ alg. Erweiterungen, dann ist $$[M:K]_S = [M:L]_S \cdot [L:K]_S$$
        \item Ist $L:K$ endlich, so ist $[L:K]_S \leq [L:K]$
    \end{enumerate}
\end{lemma}
\begin{proof}$ $\\
    $\begin{tikzcd}
    \overline{K} \arrow[rd, "\overline{\sigma}_i \text{ $K$-Hom}", bend left] \arrow[d, no head] &\\
    M \arrow[r, "\tau_j", "\text{$L$-Hom}"'] \arrow[d, no head] & \overline{K} \\
    L \arrow[ru, "\sigma_i \text{$K$-Hom}"', bend right] \arrow[d, no head] &\\
    K&
    \end{tikzcd}$
    \begin{minipage}{0.8\textwidth}
        Wir betten K, M, L in einen gemeinsamen alg. Abschluss $\overline{K}$ ein.
        Seien $(\sigma_i)_{i\in I}$ die paarweise verschiedenen $K$-Homomorphismen $L\rightarrow \overline{K}$.
        Seien $(\tau_j)_{j\in J}$ die paarweise verschiedenen $L$-Homomorphismen $M\rightarrow \overline{K}$.
        Also $\card{I} = [L:K]_S$ and $\card{J}=[M:L]_S$.
        Die $K$-Homomorphismen $\overline{\sigma_i} \circ \tau_j = f_{ij}$ sind genau die paarweise verschiedenen $K$-Homom. 
        $M\rightarrow \overline{K}$, wobei $\overline{\sigma_i}$ eine Fortsetzung von $\sigma_i$ zu einem $K$-Homomorphismus $\overline{K} \rightarrow \overline{K}$ ist (Fortsetzungssatz \cref{theo:2.31}). 
        Angenommen $\overline{\sigma_i} \circ \tau_j = \overline{\sigma_s} \circ \tau_r$, dann ist $\sigma_i = \overline{\sigma_i}_{|L} = (\overline{\sigma_i} \circ \tau_j)_{|L} =(\overline{\sigma_s} \circ \tau_s)_{|L} = \sigma_s$, also $i=s$.
        Da $\overline{\sigma_i},\overline{\sigma_s}$ automatisch injektiv sind\footnotemark, ist $\tau_j = \tau_r$ also $j = r$.
        Damit ist a) bewiesen.
    \end{minipage}
    \footnotetext{Alle Körperhomomorphismen sind $0$ oder injektiv. Wäre $\phi(a)=0$, dann $0 = \phi(a) = \phi(a)\phi(a^{-1}) = \phi(1)=1$}%Otherwise the minipage does stupid shit

    Zu b)
    Es gilt $L=K(a_1, \dots, a_n)$ für gewisse $a_i\in L$. Wegen a) wird der Gradformel \cref{theo:2.12} genügt es $L=K(a)$ zu betrachten.
    Nach \cref{theo:2.28} ist $[K(a):K]_S=\card{\{b\in \overline{K}\mid m_{a,k}(b)=0\}} \leq \deg(m_{a,K}) = [K(a):K]$
\end{proof}

\begin{theorem}[Char. separabler Erweiterungen] \label{theo:2.47}
    Sei $L|K$ eine endliche Erweiterung. Dann sind äquivalent:
    \begin{enumerate}[label=(\roman*)]
        \item $L|K$ separabel
        \item $L=K(a_1, \dots, a_n)$ für über $K$ separable Elemente $a_1, \dots, a_n \in L$
        \item $[L:K]_S = [L:K]$
    \end{enumerate}
\end{theorem}
\begin{proof}
$ $
    \begin{itemize}[align= left]
        \item[(i) $\rightarrow$ (ii)] $[L:K] < \infty$, dann gilt $a_1, \dots, a_n \in L$ mit $L=K(a_1, \dots, a_n)$. Diese Elemente sind (nach Definition separabler Körpererweiterungen \ref{theo:2.43}) automatisch separabel.
        \item[(ii) $\rightarrow$ (iii)] Wegen \cref{theo:2.46} a) and \cref{theo:2.12} reicht es (iii) für den Fall $L=K(a)$ zu zeigen.
        $$[K(a):K]_S=\card{\{b\in \overline{K}\mid m_{a,K}(b)=0\}} \overset{a\text{ sep.}}{=}\deg(m_{a,K}) = [K(a):K]$$
        \item[(iii) $\rightarrow$ (i)] Sei $a\in L$. Dann gilt
        \begin{align*}
            [L:K] = [L:K]_S &= [L:K(a)]_S \cdot [K(a):K]_S\\
            &\leq [L:K(a)] \cdot [K(a):K]\\
            &= [L:K]
        \end{align*}
        Damit ist $\card{\{b\in \overline{K}\mid m_{a,K}(b)=0\}} = [K(a):K]_S = [K(a):K] = \deg(m_{a,K})$ und $a$ ist separabel.
    \end{itemize}
\end{proof}
\begin{corollary}\label{theo:2.48}
    Ist $f\in K[X]\setminus K$ ein separables Polynom, so ist der Zerällungskörper von $f$ separabel
\end{corollary}


\begin{flushright}
VL vom 23.11.2023:
\end{flushright}
\subsection{Endliche Körper}
Ziel: Konstruktion eines Körpers $\F_q$, $q=p^n$, $p$ prim mit $q$ Elementen. Nicht zu verwechseln mit $\Z/q\Z$, der für $n>1$ Nullteiler besitzt.
\begin{lemma} \label{theo:2.49}
    Es sei $\F$ ein endlicher Körper. Dann gilt $p=char(\F)>0$. Somit $\card{\F} = q \coloneqq p^n$ für $[\F:\F_p] = n$. Es ist $\F$ der Zerfällungskörper des Polynoms $X^q-X$ über $\F_p$. Insbesondere ist die Erweiterung $\F|\F_p$ normal.
\end{lemma}
\begin{proof}
    Mit $\F$ ist auch der Primkörper\footnote{Kleinster Teilkörper eines Körpers. Er wird von $0$ und $1$ durch Abschluss von Multiplikation, Addition und der Inversen erzeugt. Er ist isomorph zu $\Q$, wenn $char(K)=0$, oder zu $\F_{char(K)}$, wenn $char(K) > 0$.} endlich, also von der Form $\F_p$. Daher $\card{\F} = \card{\F_p}^{[\F:\F_p]} = p^n$.

    Die multiplikative Gruppe $\F^\times$ hat die Ordnung $q-1$, daher ist jedes Element in $\F^\times$ Nullstelle von $X^{q-1}-1$. Also ist jedes Element von $\F$ Nullstelle von $X^q-X$.
    Insbesondere ist $\F$ Zerfällungskörper von $X^q-X$.
\end{proof}

\begin{theorem}\label{theo:2.50}
    Es sei $p$ eine Primzahl. Dann existiert zu jedem $n\in \N$ eine Erweiterung $\F_q|\F_p$ mit $q=p^n$ Elementen. Es ist $\F_q$ bis auf Isomophie der eindeutige Zerfällungskörper von $X^q-X\in \F_p[X]$.
    Es besteht $\F_q$ genau aus den Nullstellen von $X^q-X$.
    Jeder endliche Körper ist bis auf Isomorphie ein Körper des Typs $\F_q$.
\end{theorem}
\begin{proof}
    Die Eindeutigkeitsaussagen folgen aus dem Lemma.
    Sei $f:= X^q-X$. Wegen $D(f)=-1$ hat das Polynom nur einfache Nullstellen, also $q$ einfache Nullstellen in einem algebraischen Abschluss $\overline{\F}_p$ von $\F_p$.
    Diese Nullstellen bilden einen Teilkörper von $\overline{\F}_p$:


    Sei $a,b\in \overline{\F}_p$ Nullstellen. Dann $(a\pm b)^q =\sum_{i=0}^q \binom{q}{i}a^ib^{q-i} \overset{char = p}{=} a^q\pm b^q$ also ist $a\pm b$ wieder eine Nullstelle.

    $(ab^{-1})^q = a^q (b^q)^{-1} = ab^{-1}$ also ist $ab^{-1}$ wieder eine Nullstelle.
    D.h. die Nullstellen von f sind der Zerfällungskörper von $f$. Er hat $q$ Elemente.
\end{proof}

\begin{remark}
    Sei $K$ ein Körper der Charakteristik $p>0$.
    Das Argument des letzten Beweises impliziert, dass $$\{x^{p^n}\mid x\in K\}$$ ein Teilkörper von $K$ ist.
\end{remark}
\begin{corollary}
    Man bette die Körper $\F_q$, $ q=p^n$ in einen algebraischen Abschluss $\overline{\F}_p$ von $\F_p$ ein.
    Es ist $\F_q \subseteq \F_{q'}$, $(q = p^n, q'=p^{n'})$genau dann, wenn $n|n'$.
    Die Erweiterung $\F_{q'}|\F_q$ sind bis auf Isomorphie die einzigen Erweiterungen zwischen endlichen Körpern der Charakteristik $p$.
\end{corollary}
\begin{proof}
    Es gelte $\F_q \subseteq \F_{q'}$. Sei $m:= [\F_{q'}:\F_q]$. Dann $p^{n'} = \card{ \F_{q'}} = \card{\F_q}^m = (p^n)^m = p^{n\cdot m}$, also $n|n'$.
    Gilt umgekehrt $n'=n\cdot m$, so folgt für $a\in \overline{\F}_p$ aus $a^q=a$ stets $a^{q'}=a^{q^m} = a$.
    Wegen des Fortsetzungssatzes kann man jede Erweiterung $L|\F$ von endlichen Köpern der Char. $p$ in $\overline{\F}_p$ realisiert werden.
    Die Eindeutigkeit folgt dann mit dem schon Gezeigten and  dem vorherigen Satz.
\end{proof}
Ein Körper $K$ ist \emph{perfekt}, wenn jede alg. Erweiterung von $K$ separabel ist.
\begin{enumerate} %todo with points
    \item $char(K)=0$, d.h. $\Q \subseteq K$. Dann ist $K$ perfekt, weil jedes irreduzible Polynom $f$ über $K$ separabel ist. (denn $D(f) = 0$) Es gibt aber alg., nicht normale Erweiterungen von $\Q$ (Bsp. \cref{theo:2.36} \TODO[does this reference make sense???])
    \item $\F_p(t)$ ist nicht perfekt. Die Erweiterung $\F_p(t)[t^{\frac{1}{p}}] = \F_p(t^{\frac{1}{p}})$ ist normal aber nicht separabel. (Bsp. 2.41)
    Sei $p=5$. Betrachte die alg. Erweiterung
    $$\begin{tikzcd}[row sep=0.4cm,column sep=0cm]
\F_5(t^{\frac{1}{15}}) \arrow[d, no head] \arrow[dd, "\substack{\text{nicht normal} \\ \text{nicht separabel}}"', bend right, shift right=2, no head] &\cong \F_5(t^{\frac{1}{5}})/(X^3-t^{\frac{1}{5}})\\
\F_5(t^{\frac{1}{5}}) \arrow[d,"\text{nicht separabel}", no head] &\cong \F_5(t)/(X^5-t)\;\;\;\;\\
\F_5(t)&
\end{tikzcd}$$
    Warum ist $\F_5(t^{1/15})|\F(t)$ nicht normal?
    Ansonsten wären alle Nullstellen von $X^3-t^{1/5}$ in $\F_5(t^{1/15})$.\footnote{man bekommt die dritten Einheitswurzeln aus den nullstellen raus}
    Somit auch alle Nullstellen von $X^3-1= (X-1)(X^2+X+1)$.
    Jedes Element von $\F_p(t^{1/15})$ ist von der Form $\frac{p(t^{1/15})}{q(t^{1/15})}$ mit $p,q \in \F_5[X]$ teilerfremde Polynome.
    Da $t^{1/15}$ transzendent über $\F$, ist die Evaluationsabbildung $$\F_5[Y]\overset{ev}{\rightarrow} \F_5(t^{1/15})$$ injektiv, erwertert sich also auf den Quotientenkörper $$\F_5(Y)\rightarrow \F_5(t^{F/15})$$ injektiv.
    Also muss es ein $f\in \F_5(Y)$ geben mit $f^2+f+1=0$.
    Sei $g=f-2\in \F_5(Y)$.
    Dann $0=(g+2)^2+(g+2)+1 = g^2 +4g+4+g+2+1= g^2 +2$ also $g^2=-2=3$ undn $g\in \F_5$.
    Das ist ein Wiederspruch: $3$ ist kein Quadrat mod $5$.
\end{enumerate}

\begin{corollary}
    Jede algebraische Erweiterung eines endlichen Körpers ist normal und separabel. Insbesondere ist jeder endliche Körper perfekt.
\end{corollary}
\begin{proof}
    Sei $K|\F$ eine alg. Erweiterung von $\F$ endlicher Körper mit Char $p>0$. Sei zunächst $K|\F$ endlich und damit auch $K$ endlich.
     Da $f=X^q-X$ separabel und $K$ Zerfällungskörper von $f$ über $\F_p$ für ein $q=p^n$ ist, ist $K|\F_p$, insbesondere auch $K|\F$, normal und separabel.

     Allgemein lässt sich $K$ durch endliche Erweiterungen ausschöpfen.
\end{proof}
\begin{theorem}
    $Aut(\F_{p^n}| \F_p)$ ist zyklisch von Ordung $n$. Sie wird erzeugt vom \emph{Frobenius-Automorphismus}:
    \begin{align*}
        Fr: \F_{p^n} &\overset{\cong}{\rightarrow}\F_{p^n}\\
        x&\mapsto x^p
    \end{align*}
\end{theorem}
\begin{proof}
    Fr ist Homomorphismus
    Fr injektiv, also bijektiv, weil $\F_{p^n}$ endlich ist.
    Fr Erzeuger: Angenommen $Fr^m = id$ für $1\leq m<n$. Dann $X^{p^m} = X$ für alle $X\in \F_{p^n}$. Dann hätte $X^{p^m}-X$ mehr als $p^m$ Nullstellen. (Wiederspruch)
    Fr hat Ordnung $n$: $\card{\Fpn | \F_p} = [\Fpn :\F_p]_s = [\Fpn:\F_p] = n$
\end{proof}
\begin{flushright}
VL vom 24.11.2023:
\end{flushright}
\begin{theorem} \label{theo:2.55}
    Eine endliche Untergruppe der multiplikativen Gruppe eines beliebigen Körpers ist zyklisch.
\end{theorem}
\begin{proof}
    Sei $H\leq K^\times$ eine endliche Untergruppe.
    Sei $a\in H$ ein Element maximaler Ordnung $m$ in $H$.
    Sei $H_m := \{b\in H\mid ord(b)|m\} \subseteq H$.
    Die Elemente von $H_m$ sind Nullstellen des Polynoms $X^m-1\in K[X]$.
    Somit $\card{H_m}\leq m$. Wegen $a\in H_m$, somit $\langle a\rangle\subseteq H_m$, ist $\card{H_m}=m$.
    Ang. es gibt $b\in H\setminus H_m$. D.h. $ord(b)\nmid m$.
    Dann gilt $ord(ab)= kgV(ord(b),m) >m$.
    \Lightning zur Maximalität von $a$.
\end{proof}

\begin{theorem}[Satz vom primitiven Element] \label{theo:2.56}
    Sei $L|K$ eine endliche und separabel Körpererweiterung.
    Dann existiert ein $a\in L$ mit $L=K(a)$.
\end{theorem}
\begin{proof} $ $
    \begin{enumerate}[align=left]
        \item [1.Fall: $K$ ist ein Endlicher Körper] 
        Dann ist auch $L$ endlich. Nach \cref{theo:2.55} ist $L^\times$ zyklisch, d.h. $L^\times=\langle a\rangle$. 
        Somit $L=K(a)$.
        
        \item [2. Fall $K$ ist unendlich]
        Schreibe $L=K(a_1, \dots, a_n)$. Durch eine Induktion über $n$ reicht es den Fall $n=2$ zu zeigen.
        Sei $L = K(a_1,a_2)$ und $\phi_1,\dots,\phi_m: L\rightarrow \overline{K}$ seien die verschiedenen $K$-Einbettungen $L\rightarrow \overline{K}$, wobei $m=[L:K]_S=[L:K]$ weil separabel.
        Das Polynom $$g = \prod_{i<j} \left(\left(\phi_i\left(a_1\right)-\phi_j\left(a_1\right)\right)\cdot X + \left(\phi_i\left(a_2\right)-\phi_j\left(a_2\right)\right)\right)\in \overline{K}[X]$$
        Für $i<j$ ist $\phi_i(a_1)\neq \phi_j(a_1)$ oder $\phi_i(a_2)\neq\phi_j(a_2)$.
        Somit $g\neq 0$.
        Da $K$ unendlich, gibt es ein $c\in K$ mit $g(c)\neq 0$.
        Also $(\phi_i(a_1)-\phi_j(a_1))\cdot c + (\phi_i(a_2)-\phi_j(a_2)) \neq 0$ für alle $i<j$.
        Das ist $\phi_i(a_1 \cdot c + a_2)- \phi_j(a_1\cdot c + a_2) \neq 0$ und $\phi_i(a) \neq \phi_j(a)$ mit $a := a_1c+a_2\in L$ für $i<j$.
        Damit sind $\phi_1(a),\dots,\phi_m(a)$ sind verschiedene Nullstellen von $m_{a,k}$.
        Damit $[L:K] \geq [K(a):K] = \deg(m_{a,K}) \geq = m = [L:K]$, also $[L:K] = [K(a):K]$ und $L=K(a)$
    \end{enumerate}
\end{proof}
\begin{example}$ $\\
$% https://tikzcd.yichuanshen.de/#N4Igdg9gJgpgziAXAbVABwnAlgFyxMJZABgBpiBdUkANwEMAbAVxiRAGkAKAHW7gEcATjmRoKwOAF9SvAcNHickgJQhp6TLnyEUZAIxVajFmy6yhIsRJVrSG7HgJEyAJkP1mrRBzWGYUAHN4IlAAM0EIAFskPWocCCQXag8TbzRbMIjoxDIQeJjk4y8QdMkKSSA
\begin{tikzcd}
{K(\sqrt[p]{s},\sqrt[p]{t})} \arrow[d, "p"] \\
{K(\sqrt[p]{s})} \arrow[d, "p"]             \\
K                                          
\end{tikzcd}$
\begin{minipage}{0.8\textwidth}
Die Separabilität im Satz \cref{theo:2.56} ist essenziell.
    $K=\F_p(s,t)$ rationaler Funktionenkörper in zwei Variablen.
    $L=K(\sqrt[p]{s},\sqrt[p]{t})$ und damit $[L:K] = p^2$.
    
    Sei $a\in L$.
    Schreibe $a = \sum_{l,k} a_{l,k} (\sqrt[p]{s})^l(\sqrt[p]{t})^k$ mit $a_{l,k}\in K$.
    Dann ist $a^p=\sum_{l,k} a_{l,k}^p s^l t^k =: c\in K$, da $(*)^p$ eine Homomorphismus ist.
    Somit ist $a$ Nullstelle von $X^p-c\in K[X]$ und $\deg(m_{a,K}) \leq p$.
    Damit ist $K(a)\subset L$.
\end{minipage}
\end{example}


\end{document}