\documentclass[../main.tex]{subfiles}
\graphicspath{{\subfix{../images/}}}
\begin{document}
Alle Ring sind kommutativ.
\subsection{Ganze Ringerweiterungen}
\begin{definition}
    Sei $B$ ein Ring und $A\subseteq B$ ein Unterring. Ein Element $x\in B$ heißt \emph{ganz} über $A$, wenn es ein \emph{normiertes} Polynom $f\in A[X]\setminus \trivGZ$ gibt mit $f(x)=0$, d.h. $x^n + a_{n-1}x^{n-1} + \dots + a_0 = 0$ mit $a_i\in A$.
\end{definition}
\begin{theorem}[Charakterisierung ganzer Elemente]
    $A\subseteq B$ Unterring. Für $x\in B$ sind äquivalent:
    \begin{enumerate}[label=(\roman*)]
        \item $x$ ist ganz über $A$
        \item Der Ring $A[x]$\footnote{kleinester Unterring von $B$, der $A$ und $x$ enthält} ist als $A$-Modul endlich erzeugt
        \item $A[x]$ ist in einem Unterring $C\subseteq B$ enthalten, der als $A$-Modul endlich erzeugt ist.
    \end{enumerate}
\end{theorem}
\begin{proof}
    \begin{itemize}
        \item[(i) $\Rightarrow$ (ii)]
        Sei $x$ ganz, d.h. $x^n = -a_{n-1}x^{n-1}-\dots - a_0\in \langle1,\dots x^{n-1}\rangle_A$.
        Mit Induktion über $k\in \N$ sieht man , dass $x^{n+k}\in \langle 1,\dots, x^{n-1}\rangle_A$.
        => $A[x]$ ist endlich erzeugt als $A$-Modul!
        \item[(ii) $\Rightarrow$ (iii)]
        Nimm $C = A[x]$
        \item[(iii) $\Rightarrow$ (i)]
        Sei $\{c_1,\dots,c_n\}$ ein $A$-Erzeugendensystem von $C$.
        Wegen $x\in C$ ist $xc_i\in C$ für alle $i$.
        Also ist $xc_i= \sum_{j=1}^n \gamma_{ij}\cdot c_j$ für geeignete $\gamma_{ij}\in A$.
        Die Matrix $T = x\cdot I_n- (\gamma_{ij})_{i,j\in \{1,\dots,n\}}\in M_n(A[X])$.
        erfüllt $T\cdot (Spaltenvektor c_1 \dots c_n) = 0$.

        Sei $T^{adj}$ die Anjunkte\footnote{\TODO[Definition :D]} zu $T$.
        Es gibt $T^{adj}\dots T = \det(T)\cdot I_n$.
        => $\dot(T)\cdot (c_1 \dots c_n) = T^{adj} \cdot T\cdot (c_1\dots c_n) = 0$
        => $\det(T)\cdot c_i = 0$ für jedes $i\in \{1,\dots,n\}$.
        => $\det(T)\cdot c = 0$ für jedes $c\in C$.
        $\overset{c=1}{\Rightarrow}$ $\det(T) = 0$
        $= f(\mathcal{x}) = \det(\mathcal{x}\cdot I_n - (\gamma_{ij})\in A[\mathcal{x}]$ (das math x soll evtl ein groß X sein)
        => $g$ ganz
    \end{itemize}
\end{proof}
\end{document}