\documentclass[../main.tex]{subfiles}
\graphicspath{{\subfix{../images/}}}
\begin{document}
\begin{flushright}
VL vom 11.1.2024:
\end{flushright}
\subsection{Beträge}
\begin{definition} % \gaf Generic absolut function
    Ein \emph{Betrag} auf einem Körper $K$ ist eine $\gaf: K\rightarrow \R_{\geq0}$ mit den Eigenschaften
    \begin{enumerate}[label=\roman*)]
        \item $\card{x} = 0 \Leftrightarrow x =0$
        \item $\card{x\cdot y} = |x| \cdot |y|$
        \item $|x+y| \leq |x| + |y|$
    \end{enumerate}
    Aus ii) folgt $|1|\cdot |1| =|1| = 1$.
\end{definition}
\begin{example}$ $
    \begin{enumerate}[label=\roman*)] %i ii
        \item $\R$ mit dem gewöhnlichen Absolutbetrag
        \item $K$ beliebig. Dann definiert 
        $$|x|_{trivial} = \begin{cases}
            1 & falls x\neq 0\\
            0 & falls x= 0
        \end{cases}$$
        ein Betrag (genannt: \emph{trivialer Betrag})
        \item $\C$ mit dem komplexen Betrag
        \item Ist $\gaf$ ein Betrag auf $L$ und $K$ ein Teilkörper von $L$, so ist $\gaf_{|K}$ ein Betrag auf $K$.
    \end{enumerate}
\end{example}
\begin{definition} [p-adischer Betrag]
    Sei $a\in \Z$ und $p$ prim. Dann ist
    $\nu_p(a)= \sup\{e\in \N\mid p^e|a\}\in \N_0\cup \{\infty\}$
    die \emph{p-adische Betrag}.
Beachte $\nu_p(0) = \infty$.
\end{definition}
Damit gilt:
$$a = \pm \prod_{p\text{ prim}} p^{\nu_p(a)}$$
Die Funktion $\nu_p:\Z\rightarrow \N_0\cup \{\infty\}$ heißt \emph{p-adische Bewertung}. Sie hat die folgenden Eigenschaften
\begin{enumerate}[label=\roman*)]%i, ii
    \item $\nu_p(a) = \infty \Leftrightarrow a = 0$
    \item $\nu_p(ab) = \nu_p(a)+\nu_p(b)$
    \item $\nu_p(a+b) \geq \min\left\{\nu_p(a),\nu_p(b)\right\}$
\end{enumerate}

Sei nun $x\in \Q^\times$. Schreibe
$$x=p^m\cdot\frac{a}{b}$$ mit $m\in \Z$, $p\nmid a$, $p\nmid b$.
Definiere $\nu_p(x) = m$ und $\nu_p(0) = \infty$. Dies setzt die $p$-adische Berwertung von $\Z$ auf $\Q$ fort mit den gleichen Eigenschaften.
Setze $|x|_p \coloneqq p^{-\nu_p(x)}$ mit der Konvention $|0|_p = p^{-\infty} = 0$.
Dann definiert $\gaf_p$ einen Betrag auf $\Q$, den \emph{p-adischen Betrag}\footnote{Zum 2ten mal}. Die Eigenschaften i), ii) sind offensichtlich.
Zu iii):
\begin{align*}
    |x+y|_p &= p^{-\nu_p(x+y)}\\
    &\leq p^{-\min\{\nu_p(x),\nu_p(y)\}} \\
    &= \max\{p^{-\nu_p(x)}, p^{-\nu_p(y)}\} \\&
    = \max\{|x|_p,|y_p|\}
\end{align*}
%Multiline???
Dies wird \emph{ultrametrische Dreiecksungleichung} genannt.
\begin{definition}
    Eine Bewertung $\nu$ auf einem Körper $K$ ist eine Funktion $$\nu:K\rightarrow \R\cup \{\infty\}$$
    mit folgenden Eigenschaften
    \begin{enumerate}[label=\roman*)]
        \item $\nu(x) = \infty \Leftrightarrow x= 0$
        \item $\nu(x\times y) = \nu(x)+\nu(y)$
        \item $\nu(x+y) \geq \min\{\mu(x), \nu(y)\}$ 
    \end{enumerate}
\end{definition}
\begin{lemma}
    Seien $K$ ein Körper, $\nu$ eine Bewertung auf $K$ und $a\in \R_{>1}$. Dann definiert $|x| = a^{-\nu(x)}$ einen Betrag auf dem die ultimetrische Dreiecksungleichung $|x+y| \leq \max\{|x|,|y|\}$ erfüllt.
\end{lemma}
\begin{proof}
    analog zu p-adischen Betrag
\end{proof}
\begin{example}
    Sei $K$ ein Körper. Sei $K(T) = Quot(K[T])$.
    Sei $K(T)\setminus\{0\}\ni x = \frac{f}{q}$, $f,g\in K[T]$.
    Setze $\nu_{deg}(x) = -\deg(f)+\deg(g)$ und $\nu_{deg}(0) = \infty$.
    Dies definiert eine Bewertung auf $K(T)$
\end{example}

\begin{definition}[Metrik und Topologie]
    Sei $K$ ein Körper mit Betrag $\gaf$.
    Dann definiert $d(x,y)= |x-y|$ eine Metrik auf $K$.
    Insbesondere erhalten wir eine Topologie auf $K$ vermöge 
    $$U\text{ offen }:\Leftrightarrow\forall {u\in U} \exists {\epsilon >0}: B_{\epsilon}^{\gaf} = \{v\in K\mid |u-v|<\epsilon\}\subseteq U$$
\end{definition}
\begin{remark}
    Die Addition und Multiplikation definieren stetige Abbildungen $K\times K \rightarrow K$.
\end{remark}
\begin{definition}
    Zwei Bewertungen $\gaf_1$, $\gaf_2$ sind \emph{äquivalent}, wenn sie diesselbe Topologie auf $K$ induzieren.
\end{definition}
\begin{lemma}
    Für zwei Beträge $\gaf_1$, $\gaf_2$ auf $K$ sind folgende Aussagen äquivalent (schreibe $\gaf_1 \sim\gaf_2$):
    \begin{enumerate}[label=\alph*)]
        \item $\gaf_1 \sim\gaf_2$
        \item $\forall {x\in K}: |x|_1 <1 \Leftrightarrow |x|_2<1$
        \item $\exists {s\in \R_{>0}}: |x|_1 = |x|_2^s$ für alle $x\in K$
    \end{enumerate}
\end{lemma}
\begin{proof}$ $
    \begin{itemize}
        \item[a) => b)]
        Sei $x\in K$. Dann gilt
        \begin{align*}
            \card{x}_1<1&\Leftrightarrow \card{x}_1^n\underset{n\rightarrow \infty}{\longrightarrow} 0\\
            &\Leftrightarrow \card{x^n}_1 \underset{n\rightarrow \infty}{\longrightarrow} 0\\
            &\Leftrightarrow \card{x^n}_2 \underset{n\rightarrow \infty}{\longrightarrow} 0\\
            &\Leftrightarrow \card{x}_2^n \underset{n\rightarrow \infty}{\longrightarrow} 0\\
            &\Leftrightarrow |x|_2 <1
        \end{align*}
        \TODO[Wieso folgt das in der Mitte aus äquivalenz?]
        \item[b) => c)]
        Falls $\gaf_1$ nicht trivial ist, dann gilt es ein $y\in K^\times$ mit $|y|_1 <1$.
        Somit ist $\gaf_1$ trivial gdw $\gaf_2$ trivial ist.
    
        Wir nehmen nun an, dass $\gaf_1$ nicht trivial ist.
        Sei $x\in K^\times$. Schreibe $|x|_1 = |y|_1^\alpha$ für eine $\alpha\in \R$.
        Beh: $|x|_2 = |y|_2^\alpha$.
        Sei $(\frac{m_i}{n_i})_i$ eine Folge rationaler Zahlen, die von oben gegen $\alpha$ konvergiert.
        $$|x|_1\leq |y|_1^{\frac{m_i}{n_i}} \Rightarrow |x^{n_i}|_1 < |y^{m_i}|_1 \Leftrightarrow |x^{n_i}y^{-m_i}|_1 <1$$
        Damit folgt mit b), dass $|x^{n_i}y^{-m_i}|_2 <1$ und somit (ähnlich wie für $\gaf_1$) $|x|_2\leq |y|_2^{\frac{m_i}{n_i}}$.\footnote{\TODO[Wieso gilt die Rückrichtung?]}
        Limes $i\rightarrow\infty$ ergibt, dass $|x|_2\leq |y|_2^\alpha$.
        Das selbe Argument mit von unten gegen $\alpha$ konvergierende $\frac{m_i}{n_i}$ liefert $|x|_2 \geq |y|_2^\alpha$.
        Sei $s\in \R_{>0}$ mit $|y|_2 = |y|_1^s$.
        Dann folgt:
        $$|x|_2 = |y|_2^\alpha = |y|_1^{s\cdot \alpha} = |x|_1^s$$.
        \item[c) => a)]
        Sei $\epsilon >0$ und $u\in K$.
        $$B_\epsilon^{\gaf_1}(u) = \{u\in K\mid |x-u|_1<\epsilon\} = \{x\in K\mid |x-u|<\epsilon^{-s}\} = B_{\epsilon^{-s}}^{\gaf_2}(u)$$
        => $\gaf_1\sim\gaf_2$
    \end{itemize}
\end{proof}

\begin{corollary}\label{theo:4.11}
    Seien $\gaf_1,\gaf_2$ zwei nicht triviale, nicht äquivalente Beträge auf $K$. Dann gibt es ein $u\in K$ mit $|u|_1 < 1$ und $|u|_2 > 1$.
\end{corollary}
\begin{proof}
    nach 4.10 b) finden wir $y\in K$ mit $|y|_1 <1$ und $|y|_2\geq 1$.\footnote{Sollten die Ungleichheiten vertauscht sind, kann das Inverse betrachtet werden.}
     Falls $|y|_2 = 1$, wähle $w\in K$ mit $|w|_2>1$.
     Setze $u=y^m\cdot w$ für hinreichend großes $m$ eine geeignete Wahl.
\end{proof}

\begin{theorem}[Approximationssatz]
    Seien $\gaf_1, \dots,\gaf_n$ nicht trivial, paarweise inäquivalente Beträge auf $K$.
    Für alle $a_1,\dots, a_n\in K$ und $\epsilon > 0$ existiert ein $x\in K$ mit $$|a_i-x|_i <\epsilon \text{ für alle } i\in \{1,\dots,n\}$$
\end{theorem}

\begin{flushright}
VL vom 11.1.2024:
\end{flushright}

\begin{proof}
    Behauptung: $\exists z\in K: |z|_1 > 1$ und $|z|_j<1$ für $j\in \{2,...,n\}$
    Durch Induktion: Der Induktionsanfang $n=2$ ist nach \ref{theo:4.11}\checkmark.

    Angenommen es gibt $\Tilde{z}\in K$ mit $|\Tilde{z}|_1 > 1$ und $|\Tilde{z}|_j <1$ für $j\in \{2,...,n-1\}$.
    Wähle $u\in K$ mit $|u|_1>1$ und $|u|_n<1$ nach \ref{theo:4.11}.
    Falls $|\Tilde{z}|_n<1$, dann ist $z=\Tilde{z}$ das gesuchte Element.
    Falls $|\Tilde{z}|_n = 1$, dann ist $z=u\Tilde{z}^m$ für hinreichend großes $m$ das gesuchte Element.
    <Argumentation, dass z gewünschte Eigenschaften hat...>
    Falls $|\Tilde{z}|_n > 1$, dann betrachtet man die Folge $t_m\coloneqq \frac{\Tilde{z}^m}{1+\Tilde{z}^m}$., für die die
    \begin{align*}
        t_m\rightarrow 1\quad &\text{ bez. }\gaf_1\text{ und }\gaf_n\\
        t_m\rightarrow 0\quad&\text{ bez. }\gaf_2,\dots,\gaf_{n-1}
    \end{align*}
    gelten.
    Dann ist $z=u\cdot t_m$ für hinreihend großes $m$ das gesuchte Element.
    <Argumentation, dass z gewünschte Eigenschaften hat...>

    Für ein $z\in K$ wie in der obigen beh. konvergiert die Folge $(\frac{z^m}{1+z^m})_m$ gegen $1$ bez $\gaf_1$ und gegen $0$ bez $\gaf_2,\dots,\gaf_n$.

    Für jedes $i\in \{1,\dots,n\}$ können wir somit ein $z_i\in K$ konstrieren, das sehr nahe bei $1$ bez $\gaf_i$ und sehr nahe $0$ bez der restlichen Beträge.
    Das Element $x=a_1\cdot z_1 + \dots +a_n\cdot z_n$ erfüllt die Anforderungen des Approximationssatzes. 
\end{proof}

\begin{definition}
    Ein Betrag $\gaf$ auf $K$ heißt \emph{archimedisch}, wenn 
    $$\{|n\cdot1_K| \mid n\in \N\}$$
    nicht beschränkt ist.\footnote{Nach Übung: Wenn ein Betrag $\gaf$ nicht-archimedisch, gilt $\sup\{|n\cdot1_K| \mid n\in \N\} = 1$, wegen $|1| = 1$ und der ultrametrischen Dreiecksungleichung nach \ref{theo:4.14}}
\end{definition}

\begin{theorem}\label{theo:4.14}
    Ein Betrag ist genau dann nicht-archimedisch, wenn er die ultrametrische Dreiecksungleichung erfüllt.
\end{theorem}
\begin{proof}
    \begin{itemize}
        \item[$\Leftarrow$]
        $|n\cdot 1_K|= |1_K+\dots+1_K| \leq |1_K| = 1$ also ist $\gaf$ nicht -archimedisch
        \item[$\Rightarrow$]
        Seien $x,y\in K$ mit \obda $|x|\geq|y|$.
        Sei $B>0$ eine obere Schranke für $\{|n\cdot1_K| \mid n\in \N\}$.
        \begin{align*}
            |x+y|^n &= |(x+y)^n| = |\sum_{k=0}^n \binom{n}{k} x^k y^{n-k}|\\
            &\leq \sum_{k=0}^n |\binom{n}{k}\cdot 1_K| \cdot |x|^k \cdot |y|^{n-k} \leq \sum_{k=0}^n |\binom{n}{k}\cdot 1_K| \cdot |x|^n\\
            &\leq (n+1)\cdot B\cdot |x|^n
        \end{align*}
        $\Longrightarrow\;|x+y| \leq \sqrt[n]{(n+1)B} \cdot |x| \overset{n\rightarrow \infty}{\Rightarrow} |x+y| \leq |x|$
    \end{itemize}
\end{proof}
Der Trick am Ende für 'schlechte'/ungenaue Ungleichungen:
Betrachte Potenzen und ziehe von dem dort gezeigen die Wurzel. Das kann bessere Ungleichungen bringen.
\begin{remark}$ $
    \begin{enumerate}[label=\alph*)]
        \item Ist $\gaf$ ein nicht-archimedischer Betrag, dann ist $\nu(x) = -\log(|x|)$ eine Bewertung auf Körper $K$
        \item Ist $\gaf$ nicht-archimedisch, $|x|> |y|$, dann folgt 
        $$|x|\leq |x-y+y|\leq \max\{|x-y|,|y|\} = |x-y| \leq |x|$$
        und somit Gleichheit.
    \end{enumerate}
\end{remark}

\begin{theorem}[Satz von Ostrowski]
    Jeder nicht-triviale Betrag auf $\Q$ ist äquivalent zu $\gaf_p$ für eine Primzahl $p$ oder zum Absolutbetrag $\gaf_\infty$.
\end{theorem}
\begin{proof}
    Sei $\|.\|$ ein nicht-trivialer Betrag auf $\Q$.
    \begin{itemize}
        \item[1.Fall]
        Sei $\|.\|$ nicht-archimedisch, d.h. $\|n\|\leq 1$ für alle $n\in \N$.
        Sei $\mathcal{A}\coloneqq \{n\in \Z\mid \|n\| <1\}$.
        Dann ist $\mathcal{A}$ ein Ideal in $\Z$.
        Es gibt eine Primzahl $p$ mit $p\in \mathcal{A}$, ansonsten würe $\|.\|$ trivial wegen Primfaktorenzerlegung. Also
        $$p\Z\subseteq\mathcal{A}\subset \Z$$ (da $1\notin \mathcal{A}$).
        Da $p\Z$ maximales Ideal ist, ist $p\Z = \mathcal{A}$.
        Sei $q\in \Q^\times$ und schreibe $q=p^m\cdot \frac{a}{b}$ mit $p\nmid a$, $p\nmid b$.
        Dann gilt $\|q\| = \|p\|^m$ wegen $\|a\| = \|b\| =1$.
        Sei $s>0$ so, dass $p^{-s} = \|p\|$.
        Dann folgt $\|p\| =p^{-s} = (p^{-1})^s= |p|_p^s$\\
        $\Longrightarrow\|q\| = \|p^m\| = |q|_p^s$.\\
        $\Longrightarrow\;\|.\| \sim |.|_p$

        \item[2.Fall]
        Sei $\|.\|$ archimedisch.
        Beh.: Für alle natürlichen $n,m>1$ gilt
        $$\|n\|^{\frac{1}{\log(n)}} = \|m\|^{\frac{1}{\log(m)}}\quad (*)$$
        Somit $c\coloneqq\|n\|^{\frac{1}{\log(n)}}$ unabhängig von $n>1$.
        => $\|m\| = c^{\log(m)}$ für jedes $m\in \N\setminus\{1\}$.
        Wenn wir $c = e^s$, $s>0$, schreibe, dann ergibt sich für jedes positive rationale Zahl $x=\frac{a}{b}$
        $$\|x\| = e^{s\log(x)} = |x|_\infty^s$$
    \end{itemize}
    Damit folgt $\|.\|\sim \gaf_\infty$.

\TODO[rework from here]
    Zu $(*)$:
    Schreibe $m$ zur Basis $n$ (OBdA $n<m$):
    $$m= a_0 + a_1 n+\dots+a_rn^r$$
    mit $a_i\in \{0,\dots,n-1\}$, $a_r = 0$.
    Somit $n^r\leq m$, also $r\leq \frac{\log(m)}{\log(n)}$.
    Weiter $\|a_i\| \leq a_i \|1\| = a_i<n$.
    Es folgt $$\|m\|\leq \sum_{i=0}^r \|a_i\| \cdot \|n\|^i\leq \sum_{i=0}^r n \cdot \|n\|^r \leq \left(1+\frac{\log(m)}{\log(n)}\right) \cdot n \cdot \|n\|^{\frac{\log(m)}{\log(n)}}$$
    Wir ersetzen in dieser Ungleichung $m$ durch $m^k$ und ziehen die $k$-te Wurzel:
    $$\|m\| \leq \sqrt[k]{\left(1+\frac{k\cdot\log(m)}{\log(n)}\right)}\cdot n^{1^k} \cdot \|n\|^{\frac{\log(m)}{\log(n)}}$$
    => $\|m\|^{\frac{1}{\log(m)}} \leq \|n\|^{\frac{1}{\log(n)}}$
    Da die Rollen von $m$ und $n$ vertauschbar sind, folgt Gleichheit.
\end{proof}
\subsection{Vervollständigungen}
\begin{definition}
    Ein Körper mit Betrag ist \emph{vollständig}, wenn jede Cauchyfolge in $K$ konvergiert.
\end{definition}
\begin{example}
    \begin{itemize}
        \item $(R,_\infty)$, $(\C, _\C)$ sind vollständig
        \item $(\Q, _5)$ nicht vollständig\footnote{Das Argument gilt für alle Prmzahlen $p\equiv 1\mod 4$}
        Sei $a_k + 5^k\Z$ ein Element in $(\Z/5^k\Z)^\times$ dei Ordnung $4$.\footnote{Wegen $\phi(5^k) = 5^k-5^{k-1}  = 5^{k-1} \cdot 4$ ($k\geq 2$) und der Tatsache, dass $(\Z/5^k\Z)^\times$ yzklisch ist gilt, dass $(\Z/5^k\Z)^\times \cong \Z/4\Z \times \Z/5^{k-1}\Z$. Bei Hartnick gab es mehr dazu.}
        Wir können erreichen, dass $a_k\equiv a_{k+1}\mod 5^k$
        => $|a_k-a_{k+1}|_5 < \frac{1}{5^k}$, also ist $(a_k)_k$ eine Cauchyfolge bez $_5$.

        $a_k + 5^5\Z$ hat Ordnung $2$, also $a_k^2 \equiv -1 \mod 5^k$.
        => $\lim_{k\rightarrow \infty} a_k^2 = -1$ bez $_5$.
        Also müsste ein Grenzwert $a=\lim_{k\rightarrow \infty} a_k$ in $\Q$ die Gleichung $a^2= -1$ erfüllen \Lightning
    \end{itemize}
\end{example}

\begin{flushright}
VL vom 18.1.2024:
\end{flushright}

\begin{definition}% K^ statt tilde
    Sei $(K,\gaf)$ ein Körper mit Betrag. Eine \emph{Vervollständigung} von $(K,\gaf)$ ist ein Tripel $\Tilde{K},\|.\|,i)$ bestehend aus:
    \begin{itemize}
        \item einem vollständigen Körper $\Tilde{K}$ mit Betrag $\|.\|$
        \item einem Homomorphismus $i:K (injektiver right arrow) \Tilde{K}$so, dass
        \begin{enumerate}
            \item $\|i(x)\| = |x|$ für alle $x\in K$ ("betragserhaltend")
            \item $i(k)\subseteq \Tilde{K}$ ist dicht
        \end{enumerate}
    \end{itemize}
\end{definition}

\begin{theorem}
    Jeder Körper mit Betrag $(K,\gaf)$ besitzt eine Vervollständigung $(\Tilde{K}, \|.\|,i)$, die bis auf kanonische Isomorphie\footnote eindeutig ist.
\end{theorem}
Letzteres bedeutet:
Ist $(\Tilde{K}_2, \|.\|_2,i_2)$ eine andere Vervollständigung von $(K,\gaf)$, dann existiert ein Isomorphismus $\phi:\Tilde{K}\overset{\cong}{\rightarrow} \Tilde{K}_2$ do, dass 
* \TODO[Bild] kommutiert und 
* $\phi$ ist betragserhaltend.

Für den Beweis im Falle der $p$-adischen Betrags siehe Goueva: "p-adic numbers" S. 64-68.
Der Beweis von 4.20 verläuft genauso wie die Konstruktion von $\R$ durch Cauchyfolgen in der Analysis.
\begin{proof}[Beweisskizze]
    Zur Eindeutigkeit:
    Für $x\in \Tilde{K}$ wähle $(x_n)_n$ in $K$ mit $\lim_{n\rightarrow \infty} i(x_n) = x$ (Dichtheit).
    Setze $\phi(x)\coloneqq \lim_{n\rightarrow\infty} i_2(x_n)\in \Tilde{K}_2$.
    Grenzwert ex., da $(i(x_n))_n$ genau dann eine Cauchyfolge ist, wenn $(i_2(x))_n$ eine ist, weil $i$,$i_2$ betragserhaltend sind.

    Z.Z * $\phi$ Homomorphismus
    * $\phi$ betragserhaltend
    * die Umkehrabbildung wird ähnlich konstruiert

    Zur Existenz:
    Sei $\mathcal{R}\coloneqq \{\text{Cauchyfolgen in $K$}\}$
    Durch punktweise Addition und Multiplikation wird $\mathcal{R}$ ein Ring mit Null $(0)_n$ und Eins $(1)_n$.
    Sei $I = \{(x_n)_n\in \mathcal{R}\mid \lim_{n\rightarrow \infty} x_n = 0\}$.
    Dann ist I ein Ideal.
    Setze $\Tilde{K}\coloneqq \mathcal{R}/I$.
    $\Tilde{K}$ ist ein Körper: Sei $x+I\in \Tilde{K}\setminus \{0\}$ d.h.
    $x= (x_n)_n$ mit $(x_n)_n \notin I$.
    Insb. existiert $n_0\in\N$ mit $x_n\neq 0$ für $x_n>n_0$.
    Setze $y_n\begin{cases}
        1\quad&n\leq n_0\\
        x_n^{-1}\quad &n>n_0
    \end{cases}$.
    Dann ist $(y_n)_n$ eine Cauchyfolge.
    Also $(y_n)_n\in \R$.
    Weiter gilt $(x_n)_n\cdot(y_n)_n = (x_n\cdot y_n)_n = (1)_n + (u_n)_n$ mit $u_n =0$ für $n>n_0$, $(u_n)_n\in I$.
    => $(y_n)_n+I$ Inverses von $(x_n)_n+I$ in $\Tilde{K}$.
    Der Betrag auf $\Tilde{K}$ ist definiert durch $\|(x_n)_n+I\| \coloneqq \lim_{n\rightarrow \infty} |x_n|$ (wohldefiniert).

    Definiere $i(x) = (x)_n+I\in \Tilde{K}$.
    \begin{itemize}
        \item $i$ offensichtlich betragserhaltend
        \item $i(K)\subseteq\Tilde{K}$ dicht: Ist $(x_n)_n$ eine Cauchyfolge in $K$, dann $$\lim_{n\rightarrow\infty} i(x_n) = (x_n)_n+I$$.
        \item $\Tilde{K}$ ist vollständig:
        Ist $(y_n)_n$ eine Cauchyfolge in $\Tilde{K}$, dann wählen wir $x_n\in K$ mit $\|y_n-i(x_n)\|<\frac{1}{n}$.
        Dann gilt $$\lim_{n\rightarrow\infty} y_n = \lim_{n\rightarrow\infty} i(x_n) = (x_n)_n +I$$.
    \end{itemize}
\end{proof}

\begin{definition}
    Die Vervollständigung von $\Q$ bezüglich des p-adischen Betrags $\gaf_p$ nennt man den \emph{Körper der $p$-adischen Zahlen} $\Q_p$
\end{definition}

\begin{remark}
    Es gilt forgender Satz:\\
    Ein Körper, der vollständig bez. eines archimdischen Betrags ist, ist isomorph zu $\R$ oder $\C$. Der Betrag ist äquivalent zum gewöhnlichen Betrag.
\end{remark}

\subsection{Bewertungen und Bewertungsringe}
\begin{theorem}
    Sei $K$ ein Körper mit Bewertung $\nu:K\rightarrow\R\cup \{\infty\}$.
    Dann ist $\mathcal{O}_\nu = \{x\in K\mid \nu(x)\geq0\}$ [$\mathcal{O}_\nu = \{x\in K\mid \|x\|_\nu \leq 1\}$ für $\|x\|_\nu = a^{-\nu(x)}$] ein Ring (genannt \emph{Bewertungsring}).
    Der Ring $\mathcal{O}_\nu$ hat genau ein Maximales Ideal $m_\nu = \{x\in K\mid \nu(x)>0\}$ [$m_\nu = \{x\in K\mid \|x\|_\nu <1\}$]
    und die Einheitengruppe $\mathcal{O}_\nu^\times = \{x\in K\mid \nu(x) = 0\}$ [$\mathcal{O}_\nu^\times = \{x\in K\mid \|x\|_\nu = 1\}$]
\end{theorem}
\begin{proof}
    $\mathcal{O}_\nu$ ist ein Ring wegen der ultrametrischen Dreicksungleichung.
    Die Aussagee über $\mathcal{O}_\nu^\times$ folgt aus $\|x^{-1}\|_\nu = \|x\|_\nu^{-1}$.
    $m_\nu$ ist ein Ideal: Für $a\in \mathcal{O}_\nu$ und $x\in m_\nu$ ist $\|ax\|_\nu = \|a\|_\nu \cdot \|x\|_\nu < 1$
    Für $x,y\in m_\nu$ ist $\|x+y\|_\nu \leq \max\left\{\|x\|_\nu, \|y\|_\nu\right\} < 1$.

    Sei $I\subset \mathcal{O}_\nu.$. dann $I\cap \mathcal{O}_\nu = \emptyset$.
    => $I\subseteq m_\nu$.
    Somit ist $m_\nu$ das eindeutige maximale Ideal.
\end{proof}

\begin{definition}
    Ein kommultiver Ring $R$ mit genau einem maximalen Ideal $m$ heißt \emph{lokaler Ring}. Der Körper $R/m$ ist der \emph{Restklassenkörper} vorn $R$.
\end{definition}

\begin{example}
    Betrachet $\Q$ mit der $p$-adischen Bewertung $\nu_p$.
    $$\mathcal{O}_{\nu_p} =\{\frac{a}{b}\in \Q\mid a,b\in \Z, p\nmid b\} = \Z_{(p)}$$\footnote{Siehe A3 auf Blatt 10}
    $$m_{\nu_p} = \{\frac{a}{b}\in \Q\mid p|a, p\nmid b\} = p\cdot \mathcal{O}_{\nu_p} =p\Z_{(p)}$$% algihn
    Weiter ist $\mathcal{O}_{\nu_p}/m_{\nu_p} \overset{\cong}{\longrightarrow} \F_p$
    $$[\frac{a}{b}] \mapsto \overline{a} \cdot \overline{b}^{-1}$$ % align stuff
    
\end{example}

\begin{lemma}
    Sei $R$ ein lokaler Ring mit maximalem Ideal $m$. Dann gilt $R^\times = R/m$.
\end{lemma}
\begin{proof}
    * $x\in R^\times$ =>$xR = R$ => $x\notin m$.
    * $x\in R\setminus m$. Dann ist das Ideal $xR$ nicht in $m$ enhalten.
    Da $m$ das einzige maximale Ideal und jedes Ideal $\neq R$ in einem maximalen Ideal enthalten ist, ist $xR = R$ und somit $x\in R^\times$.
\end{proof}

\end{document}