\documentclass[../main.tex]{subfiles}
\graphicspath{{\subfix{../images/}}}
\begin{document}
\begin{flushright}
VL vom 11.1.2024:
\end{flushright}
\TODO[right-flush und rework]
\subsection{Beträge}
\begin{definition}
    Ein \emph{Betrag} auf einem Körper $K$ ist eine $\card{.}: K\rightarrow \R_{\geq0}$ mit den Eigenschaften
    \begin{enumerate}
        \item $\card{x} = 0 \Leftrightarrow x =0$
        \item $\card{x\cdot y} = |x| \cdot |y|$
        \item $|x+y| \leq |x| + |y|$
    \end{enumerate}
    Aus ii) folgt $|1| = 1$.
\end{definition}
\begin{example}
    \begin{enumerate}[label=\roman*)] %i ii
        \item $\R$ mit dem gewöhnlichen Absolutbetrag
        \item $K$ beliebig. Dann definiere 
        $$|x|_{trivial} = \begin{cases}
            1 & falls x\neq 0\\
            0 & falls x= 0
        \end{cases}$$
        ein Betrag (genannt: \emph{trivialer Betrag})
        \item $\C$ mit dem komplexen Betrag
        \item Ist $|.|$ ein Betrag auf $L$ und $K$ ein Teilkörper von $L$, so ist $|.|_{|K}$ ein Betrag auf $K$.
    \end{enumerate}
\end{example}
\begin{definition} [p-adischer Betrag]
    Sei $a\in \Z$ und $p$ prim.
    $\nu_p(a)= \sup\{e\in \N\mid p^e|a\}\in \N_0\cup \{\infty\}$
\end{definition}
Beachte $\nu_p(0) = \infty$
$a = \pm \prod_{p prim} p^{\nu_p(a)}$
Die Funktion $\nu_p:\Z\rightarrow \N_0\cup \{\infty\}$ heißt \emph{p-adische Bewertung}. Sie hat die folgenden Eigenschaften
\begin{enumerate}[label=\roman*)]%i, ii
    \item $\nu_p(a) = \infty \Leftrightarrow a = 0$
    \item $\nu_p(ab) = \nu_p(a)+\nu_p(b)$
    \item $\nu_p(a+b) \geq \min{\nu_p(a),\nu_p(b)}$
\end{enumerate}

Sei nun $x\in \Q^\times$. Schreibe
$$x=p^m\cdot\frac{a}{b}$$ mit $m\in \Z$, $p\nmid a$, $p\nmid b$.
Definiere $\nu_p(x) = m$ und $\nu_p(0) = \infty$. Dies setzt die $p$-adische Berwertung von $\Z$ auf $\Q$ fort mit den gleichen Eigenschaften.
Setze $|x|_p \coloneqq p^{-\nu_p(x)}$ mit der Konventiion $|0|_p = p^{-\infty} = 0$
Dann definiert $|.|_p$ einen Betrag auf $\Q$, den \emph{p-adischen Betrag}\footnote{Zum 2ten mal}. Die Eigenschaften i), ii) sind offensichtlich.
Zu iii) $|x+y|_p = p^{-\nu_p(x+y}\leq p^{\min{\nu_p(x),\nu_p(y)}} = \max{p^{-\nu_p(x)}, p^{-\nu_p(y)}} = \max{|x|_p,|y_p|}$ %Multiline???
Dies wird \emph{ultrametische Dreiecksungleichung} genannt.
\begin{definition}
    Eine Bewertung $\nu$ auf einem Körper $K$ ist eine Funktion $$\nu:K\rightarrow \R\cup \{\infty\}$$
    mit folgenden Eigenschaften
    \begin{enumerate}[label=\roman*)]
        \item $\nu(x) = \infty \Leftrightarrow x= 0$
        \item $\nu(x\times y) = \nu(x)+\nu(y)$
        \item $\nu(x+y) \geq \min{\mu(x), \nu(y)}$ 
    \end{enumerate}
\end{definition}
\begin{lemma}
    Seien $K$ ein Körper, $\nu$ eine Bewertung auf $K$ und $a\in \R_{>1}$. Dann definiert $|x| = a^{-\nu(x)}$ einen Betrag auf dei die ultimetrische Dreiecksungleichung $|x+y| \leq \max{|x|,|y|}$ erfüllt.
    Beweis: analog zu p-adischen Betrag
\end{lemma}
\begin{example}
    Sei $K$ ein Körper. Sei $K(T) = Quot(K[T])$.
    Sei $K(T)\setminus\{0\}\ni x = \frac{f}{q}$, $f,g\in K[T]$.
    Setze $\nu_{deg}(x) = -\deg(f)+\deg(g)$ und $\nu_{deg}(0) = \infty$.
    Dies definiert eine Bewertung auf $K(T)$
\end{example}

\begin{definition}[Metrik und Topologie]
    Sei $K$ ein Körper mit Betrag $|.|$.
    Dann definiert $d(x,y)= |x-y|$ eine Metrik auf $K$.
    Insbesondere erhalten wir eine Topologie auf $K$ vermöge 
    $$U offen :\Leftrightarrow\forall {u\in U} \exists {\epsilon >0}: B_{\epsilon}^{|.|} = \{v\in K\mid |u-v|<\epsilon\}\subseteq U$$
\end{definition}
\begin{remark}
    Die Addition und Multiplikation definieren stetige Abbildungen $K\times K \rightarrow K$.
\end{remark}
\begin{definition}
    Zwei Bewertungen $|.|_1$, $|.|_2$ sind \emph{äuivalent}, wenn sie diesselbe Topologie auf $K$ induzieren.
\end{definition}
\begin{lemma}
    Für zwei Beträge $|.|_1$, $|.|_2$ auf $K$ sind folgende Aussagen äquivalent (Schreibe $|.|_1 \sim|.|_2$):
    \begin{enumerate}[label=\alph*)]
        \item $|.|_1 \sim|.|_2$
        \item $\forall {x\in K}: |x|_1 <1 \Leftrightarrow |x|_2<1$
        \item $\exists {s\in \R_{>0}}: |x|_1 = |x|_2^s$ für alle $x\in K$
    \end{enumerate}
\end{lemma}
\begin{proof}
    a) => b):% Reformat that shit!! underset for all arrows!
    Sei $x\in K$. Dann gilt $\card{x}_1<1\Leftrightarrow \card{x}_1^n\underset{n\rightarrow \infty}{\rightarrow} 0 \Leftrightarrow \card{x^n}_1 -> 0 \Leftrightarrow |x^n|_2 -> 0\Leftrightarrow ->0 ...\Leftrightarrow |x|_2 <0$
    b) => c):
    Falls $|.|_1$ nich trivial ist, dann gilt es ein $y\in K^\times$ mit $|y|_1 <1$.
    Somit ist $|.|_1$ triviial gdw $|.|_2$ trivial ist.

    Wir nehmen nun an, dass $|.|_1$ nicht trivial ist.
    Sei $x\in K^\times$. Schreibe $|x|_1 = |ny|_1^\alpha$ für eine $\alpha\in \R$.
    Beh: $|x|_2 = |y|_2^\alpha$.
    Sei $(\frac{m_i}{n_i})_i$ eine Folge rationaler Zahlen, die von oben gegen $\alpha$ konvergiert.
    $$|x|_1< |y|_2^{\frac{m_i}{n_i}} \Rightarrow |x^{n_i}| < |y^{m_i}| \Rightarrow |x^{n_i}y^{-m_i}| <1$$
     Damit folgt mit b), dass $|x^{n_i}y^{-m_i}|_2 <1$ und somit (ählich wie für $|.|_1$) $|x|_2< |y|_2^{\frac{m_i}{n_i}}$.
     Limes $i\rightarrow\infty$ ergibt, dass $|x|_2\leq |y|_2^\alpha$.
      Das selbe Argument mit $\frac{m_i}{n_i} pfeil nach oben \alpha$ liefert $|x|_2 \geq |y|_2^\alpha$.
      Sei $s\in \R_{>0}$ mit $|y|_2 = |y|_1^s$.
      Dann folgt:
      $$|x|_2 = |y|_2^\alpha = |y|_1^{s\cdot \alpha} = |x|_1^s$$.
      c) => a)
      Sei $\epsilon >0$ und $u\in K$.
      $$B_\epsilon^{|.|_1}(u) = \{u\in K\mid |x-u|_1<\epsilon\} = \{x\in K\mid |x-u|<\epsilon^{-s}\} = B_{\epsilon^{-s}}^{|.|_2}(u)$$
      => $||_1\sim||_2$
\end{proof}

\begin{corollary}
    Seien $||,||$ zwei nicht triviale, nicht äquivalente Beträge auf $K$. Dann gibt es ein $u\in K$ mit $|u|_1 < 1$0und $|u|_2 > 1$.
\end{corollary}
\begin{proof}
    nach 4.10 b) finden wir $y\in K$ mit $|y|_1 <1$ und $|y|_2\geq 1$.
     Falls $|y|_2 = 1$, wähle $w\in K$ mit $|w|_2>1$.
     Setze $u=y^m\cdot w$ für hinreichend großes $m$ eine geeignete Wahl.
\end{proof}

\begin{theorem}[Approximationssatz]
    Seien $|.|_1, \dots,|.|_n$ nicht trivial, paarweise inäquivalente Beträge auf $K$.
    Für alle $a_1,\dots, a_n\in K$ und $\epsilon > 0$ existiert ein $x\in K$ mit $$|a_i-x|_i <\epsilon \text{ für alle } i\in \{1,\dots,n\}$$
\end{theorem}

\begin{proof}
    1. Schritt: $\exists z\in K: |z|_j > 1$ für alle $j\in \{2,\dots,n\}$.
    Durch Induktion: * $n=2$ ist 4.11 \checkmark
    * Induktionannahme: $\Tilde{z}\in K$, $|\Tilde{z}| < 1$ und $|\Tilde{z}|_j > 1$ für $j\in \{2,\dots, n-1\}$
    Falls $|\Tilde{z}|_n > 1$, setze $z\coloneqq \Tilde{z}$
    Falls $|\Tilde{z}|_n = 1$: Finde ein $u\in K$ mit $|u|_1 >1$ und $|u|_n <1$ (siehe 4.11)
    Dann ist $z = u\cdot \Tilde{z}^m$ für $m>>1$ eine geeignete Wahl.
    Falls $|\Tilde{z}|_n >1$: bann betracheten wir die Folge $t_m = \frac{\Tilde{z}^m}{\Tilde{z}^m +1}$. Es Gilt $t_m \underset{\longrightarrow}{m\rightarrow\infty} 0$ bez. $|.|_1$ und $|.|_n$.
    $t_m \underset{\longrightarrow}{m\rightarrow\infty} 1$ bez $|.|_j$ für $j\notin \{1,n\}$.

    Wähle $n\in K$ mit $|u|_1<1$ und $|u|_n>1$ und definiere $z=u\cdot t_m$ für $m>>1$


\TODO[Merge different versions of the first part]


    
    Behauptung: $\exists z\in K: |z|_1 > 1$ und $|z|_j<1$ für $j\in \{2,...,n\}$
    Durch Induktion: $n=2$ \checkmark (nach 4.11)

    Angenommen es gibt $\Tilde{z}\in K$ mit $|\Tilde{z}|_1 > 1$ und $|\Tilde{z}|_j <1$ für $j\in \{2,...,n-1\}$.
    Wähle $u\in K$ mit $|u|_1>1$ und $|u|_n<1$ nach 4.11.
    Falls $|\Tilde{z}|_n<1$, dann ist $z=\Tilde{z}$ das gesuchte Element.
    Falls $|\Tilde{z}|_n = 1$, dann ist $z=u\Tilde{z}^m$ für hinreichend großes $m$ das gesuchte Element.
    <Argumentation, dass z gewünschte Eigenschaften hat...>
    Falls $|\Tilde{z}|_n > 1$, dann betrachtet man die Folge $t_m\coloneqq \frac{\Tilde{z}^m}{1+\Tilde{z}^m}$., für die die 
    $t_m\rightarrow 1$ bez. $|.|_1$ und $|.|_n$
    $t_m\rightarrow 0$ bez. $|.|_2,\dots,|.|_{n-1}$ gelten.
    Dann ist $z=u\cdot t_m$ für hinreihend großes $m$ das gesuchte Element.
    <Argumentation, dass z gewünschte Eigenschaften hat...>

    Für ein $z\in K$ wie in der obigen beh. konvergiert die Folge $(\frac{z^m}{1+z^m})_m$ gegen $1$ bez $|.|_1$ und gegen $0$ bez $|.|_2,\dots,|.|_n$.

    Für jedes $i\in \{1,\dots,n\}$ können wir somit ein $z_i\in K$ konstrieren, das sehr nahe bei $1$ bez $|.|_i$ und sehr nahe $0$ bez der restlichen Beträge.
    Das Element $x=a_1\cdot z_1 + \dots +a_n\cdot z_n$ erfüllt die Anforderungen des Approximationssatzes. 
\end{proof}
\begin{flushright}
VL vom 11.1.2024:
\end{flushright}

\begin{definition}
    Ein Betrag $|.|$ auf $K$ heißt \emph{archimedisch}, wenn 
    $$\{|n\cdot1_K| \mid n\in \N\}$$
    nicht beschränkt ist.
\end{definition}

\begin{theorem}
    Ein Betrag ist genau dann nicht-archimedisch, wenn er die ultrametrische Dreiecksungleichung erfüllt.
\end{theorem}
\begin{proof}
    "<=":
    $|n\cdot 1_K|= |1_K+\dots+1_K| \leq |1_K| = 1$ also ist $|.|$ nicht -archimedisch

    "=>":
    Seien $x,y\in K$ mit $|x|\geq|y|$.Sei $B>0$ eine obere Schranke für $\{|n\cdot1_K| \mid n\in \N\}$.
    $|x+y|^n = |(x+y)^n| = |\sum_{k=0}^n \binom{n}{k} x^k y^{n-k}| \leq \sum_{k=0}^n |\binom{n}{k}\cdot 1_K| \cdot |x|^n\leq (n+1)\cdot B\cdot |x|^n$
    => $|x+y| \leq ((n+1)B)^{1/n} \cdot |x| \overset{n\rightarrow \infty}{\Rightarrow} |x+y| \leq |x|$
\end{proof}
Der Trick am Ende für "schlechte"/ungenaue ungleichungen:
Betrachte Potenzen und ziehe von dem dort gezeigen die Wurzel. Das kann bessere Ungleichungen bringen.
\begin{remark}$ $
    \begin{enumerate}[label=\alph*)]
        \item Ist $|.|$ ein nicht-archimedischr Betrag, denn ist $\nu(x) = -\log(|x|)$ eine Bewertung auf $K$
        \item Ist $|.|$ nicht-archimedisch, $|x|> |y|$, dann folgt 
        $$|x|\leq |x-y+y|\leq \max\{|x-y|,|y|\} = |x-y| \leq |x|$$
        und somit Gleichheit.
    \end{enumerate}
\end{remark}

\begin{theorem}[Satz von Ostrowski]
    Jeder nicht-triviale Betrag auf $\Q$ ist äquivalent zu $|.|_p$ für eine Primzahl $p$ oder zum Absolutbetrag $|.|_\infty$.
\end{theorem}
\begin{proof}
    Sei $||.||$ ein nicht-trivialer Betrag auf $\Q$.

    1. Fall: Sei $||.||$ nicht-archimedisch, d.h. $||n||\leq 1$ für alle $n\in \N$.
    Sei $\mathcal{A}\coloneqq \{n\in \Z\mid ||n|| <1\}$.
    Dann ist $\mathcal{A}$ ein Ideal in $\Z$.
    Es gibt eine Primzahl $p$ mit $p\in \mathcal{A}$, ansonsten würe $||.||$ trivial wegen Primfaktorenzerlegung. Also
    $$p\Z\subseteq\mathcal{A}\subset \Z$$ (da $1\notin \mathcal{A}$).
    Da $p\Z$ maximales Ideal ist, ist $p\Z = \mathcal{A}$.
    Sei $q\in \Q^\times$ und schreibe $q=p^m\cdot \frac{a}{b}$ mit $p\nmid a$, $p\nmid b$.
    Dann gilt $||q|| = ||p||^m$ wegen $||a|| = ||b|| =1$.
    Sei $s>0$ so, dass $p^{-s} = ||p||$.
    Dann folgt $||p|| =p^{-s} = (p^{-1})^s= |p|_p^s$ => $||q| = ||p^m|| = |q|_p^s$.
    => $||.|| \sim |.|_p$
    

    2. Fall: Sei $||.||$ archimedisch.
    Beh.: Für alle natürlichen $n,m>1$ gilt
    $$||n||^{\frac{1}{\log(n)}} = ||m||^{\frac{1}{\log(m)}}\quad (*)$$
    Somit $c\coloneqq||n||^{\frac{1}{\log(n)}}$ unabhängig von $n>1$.
    => $||m|| = c^{\log(m)}$ für jedes $m\in \N\setminus\{1\}$.
    Wenn wir $e = e^s$, $s>0$, schreibe, dann ergibt sich für jedes positive rationale Zahl $x=\frac{a}{b}$
    $$||x|| = e^{s\log(x)} = |x|_\infty^s$$
    Damit folgt $||.||\sim |.|_\infty$.

    Nun zu $(*)$:
    Schreibe $m$ zur Basis $n$ (OBdA $n<m$):
    $$m= a_0 + a_1 n+\dots+a_rn^r$$
    mit $a_i\in \{0,\dots,n-1\}$, $a_r = 0$.
    Somit $n^r\leq m$, also $r\leq \frac{\log(m)}{\log(n)}$.
    Weiter $||a_i|| \leq a_i ||1|| = a_i<n$.
    Es folgt $$||m||\leq \sum_{i=0}^r ||a_i|| \cdot ||n||^i\leq \sum_{i=0}^r n \cdot ||n||^r \leq \left(1+\frac{\log(m)}{\log(n)}\right) \cdot n \cdot ||n||^{\frac{\log(m)}{\log(n)}}$$
    Wir ersetzen in dieser Ungleichung $m$ durch $m^k$ und ziehen die $k$-te Wurzel:
    $$||m|| \leq \sqrt[k]{\left(1+\frac{k\cdot\log(m)}{\log(n)}\right)}\cdot n^{1^k} \cdot ||n||^{\frac{\log(m)}{\log(n)}}$$
    => $||m||^{\frac{1}{\log(m)}} \leq ||n||^{\frac{1}{\log(n)}}$
    Da die Rollen von $m$ und $n$ vertauschbar sind, folgt Gleichheit.
\end{proof}
\subsection{Vervollständigungen}
\begin{definition}
    Ein Körper mit Betrag ist \emph{vollständig}, wenn jede Cauchyfolge in $K$ konvergiert.
\end{definition}
\begin{example}
    \begin{itemize}
        \item $(R,|.|_\infty)$, $(\C, |.|_\C)$ sind vollständig
        \item $(\Q, |.|_5)$ nicht vollständig\footnote{Das Argument gilt für alle Prmzahlen $p\equiv 1\mod 4$}
        Sei $a_k + 5^k\Z$ ein Element in $(\Z/5^k\Z)^\times$ dei Ordnung $4$.\footnote{Wegen $\phi(5^k) = 5^k-5^{k-1}  = 5^{k-1} \cdot 4$ ($k\geq 2$) und der Tatsache, dass $(\Z/5^k\Z)^\times$ yzklisch ist gilt, dass $(\Z/5^k\Z)^\times \cong \Z/4\Z \times \Z/5^{k-1}\Z$. Bei Hartnick gab es mehr dazu.}
        Wir können erreichen, dass $a_k\equiv a_{k+1}\mod 5^k$
        => $|a_k-a_{k+1}|_5 < \frac{1}{5^k}$, also ist $(a_k)_k$ eine Cauchyfolge bez $|.|_5$.

        $a_k + 5^5\Z$ hat Ordnung $2$, also $a_k^2 \equiv -1 \mod 5^k$.
        => $\lim_{k\rightarrow \infty} a_k^2 = -1$ bez $|.|_5$.
        Also müsste ein Grenzwert $a=\lim_{k\rightarrow \infty} a_k$ in $\Q$ die Gleichung $a^2= -1$ erfüllen \Lightning
    \end{itemize}
\end{example}
\end{document}