\documentclass[../main.tex]{subfiles}
\graphicspath{{\subfix{../images/}}}
\begin{document}
\begin{flushright}
VL vom 11.1.2024:
\end{flushright}
\subsection{Beträge}
\begin{definition} % \gaf Generic absolut function
    Ein \emph{Betrag} auf einem Körper $K$ ist eine $\gaf: K\rightarrow \R_{\geq0}$ mit den Eigenschaften
    \begin{enumerate}[label=\roman*)]
        \item $\card{x} = 0 \Leftrightarrow x =0$
        \item $\card{x\cdot y} = |x| \cdot |y|$
        \item $|x+y| \leq |x| + |y|$
    \end{enumerate}
    Aus ii) folgt $|1|\cdot |1| =|1| = 1$.
\end{definition}
\begin{example}$ $
    \begin{enumerate}[label=\roman*)] %i ii
        \item $\R$ mit dem gewöhnlichen Absolutbetrag
        \item $K$ beliebig. Dann definiert 
        $$|x|_{trivial} = \begin{cases}
            1 & falls x\neq 0\\
            0 & falls x= 0
        \end{cases}$$
        ein Betrag (genannt: \emph{trivialer Betrag})
        \item $\C$ mit dem komplexen Betrag
        \item Ist $\gaf$ ein Betrag auf $L$ und $K$ ein Teilkörper von $L$, so ist $\gaf_{|K}$ ein Betrag auf $K$.
    \end{enumerate}
\end{example}
\begin{definition} [p-adischer Betrag]
    Sei $a\in \Z$ und $p$ prim. Dann ist
    $\nu_p(a)= \sup\{e\in \N\mid p^e|a\}\in \N_0\cup \{\infty\}$
    die \emph{p-adische Betrag}.
Beachte $\nu_p(0) = \infty$.
\end{definition}
Damit gilt:
$$a = \pm \prod_{p\text{ prim}} p^{\nu_p(a)}$$
Die Funktion $\nu_p:\Z\rightarrow \N_0\cup \{\infty\}$ heißt \emph{p-adische Bewertung}. Sie hat die folgenden Eigenschaften
\begin{enumerate}[label=\roman*)]%i, ii
    \item $\nu_p(a) = \infty \Leftrightarrow a = 0$
    \item $\nu_p(ab) = \nu_p(a)+\nu_p(b)$
    \item $\nu_p(a+b) \geq \min\left\{\nu_p(a),\nu_p(b)\right\}$
\end{enumerate}

Sei nun $x\in \Q^\times$. Schreibe
$$x=p^m\cdot\frac{a}{b}$$ mit $m\in \Z$, $p\nmid a$, $p\nmid b$.
Definiere $\nu_p(x) = m$ und $\nu_p(0) = \infty$. Dies setzt die $p$-adische Berwertung von $\Z$ auf $\Q$ fort mit den gleichen Eigenschaften.
Setze $|x|_p \coloneqq p^{-\nu_p(x)}$ mit der Konvention $|0|_p = p^{-\infty} = 0$.
Dann definiert $\gaf_p$ einen Betrag auf $\Q$, den \emph{p-adischen Betrag}\footnote{Zum 2ten mal}. Die Eigenschaften i), ii) sind offensichtlich.
Zu iii):
\begin{align*}
    |x+y|_p &= p^{-\nu_p(x+y)}\\
    &\leq p^{-\min\{\nu_p(x),\nu_p(y)\}} \\
    &= \max\{p^{-\nu_p(x)}, p^{-\nu_p(y)}\} \\&
    = \max\{|x|_p,|y_p|\}
\end{align*}
%Multiline???
Dies wird \emph{ultrametrische Dreiecksungleichung} genannt.
\begin{definition}
    Eine Bewertung $\nu$ auf einem Körper $K$ ist eine Funktion $$\nu:K\rightarrow \R\cup \{\infty\}$$
    mit folgenden Eigenschaften
    \begin{enumerate}[label=\roman*)]
        \item $\nu(x) = \infty \Leftrightarrow x= 0$
        \item $\nu(x\times y) = \nu(x)+\nu(y)$
        \item $\nu(x+y) \geq \min\{\mu(x), \nu(y)\}$ 
    \end{enumerate}
\end{definition}
\begin{lemma}
    Seien $K$ ein Körper, $\nu$ eine Bewertung auf $K$ und $a\in \R_{>1}$. Dann definiert $|x| = a^{-\nu(x)}$ einen Betrag auf dem die ultimetrische Dreiecksungleichung $|x+y| \leq \max\{|x|,|y|\}$ erfüllt.
\end{lemma}
\begin{proof}
    analog zu p-adischen Betrag
\end{proof}
\begin{example}
    Sei $K$ ein Körper. Sei $K(T) = Quot(K[T])$.
    Sei $K(T)\setminus\{0\}\ni x = \frac{f}{q}$, $f,g\in K[T]$.
    Setze $\nu_{deg}(x) = -\deg(f)+\deg(g)$ und $\nu_{deg}(0) = \infty$.
    Dies definiert eine Bewertung auf $K(T)$
\end{example}

\begin{definition}[Metrik und Topologie]
    Sei $K$ ein Körper mit Betrag $\gaf$.
    Dann definiert $d(x,y)= |x-y|$ eine Metrik auf $K$.
    Insbesondere erhalten wir eine Topologie auf $K$ vermöge 
    $$U\text{ offen }:\Leftrightarrow\forall {u\in U} \exists {\epsilon >0}: B_{\epsilon}^{\gaf} = \{v\in K\mid |u-v|<\epsilon\}\subseteq U$$
\end{definition}
\begin{remark}
    Die Addition und Multiplikation definieren stetige Abbildungen $K\times K \rightarrow K$.
\end{remark}
\begin{definition}
    Zwei Bewertungen $\gaf_1$, $\gaf_2$ sind \emph{äquivalent}, wenn sie diesselbe Topologie auf $K$ induzieren.
\end{definition}
\begin{lemma}
    Für zwei Beträge $\gaf_1$, $\gaf_2$ auf $K$ sind folgende Aussagen äquivalent (schreibe $\gaf_1 \sim\gaf_2$):
    \begin{enumerate}[label=\alph*)]
        \item $\gaf_1 \sim\gaf_2$
        \item $\forall {x\in K}: |x|_1 <1 \Leftrightarrow |x|_2<1$
        \item $\exists {s\in \R_{>0}}: |x|_1 = |x|_2^s$ für alle $x\in K$
    \end{enumerate}
\end{lemma}
\begin{proof}$ $
    \begin{itemize}
        \item[a) $\Rightarrow$ b)]
        Sei $x\in K$. Dann gilt
        \begin{align*}
            \card{x}_1<1&\Leftrightarrow \card{x}_1^n\underset{n\rightarrow \infty}{\longrightarrow} 0\\
            &\Leftrightarrow \card{x^n}_1 \underset{n\rightarrow \infty}{\longrightarrow} 0\\
            &\Leftrightarrow \card{x^n}_2 \underset{n\rightarrow \infty}{\longrightarrow} 0\\
            &\Leftrightarrow \card{x}_2^n \underset{n\rightarrow \infty}{\longrightarrow} 0\\
            &\Leftrightarrow |x|_2 <1
        \end{align*}
        \TODO[Wieso folgt das in der Mitte aus äquivalenz?]
        \item[b) $\Rightarrow$ c)]
        Falls $\gaf_1$ nicht trivial ist, dann gilt es ein $y\in K^\times$ mit $|y|_1 <1$.
        Somit ist $\gaf_1$ trivial gdw $\gaf_2$ trivial ist.
    
        Wir nehmen nun an, dass $\gaf_1$ nicht trivial ist.
        Sei $x\in K^\times$. Schreibe $|x|_1 = |y|_1^\alpha$ für eine $\alpha\in \R$.
        Beh: $|x|_2 = |y|_2^\alpha$.
        Sei $(\frac{m_i}{n_i})_i$ eine Folge rationaler Zahlen, die von oben gegen $\alpha$ konvergiert.
        $$|x|_1\leq |y|_1^{\frac{m_i}{n_i}} \Rightarrow |x^{n_i}|_1 < |y^{m_i}|_1 \Leftrightarrow |x^{n_i}y^{-m_i}|_1 <1$$
        Damit folgt mit b), dass $|x^{n_i}y^{-m_i}|_2 <1$ und somit (ähnlich wie für $\gaf_1$) $|x|_2\leq |y|_2^{\frac{m_i}{n_i}}$.\footnote{\TODO[Wieso gilt die Rückrichtung?]}
        Limes $i\rightarrow\infty$ ergibt, dass $|x|_2\leq |y|_2^\alpha$.
        Das selbe Argument mit von unten gegen $\alpha$ konvergierende $\frac{m_i}{n_i}$ liefert $|x|_2 \geq |y|_2^\alpha$.
        Sei $s\in \R_{>0}$ mit $|y|_2 = |y|_1^s$.
        Dann folgt:
        $$|x|_2 = |y|_2^\alpha = |y|_1^{s\cdot \alpha} = |x|_1^s$$.
        \item[c) $\Rightarrow$ a)]
        Sei $\epsilon >0$ und $u\in K$.
        $$B_\epsilon^{\gaf_1}(u) = \{u\in K\mid |x-u|_1<\epsilon\} = \{x\in K\mid |x-u|<\epsilon^{-s}\} = B_{\epsilon^{-s}}^{\gaf_2}(u)$$
        $\Longrightarrow$ $\gaf_1\sim\gaf_2$
    \end{itemize}
\end{proof}

\begin{corollary}\label{theo:4.11}
    Seien $\gaf_1,\gaf_2$ zwei nicht triviale, nicht äquivalente Beträge auf $K$. Dann gibt es ein $u\in K$ mit $|u|_1 < 1$ und $|u|_2 > 1$.
\end{corollary}
\begin{proof}
    nach 4.10 b) finden wir $y\in K$ mit $|y|_1 <1$ und $|y|_2\geq 1$.\footnote{Sollten die Ungleichheiten vertauscht sind, kann das Inverse betrachtet werden.}
     Falls $|y|_2 = 1$, wähle $w\in K$ mit $|w|_2>1$.
     Setze $u=y^m\cdot w$ für hinreichend großes $m$ eine geeignete Wahl.
\end{proof}

\begin{theorem}[Approximationssatz]
    Seien $\gaf_1, \dots,\gaf_n$ nicht trivial, paarweise inäquivalente Beträge auf $K$.
    Für alle $a_1,\dots, a_n\in K$ und $\epsilon > 0$ existiert ein $x\in K$ mit $$|a_i-x|_i <\epsilon \text{ für alle } i\in \{1,\dots,n\}$$
\end{theorem}

\begin{flushright}
VL vom 11.1.2024:
\end{flushright}

\begin{proof}
    Behauptung: $\exists z\in K: |z|_1 > 1$ und $|z|_j<1$ für $j\in \{2,...,n\}$
    Durch Induktion: Der Induktionsanfang $n=2$ ist nach \ref{theo:4.11}\checkmark.

    Angenommen es gibt $\Tilde{z}\in K$ mit $|\Tilde{z}|_1 > 1$ und $|\Tilde{z}|_j <1$ für $j\in \{2,...,n-1\}$.
    Wähle $u\in K$ mit $|u|_1>1$ und $|u|_n<1$ nach \ref{theo:4.11}.
    Falls $|\Tilde{z}|_n<1$, dann ist $z=\Tilde{z}$ das gesuchte Element.
    Falls $|\Tilde{z}|_n = 1$, dann ist $z=u\Tilde{z}^m$ für hinreichend großes $m$ das gesuchte Element.
    <Argumentation, dass z gewünschte Eigenschaften hat...>
    Falls $|\Tilde{z}|_n > 1$, dann betrachtet man die Folge $t_m\coloneqq \frac{\Tilde{z}^m}{1+\Tilde{z}^m}$., für die die
    \begin{align*}
        t_m\rightarrow 1\quad &\text{ bez. }\gaf_1\text{ und }\gaf_n\\
        t_m\rightarrow 0\quad&\text{ bez. }\gaf_2,\dots,\gaf_{n-1}
    \end{align*}
    gelten.
    Dann ist $z=u\cdot t_m$ für hinreihend großes $m$ das gesuchte Element.
    <Argumentation, dass z gewünschte Eigenschaften hat...>

    Für ein $z\in K$ wie in der obigen beh. konvergiert die Folge $(\frac{z^m}{1+z^m})_m$ gegen $1$ bez $\gaf_1$ und gegen $0$ bez $\gaf_2,\dots,\gaf_n$.

    Für jedes $i\in \{1,\dots,n\}$ können wir somit ein $z_i\in K$ konstrieren, das sehr nahe bei $1$ bez $\gaf_i$ und sehr nahe $0$ bez der restlichen Beträge.
    Das Element $x=a_1\cdot z_1 + \dots +a_n\cdot z_n$ erfüllt die Anforderungen des Approximationssatzes. 
\end{proof}

\begin{definition}
    Ein Betrag $\gaf$ auf $K$ heißt \emph{archimedisch}, wenn 
    $$\{|n\cdot1_K| \mid n\in \N\}$$
    nicht beschränkt ist.\footnote{Nach Übung: Wenn ein Betrag $\gaf$ nicht-archimedisch, gilt $\sup\{|n\cdot1_K| \mid n\in \N\} = 1$, wegen $|1| = 1$ und der ultrametrischen Dreiecksungleichung nach \ref{theo:4.14}}
\end{definition}

\begin{theorem}\label{theo:4.14}
    Ein Betrag ist genau dann nicht-archimedisch, wenn er die ultrametrische Dreiecksungleichung erfüllt.
\end{theorem}
\begin{proof}
    \begin{itemize}
        \item[$\Leftarrow$]
        $|n\cdot 1_K|= |1_K+\dots+1_K| \leq |1_K| = 1$ also ist $\gaf$ nicht -archimedisch
        \item[$\Rightarrow$]
        Seien $x,y\in K$ mit \obda $|x|\geq|y|$.
        Sei $B>0$ eine obere Schranke für $\{|n\cdot1_K| \mid n\in \N\}$.
        \begin{align*}
            |x+y|^n &= |(x+y)^n| = |\sum_{k=0}^n \binom{n}{k} x^k y^{n-k}|\\
            &\leq \sum_{k=0}^n |\binom{n}{k}\cdot 1_K| \cdot |x|^k \cdot |y|^{n-k} \leq \sum_{k=0}^n |\binom{n}{k}\cdot 1_K| \cdot |x|^n\\
            &\leq (n+1)\cdot B\cdot |x|^n
        \end{align*}
        $\Longrightarrow\;|x+y| \leq \sqrt[n]{(n+1)B} \cdot |x| \overset{n\rightarrow \infty}{\Rightarrow} |x+y| \leq |x|$
    \end{itemize}
\end{proof}
Der Trick am Ende für 'schlechte'/ungenaue Ungleichungen:
Betrachte Potenzen und ziehe von dem dort gezeigen die Wurzel. Das kann bessere Ungleichungen bringen.
\begin{remark}$ $
    \begin{enumerate}[label=\alph*)]
        \item Ist $\gaf$ ein nicht-archimedischer Betrag, dann ist $\nu(x) = -\log(|x|)$ eine Bewertung auf Körper $K$
        \item Ist $\gaf$ nicht-archimedisch, $|x|> |y|$, dann folgt 
        $$|x|\leq |x-y+y|\leq \max\{|x-y|,|y|\} = |x-y| \leq |x|$$
        und somit Gleichheit.
    \end{enumerate}
\end{remark}

\begin{theorem}[Satz von Ostrowski]
    Jeder nicht-triviale Betrag auf $\Q$ ist äquivalent zu $\gaf_p$ für eine Primzahl $p$ oder zum Absolutbetrag $\gaf_\infty$.
\end{theorem}
\begin{proof}
    Sei $\|.\|$ ein nicht-trivialer Betrag auf $\Q$.
    \begin{itemize}
        \item[1.Fall]
        Sei $\|.\|$ nicht-archimedisch, d.h. $\|n\|\leq 1$ für alle $n\in \N$.
        Sei $\mathcal{A}\coloneqq \{n\in \Z\mid \|n\| <1\}$.
        Dann ist $\mathcal{A}$ ein Ideal in $\Z$.
        Es gibt eine Primzahl $p$ mit $p\in \mathcal{A}$, ansonsten würe $\|.\|$ trivial wegen Primfaktorenzerlegung. Also
        $$p\Z\subseteq\mathcal{A}\subset \Z$$ (da $1\notin \mathcal{A}$).
        Da $p\Z$ maximales Ideal ist, ist $p\Z = \mathcal{A}$.
        Sei $q\in \Q^\times$ und schreibe $q=p^m\cdot \frac{a}{b}$ mit $p\nmid a$, $p\nmid b$.
        Dann gilt $\|q\| = \|p\|^m$ wegen $\|a\| = \|b\| =1$.
        Sei $s>0$ so, dass $p^{-s} = \|p\|$.
        Dann folgt $\|p\| =p^{-s} = (p^{-1})^s= |p|_p^s$\\
        $\Longrightarrow\|q\| = \|p^m\| = |q|_p^s$.\\
        $\Longrightarrow\;\|.\| \sim |.|_p$

        \item[2.Fall]
        Sei $\|.\|$ archimedisch.
        Beh.: Für alle natürlichen $n,m\in\N_{>1}$ gilt
        $$\|n\|^{\frac{1}{\log(n)}} = \|m\|^{\frac{1}{\log(m)}}\quad (*)$$
        Somit $c\coloneqq\|n\|^{\frac{1}{\log(n)}}$ unabhängig von $n>1$.
        => $\|m\| = c^{\log(m)}$ für jedes $m\in \N\setminus\{1\}$.
        Wenn wir $c = e^s$, $s>0$, schreibe, dann ergibt sich für jedes positive rationale Zahl $x=\frac{a}{b}$
        $$\|x\| = e^{s\log(x)} = |x|_\infty^s$$
    \end{itemize}
    Damit folgt $\|.\|\sim \gaf_\infty$.

    Zu $(*)$:\\
    Schreibe $m\in \N$ zur Basis $n\in\N$ (\obda\ $n<m$):
    $$m= a_0 + a_1 n+\dots+a_rn^r$$
    mit $a_i\in \{0,\dots,n-1\}$, $a_r \neq 0$.
    Somit $n^r\leq m$, also $r\leq \frac{\log(m)}{\log(n)}$.
    Weiter $\|a_i\| \leq a_i \|1\| = a_i<n$.
    Es folgt $$\|m\|\leq \sum_{i=0}^r \|a_i\| \cdot \|n\|^i\leq \sum_{i=0}^r n \cdot \|n\|^r \leq \left(1+\frac{\log(m)}{\log(n)}\right) \cdot n \cdot \|n\|^{\frac{\log(m)}{\log(n)}}$$
    Wir ersetzen in dieser Ungleichung $m$ durch $m^k$ und ziehen die $k$-te Wurzel:
    $$\|m\| \leq \sqrt[k]{\left(1+\frac{k\cdot\log(m)}{\log(n)}\right)}\cdot n^{1/k} \cdot \|n\|^{\frac{\log(m)}{\log(n)}}$$
    $\Longrightarrow$ $\|m\|^{\frac{1}{\log(m)}} \leq \|n\|^{\frac{1}{\log(n)}}$\\
    Da die Rollen von $m$ und $n$ vertauschbar sind, folgt Gleichheit.
\end{proof}
\subsection{Vervollständigungen}
\begin{definition}
    Ein Körper mit Betrag ist \emph{vollständig}, wenn jede Cauchyfolge\footnote{Eine Folge $(a_n)_{n\in\N}$ ist eine \emph{Cauchy-Folge}, wenn $\forall\epsilon>0\;\exists N\in \N\;\forall m,n\geq N\colon |a_m - a_n| <\epsilon$} in $K$ konvergiert.
\end{definition}
\begin{example} \label{theo:4.18}$ $
    \begin{itemize}
        \item $(R,\gaf_\infty)$, $(\C, \gaf_\C)$ sind vollständig
        \item $(\Q, \gaf_5)$ nicht vollständig\footnote{Das Argument gilt für alle Primzahlen $p\equiv 1\mod 4$}
        Sei $a_k + 5^k\Z$ ein Element in $(\Z/5^k\Z)^\times$ der Ordnung $4$.\footnote{Wegen $\phi(5^k) = 5^k-5^{k-1}  = 5^{k-1} \cdot 4$ ($k\geq 2$) und der Tatsache, dass $(\Z/5^k\Z)^\times$ yzklisch ist gilt, dass $(\Z/5^k\Z)^\times \cong \Z/4\Z \times \Z/5^{k-1}\Z$. Bei Hartnick gab es mehr dazu.}
        Wir können erreichen, dass $a_k\equiv a_{k+1}\mod 5^k$
        => $|a_k-a_{k+1}|_5 < \frac{1}{5^k}$, also ist $(a_k)_k$ eine Cauchyfolge bezüglich $\gaf_5$.

        $a_k + 5^k\Z$ hat Ordnung\footnote{Ordnung $ord_G(a) = \min\{i\in\N\mid a^i = 1\}$} $4$, also $a_k^2 \equiv -1 \mod 5^k$ $\Longrightarrow$ $\lim_{k\rightarrow \infty} a_k^2 = -1$ bez $\gaf_5$.
        Also müsste ein Grenzwert $a=\lim_{k\rightarrow \infty} a_k$ in $\Q$ die Gleichung $a^2= -1$ erfüllen \Lightning
        \TODO[Check for corectness]
    \end{itemize}
\end{example}

\begin{flushright}
VL vom 18.1.2024:
\end{flushright}

\begin{definition}% K^ statt tilde
    Sei $(K,\gaf)$ ein Körper mit Betrag. Eine \emph{Vervollständigung} von $(K,\gaf)$ ist ein Tripel $(\hat{K},\ggaf,i)$ bestehend aus:
    \begin{itemize}
        \item einem vollständigen Körper $\hat{K}$ mit Betrag $\ggaf$
        \item einem Homomorphismus $i:K \hookrightarrow \hat{K}$ so, dass
        \begin{enumerate}
            \item $\|i(x)\| = |x|$ für alle $x\in K$ ("betragserhaltend")
            \item $i(K)\subseteq \hat{K}$ ist dicht\footnote{
            Eine Menge $M\subseteq X$ ist dicht in $X$, wenn eine der folgenden äquivalenten Aussagen zutrifft:
            \begin{itemize}
                \item Zu jedem $x\in X$ und jedem $r>0$ existiert ein Punkt $y\in M$, sodass $|x-y|<r$
                \item Zu jedem $x\in X$ existiert eine Folge $(x_n)_{n\in\N}$ von Punkten aus $M$, sodass $\lim_{n\rightarrow\infty} x_n = x$.
            \end{itemize}
            }
        \end{enumerate}
    \end{itemize}
\end{definition}

\begin{theorem}
    Jeder Körper mit Betrag $(K,\gaf)$ besitzt eine Vervollständigung $(\hat{K}, \ggaf,i)$, die bis auf kanonische Isomorphie eindeutig ist.
\end{theorem}
Letzteres bedeutet:
Ist $(\hat{K}_2, \ggaf_2,i_2)$ eine andere Vervollständigung von $(K,\gaf)$, dann existiert ein Isomorphismus $\phi:\hat{K}\overset{\cong}{\rightarrow} \hat{K}_2$ so, dass 1. \TODO[Bild] kommutiert und 2. $\phi$ ist betragserhaltend.

Für den Beweis im Falle der $p$-adischen Betrags siehe Goueva: "p-adic numbers" S. 64-68.
Der Beweis von 4.20 verläuft genauso wie die Konstruktion von $\R$ durch Cauchyfolgen in der Analysis.
\begin{proof}[Beweisskizze]$ $\\
    Zur Eindeutigkeit:\\
    Für $x\in \hat{K}$ wähle $(x_n)_n$ in $K$ mit $\lim_{n\rightarrow \infty} i(x_n) = x$ (Dichtheit).
    Setze $\phi(x)\coloneqq \lim_{n\rightarrow\infty} i_2(x_n)\in \hat{K}_2$.
    Grenzwert ex., da $(i(x_n))_n$ genau dann eine Cauchyfolge ist, wenn $\left(i_2(x)\right)_n$ eine ist, weil $i$, $i_2$ betragserhaltend sind.
    Zu zeigen:
    \begin{itemize}[noitemsep]
        \item $\phi$ Homomorphismus
        \item $\phi$ betragserhaltend
        \item die Umkehrabbildung wird ähnlich konstruiert
    \end{itemize}

    \noindent
    Zur Existenz:\\
    Sei $\mathcal{R}\coloneqq \{\text{Cauchyfolgen in $K$}\}$
    Durch punktweise Addition und Multiplikation wird $\mathcal{R}$ ein Ring mit Null $(0)_n$ und Eins $(1)_n$.
    Sei $I = \{(x_n)_n\in \mathcal{R}\mid \lim_{n\rightarrow \infty} x_n = 0\}$.
    Dann ist I ein Ideal.
    Setze $\hat{K}\coloneqq \mathcal{R}/I$.
    $\hat{K}$ ist ein Körper: Sei $x+I\in \hat{K}\setminus \{0\}$ d.h.
    $x= (x_n)_n$ mit $(x_n)_n \notin I$.
    Insbesondere existiert $n_0\in\N$ mit $x_n\neq 0$ für $x_n>n_0$.
    Setze $y_n\begin{cases}
        1\quad&n\leq n_0\\
        x_n^{-1}\quad &n>n_0
    \end{cases}$.
    Dann ist $(y_n)_n$ eine Cauchyfolge.
    Also $(y_n)_n\in \R$.
    Weiter gilt $(x_n)_n\cdot(y_n)_n = (x_n\cdot y_n)_n = (1)_n + (u_n)_n$ mit $u_n =0$ für $n>n_0$, $(u_n)_n\in I$.
    $\Longrightarrow$ $(y_n)_n+I$ Inverses von $(x_n)_n+I$ in $\hat{K}$.
    Der Betrag auf $\hat{K}$ ist definiert durch $\|(x_n)_n+I\| \coloneqq \lim_{n\rightarrow \infty} |x_n|$ (wohldefiniert).

    Definiere $i(x) = (x)_n+I\in \hat{K}$.
    \begin{itemize}[noitemsep]
        \item $i$ offensichtlich betragserhaltend
        \item $i(K)\subseteq\hat{K}$ dicht: Ist $(x_n)_n$ eine Cauchyfolge in $K$, dann $$\lim_{n\rightarrow\infty} i(x_n) = (x_n)_n+I$$.
        \item $\hat{K}$ ist vollständig:
        Ist $(y_n)_n$ eine Cauchyfolge in $\hat{K}$, dann wählen wir $x_n\in K$ mit $\|y_n-i(x_n)\|<\frac{1}{n}$.
        Dann gilt $$\lim_{n\rightarrow\infty} y_n = \lim_{n\rightarrow\infty} i(x_n) = (x_n)_n +I$$.
    \end{itemize}
\end{proof}

\begin{definition}
    Die Vervollständigung von $\Q$ bezüglich des p-adischen Betrags $\gaf_p$ nennt man den \emph{Körper der $p$-adischen Zahlen} $\Q_p$
\end{definition}

\begin{remark}
    Es gilt folgender Satz:\\
    Ein Körper, der vollständig bez. eines archimdischen Betrags ist, ist isomorph zu $\R$ oder $\C$. Der Betrag ist äquivalent zum gewöhnlichen Betrag.
\end{remark}

\subsection{Bewertungen und Bewertungsringe}
\begin{theorem}\label{theo:4.23}
    Sei $K$ ein Körper mit Bewertung $\nu:K\rightarrow\R\cup \{\infty\}$.
    Dann ist $$\mathcal{O}_\nu = \{x\in K\mid \nu(x)\geq0\}$$ [$\mathcal{O}_\nu = \{x\in K\mid \|x\|_\nu \leq 1\}$ für $\|x\|_\nu = a^{-\nu(x)}$] ein Ring (genannt \emph{Bewertungsring}).
    Der Ring $\mathcal{O}_\nu$ hat genau ein Maximales Ideal $$\mathcal{m}_\nu = \{x\in K\mid \nu(x)>0\}$$ [$\mathcal{m}_\nu = \{x\in K\mid \|x\|_\nu <1\}$]
    und die Einheitengruppe $$\mathcal{O}_\nu^\times = \{x\in K\mid \nu(x) = 0\}$$ [$\mathcal{O}_\nu^\times = \{x\in K\mid \|x\|_\nu = 1\}$]
\end{theorem}
\begin{proof}
    $\mathcal{O}_\nu$ ist ein Ring wegen der ultrametrischen Dreicksungleichung.
    Die Aussage über $\mathcal{O}_\nu^\times$ folgt aus $\|x^{-1}\|_\nu = \|x\|_\nu^{-1}$.
    $\mathcal{m}_\nu$ ist ein Ideal: Für $a\in \mathcal{O}_\nu$ und $x\in \mathcal{m}_\nu$ ist $\|ax\|_\nu = \|a\|_\nu \cdot \|x\|_\nu < 1$
    Für $x,y\in \mathcal{m}_\nu$ ist $\|x+y\|_\nu \leq \max\left\{\|x\|_\nu, \|y\|_\nu\right\} < 1$.

    Sei $\emptyset\neq I\subset \mathcal{O}_\nu$. Dann $I\cap \mathcal{O}_\nu^\times = \emptyset$, sonst $1\in I$ und $I=\mathcal{O}_\nu$.
    $\Longrightarrow$ $I\subseteq \mathcal{m}_\nu$.
    Somit ist $\mathcal{m}_\nu$ das eindeutige maximale Ideal.
\end{proof}

\begin{definition}
    Ein kommultiver Ring $R$ mit genau einem maximalen Ideal $\mathcal{m}$ heißt \emph{lokaler Ring}. Der Körper $R/\mathcal{m}$ ist der \emph{Restklassenkörper} vorn $R$.
\end{definition}

\begin{example}
    Betrachet $\Q$ mit der $p$-adischen Bewertung $\nu_p$.
    $$\mathcal{O}_{\nu_p} =\left\{\frac{a}{b}\in \Q \mathrel{\Big|} a,b\in \Z, p\nmid b\right\} = \Z_{(p)}\footnote{Siehe A3 auf Blatt 10}$$
    $$\mathcal{m}_{\nu_p} = \left\{\frac{a}{b}\in \Q \mathrel{\Big|} p|a, p\nmid b\right\} = p\cdot \mathcal{O}_{\nu_p} =p\Z_{(p)}$$% algihn
    Weiter ist $\mathcal{O}_{\nu_p}/\mathcal{m}_{\nu_p} \overset{\cong}{\longrightarrow} \F_p$
    $$\left[\frac{a}{b}\right] \mapsto \overline{a} \cdot \overline{b}^{-1}$$ % align stuff
    
\end{example}

\begin{lemma}\label{theo:4.26}
    Sei $R$ ein lokaler Ring mit maximalem Ideal $\mathcal{m}$. Dann gilt $R^\times = R\setminus\mathcal{m}$.
\end{lemma}
\begin{proof} $ $
    \begin{itemize}
        \item[$\subseteq$]
        $x\in R^\times\;\Rightarrow\;xR = R\;\Rightarrow\;x\notin \mathcal{m}$
        \item[$\supseteq$]
        $x\in R\setminus \mathcal{m}$. Dann ist das Ideal $xR$ nicht in $\mathcal{m}$ enhalten.
        Da $\mathcal{m}$ das einzige maximale Ideal und jedes Ideal $\neq R$ in einem maximalen Ideal enthalten ist, ist $xR = R$ $\Longrightarrow\; x\in R^\times$
    \end{itemize}
\end{proof}

\begin{flushright}
VL vom 19.1.2024:
\end{flushright}

\begin{definition}
    Man nennt eine Bewertung \emph{diskret}, wenn sie Werte in $\Z\cup \{\infty\}$ annimmt.
    Ein Element $\pi$ mit Bewertung $\nu(\pi) = 1$ heißt \emph{uniformisierndes Element}.
\end{definition}
\begin{remark} \label{theo:4.28}
    Sei $\pi$ uniformisierendes Element von $K$ mit Bewertung $\nu$.
    Dann kann man jedes $x\in K^\times$ schreiben als $x=\pi^m\cdot u$ mit $m=\nu(x)$, $u\in \mathcal{O}_\nu^\times$, denn $\nu(\pi^m x^{-1}) = 0$ also $\pi^m x^{-1} \in \mathcal{O}_\nu^\times$ und damit existiert $\mathcal{O}_\nu^\times\ni(\pi^m x^{-1})^{-1} = x\pi^{-m}\eqcolon u$.
    
\end{remark}

\begin{theorem}%Satz und Definition
    Für einen Kommutativen Ring $A$ sind folgende Aussagen äquivalent:
    \begin{enumerate}[label=(\alph*)]
        \item $A$ ist Bewertungsring einer diskreten (nicht-trivialen??) Bewertung auf einem Körper.
        \item $A$ ist ein lokaler Hauptidealring, aber kein Körper
    \end{enumerate}
    Jeder Ring, der diese Bedingungen erfüllt, heißt \emph{diskreter Bewertungsring}.
\end{theorem}
\begin{proof} $ $
    \begin{itemize}
        \item[(a) $\Rightarrow$ (b)]
        Sei $A=\mathcal{O}_\nu$ für eine diskrete Bewertung $\nu$ auf $K$.
        Nach \ref{theo:4.23} ist $A$ lokaler Ring
        $A$ ist kein Körper, weil ein uniformisierendes Element nicht invertierbar ist.\TODO[Wieso muss ein solches Element existieren? Jeder nicht-triviale Betrag lässt sich zu einem äquivalenten Betrag machen, der ein uniformisierendes Element hat (mit ggt der möglichen Werte 1 oder nicht ein und dann einfach durch den ggt Teilt)]
        Sei $I\subseteq$ ein Ideal mit $I\neq (0)$.
        Beh: $I=\pi^n\cdot A$ für ein $n\in \N$, wobei $\pi$ uniformisierend.
        Sei $y\in A\setminus\{0\}$. Schreibe $y=\pi^m\cdot u$ mit $m=\nu(y)$, $u\in A^\times = \mathcal{O}_\nu^\times$.
        Daher gilt $y\in I\Leftrightarrow u^{-1}y\in I \Leftrightarrow \pi^m = \pi^{\nu(y)}\in I$.
        Sei $n\in \N_0$ minimal mit $\pi^n\in I$.
        Dann $I = \pi^n\cdot A = \{x\in A\mid \nu(x)\geq n\}$.
        
        \item[(b) $\Rightarrow$ (a)]
        Setze $K\coloneqq Quot(A)$. Sei $\mathcal{m}\subseteq A$ das (eindeutige) maximale Ideal.
        Sei $\pi\in \mathcal{m}$ ein Erzeuger, d.h. $\pi A = \mathcal{m}$.
        $A$ ist kein Körper $\Longrightarrow$ $\mathcal{m}\neq (0)$ $\Longrightarrow$ $\pi\neq 0$.
        $\mathcal{m}$ ist auch das einzige Primideal, da in Hauptidealringen Primideale maximal sind.\\
        Eindeutige Primfaktorzerlegung und \ref{theo:4.26}\footnote{\TODO[Wieso gibt es eine solche Primfaktorzerlegung?] Gegeben eine Primfaktorzerlegung $a=\epsilon\cdot q_1\cdot\dots q_n$ mit $\epsilon\in A^\times$ und $q_i\in A\setminus A^\times$ irreduzibel, ist nach \ref{theo:4.26} $q_i\in \mathcal{m}$ $\Rightarrow$ $q_i = a_i \cdot \pi$. Wenn $a_i\notin A^\times$, wäre $q_i = \underbrace{a_i}_{\notin A^\times}\cdot \underbrace{\pi}_{\notin A^\times}$ nicht irreduzibel \Lightning $\Rightarrow\ a_i\in A^\times$. Damit ergibt sich $a=\underbrace{\epsilon\cdot a_1\cdot \dots \cdot a_n}_{=:u\in A^\times} \cdot \pi^{n}$}:
        $$A\setminus\{0\} \ni a = \pi^n\cdot u\text{ mit }n\in \N_0\text{, }u\in A^\times$$
        Jedes $x\in K\setminus\{0\}$ schreibt sich eindeutig als
        $$x = \pi^m\cdot u,m\in \Z, u\in A^\times$$
        Definiere $\nu(x)\coloneqq m = \max \{k\in \Z\mid x\in \pi^kA\}$ und $\nu(0) = \infty$.
        $\nu$ ist eine Bewertung auf $K$ (A3 auf Blatt 10).
        Es ist dann $A=\mathcal{O}_\nu$.
    \end{itemize}
    
\end{proof}

\begin{remark}
    Sei $K$ ein Körper mit diskreter Bewertung $\nu$.
    Sei $|x| = a^{-\nu(x)}$ ein Betrag mit $a>1$.
    Eine Folge $(x_n)_n$ in $K$ ist eine Cauchy bezüglich $\gaf$ gdw.
    $$\forall B>0\ \exists N\in \N\ \forall n,m>N: \nu(x_n-x_m)>B$$
    Ist $K$ vollständig bezüglich $\gaf$, dann ist $K$ vollständig bezüglich aller von $\nu$ induzierten Beträge, und wir sagen, dass $K$ \emph{vollständig bez. $\nu$} ist.
\end{remark}

\begin{theorem}[Satz über Reihendarstellung]
    Sei $K$ vollständig mit disketer Bewertung $\nu$.
    Seien $\pi\in \mathcal{O}_\nu$ ein uniformisierendes Element und $\Omega\subseteq \mathcal{O}_\nu$ ein Repräsentantensystem für $\mathcal{O}_\nu/\mathcal{m} = \mathcal{O}_\nu/\pi \mathcal{O}_\nu$.
    \begin{enumerate}[label=(\alph*)]
        \item Jede Laurentreihe $\sum_{n=m}^\infty a_n\cdot \pi^n$ mit $m\in \Z, a_n\in \Omega$, konvergiert in $K$.
        \item Jedes $x\in K^\times$ lässt sich eindeutig schreiben als $x=\sum_{n = \nu(x)}^\infty a_n\cdot \pi^n$ mit $a_n\in \Omega$.
    \end{enumerate}
\end{theorem}

\begin{proof}
    \begin{enumerate}[label=(\alph*)]
        \item Betrachte die Partialsumme $S_k = \sum_{n=m}^k a_n\cdot \pi^n$.
        Z.z. $(S_k)_{k\in\N}$ ist eine Cauchyfolge bezüglich $\nu$.
        \begin{align*}
            \nu(S_k-S_l) =& \nu\left(\sum_{n=l+1}^k a_n\cdot \pi^n\right)
            \geq \min\{\nu\left(a_n\pi^n\right)\mid l+1\leq n\leq k\}\\
            =& \min\{n\cdot \underbrace{\nu(\pi)}_{=1}+\underbrace{\nu(a_n)}_{\geq 0}\mid l+1\leq n \leq k\} 
            \geq \min\{n\mid l+1\leq n \leq k\} >l
        \end{align*}
        impliziert Cauchy.
        \item Wir zeigen per Induktion zunächst Existenz: 
        Es gibt $a_{\nu(x)},\dots,a_k\in \Omega$ mit $\nu\left(x-\sum_{n=\nu(x)}^k a_n\pi^n\right)\geq {k+1}$.\\
        Induktionsanfang $k<\nu(x)$: 
        $\nu(x-0)\geq k+1$\checkmark.\\
        Induktionsschritt: Seien $a_{\nu(x)}, \dots, a_k$ wie oben bekannt.
        Es gibt ein eindeutiges ${a_{k+1}\in \Omega}$ mit $(\underbrace{x-\sum_{n=\nu(x)}^k a_n\pi^n}_{\in \pi^{k+1}\mathcal{O}_\nu\text{ nach \ref{theo:4.28}}})\pi^{-(k+1)}-a_{k+1}\in\pi \mathcal{O}_\nu$.\footnote{Wähle $a_{k+1}$ sodass es die passende nebenklasse in $\pi\mathcal{O}_\nu$}\\
        $\Longrightarrow$ $x-\sum_{n=\nu(x)}^{k+1} a_n\pi^n\in \pi^{k+2}\mathcal{O}_\nu$
        $\Longrightarrow$ $\nu(x-\sum_{n=\nu(x)}^{k+1} a_n\pi^n)\geq k+2$

        Zur Eindeutigkeit:
        Sei $x=\sum_{n=\nu(x)}^\infty b_n \pi^n$ eine andere solche Darstellung.
        Sei $k$ minimal mit $a_k\neq b_k$.\\
        $\Longrightarrow$ $0=(a_k-b_k)\pi^k + r\pi^{k+1}$ für ein $r\in \mathcal{O}_\nu$.\\
        $\Longrightarrow$ $a_k\equiv b_k \mod \pi\mathcal{O}_\nu$.\\
        $\Longrightarrow$ $a_k = b_k$, weil $\Omega$ ein Repräsentantensystem von $\mathcal{O}_\nu/\pi \mathcal{O}_\nu$ ist.
    \end{enumerate}
\end{proof}

\begin{lemma}[Henselsches Lemma]
    Sei $K$ ein vollständiger Körper mit diskreter Bewertung $\nu$. Sei $q: \mathcal{O}_\nu\rightarrow \mathcal{O}_\nu/\mathcal{m}_\nu$ die Projektion.
    Sei $f\in \mathcal{O}_\nu[X]$. Hat $q_*(f)\in \mathcal{O}_\nu/\mathcal{m}_\nu [X]$ eine einfache Nullstelle in $\mathcal{O}_\nu/\mathcal{m}_\nu$, dann hat $f$ eine Nullstelle in $\mathcal{O}_\nu$.
\end{lemma}
\begin{example*}
    Betrachte $f(x)= x^2+1\in\Q_5[X]$ (siehe \ref{theo:4.18}).
    $q_*(f)(x) = x^2+1\in \F_5[X]$ hat Nullstelle $\overline{2}\in \F_5$. Diese ist einfach, weil $D_{q_*}(f)(\overline{2}) = 2\cdot \overline{2}\neq 0$ in $\F_5$.
    Nach Hensel hat $f$ eine Nullstelle in $\Z_5$.
\end{example*}
\begin{proof}
    Wir zeigen induktiv:
    Für alle $n\in \N$ gibt es $a_n\in \mathcal{O}_\nu$ gilt
    \begin{enumerate}[label=(\roman*),noitemsep]
        \item $f(a_n) \equiv 0 \mod \pi^n\mathcal{O}_\nu$
        \item $a_n\equiv a_{n+1} \mod \pi^n\mathcal{O}_\nu$.
    \end{enumerate}
    Wähle $a_1\in \mathcal{O}_\nu$ so, dass $q_*(f)(\overline{a}_1) = 0\in \mathcal{O}_\nu/\pi \mathcal{O}_\nu \Leftrightarrow f(a_1) \equiv 0\mod \pi \mathcal{O}_\nu$.
    Angenommen wir haben $a_n\in \mathcal{O}_\nu$ wie oben.\\
    Ansatz: $a_{n+1} = a_n + \pi^n \cdot b$ für $b\in \mathcal{O}_\nu$.\\
    Beobachtung:
    \begin{enumerate}[noitemsep]
        \item $(a_n+\pi^n\cdot b)^m = a_n^m + m\cdot \pi^n\cdot b\cdot a_n^{m-1} \mod \pi^{n+1}\mathcal{O}_\nu$.\\
        $\Longrightarrow$ $f(a_n+\pi^nb) \equiv f(a_n) + \pi^n\cdot b\cdot D(f(a_n)) \mod \pi^{n+1} \mathcal{O}_\nu$.
        \item Da $\overline{a}_1 \in \mathcal{O}_\nu/ \pi \mathcal{O}_\nu$ eine einfache Nullstelle von $q_* f$, gilt $0\neq q_* D(f)(\overline{a}_1) \overset{(ii)}{=} q_* D(f)(\overline{a}_n)$. Somit $D(f)(a_n)\in \mathcal{O}_\nu^\times$.
    \end{enumerate}
    
    Nach (i) ist $f(a_n) = \pi^n\cdot u$ mit $u\in \mathcal{O}_\nu$. Setze $b\coloneqq -u\cdot D(f)(a_n)^{-1}$.
    Dann $f(a_{n+1}) \equiv 0 \mod \pi^{n+1}\mathcal{O}_\nu$ 
\end{proof}

Fortsetzung des Beispiels? (Lose formatiert)\\
$a_1=2$ \\
$(2+b\cdot 5)^2\equiv -1 \mod 5^2$\\
$4+4\cdot 5\cdot b \equiv -1 \mod 5^2$\\
=> $5^2| 5+4\cdot 5\cdot b = 5(1+4\cdot b)$\\
<=> $5| 1+4\cdot b$\\
$b=1$ ist Lösung\\

$a_2 = 2 \cdot 5^0+1\cdot 5^1$\\
$(2+1\cdot 5 + c\cdot 5^2)^2\equiv-1 \mod 5^3$\\
$7^2+2\cdot 7\cdot c \cdot 5^2\equiv -1\mod 5^3$\\
<=> $50+2\cdot 7 \cdot c \cdot 5^2$\\
<=>
$5| 2+2\cdot 7\cdot c$\\
$c = 2$ Lösung


$a_3 = 2\cdot 5^0+ 1\cdot 5^1 + 2\cdot 5^3+\dots$

\begin{flushright}
    14.1.2024
\end{flushright}


\begin{theorem}
    Sei $K$ ein vollständiger Körper bezüglich einer Bewertung $\nu$.
    Dann besitzt $\nu$ eine eindeutige Bewertungsfortsetzung auf jeder endlichen Erweiterung $L|K$.
    Diese ist durch $$\nu_L(\alpha) = \sqrt[n]{\nu\left(N_{L|K}(a)\right)}$$ gegeben, wobei $n=[L:K]$. Weiter ist $L$ bezüglich $\nu_L$ vollständig.
\end{theorem}
Bachte, dass für $\alpha\in K$:
$N_{L|K}(\alpha) = \alpha^n $ und
$\nu(N_{L|K}(\alpha)) = \nu(\alpha)^n$.

\begin{remark}[Die $p$-adischen Zahlen]
    $\Q_p$ mit $\nu_p$ bezüglich $\gaf_p$ $p$-adische (rationale) Zahlen.
    Einnerung: $|x|_p = p^{-\nu_p(x)}$, wobei ${\nu_p}_{|\Q}$ gegeben ist durch $\nu_p(p^n\frac{a}{b}) = n$ mit $p\nmid a$, $p\nmid b$.
    $p$ ist uniformisierendes Element.

    Der Bewertungsring $\Z_p\coloneqq \mathcal{O}_{\nu_p}$ heißt der \emph{Ring der ganzen $p$-adischen Zahlen}.
    Es gilt $$\Z_p\cap \Q = \{x\in\Q\mid \nu_p(x)\geq 0\} = \Z_{(p)}\footnote{$A_{(p)} \coloneqq \left\{x\in K\mid x = \frac{a}{b}\text{ mit } a\in A \text{ und } b\in A\setminus (p)\right\}$ nach Blatt 10 Aufgabe 3}$$
    Teilmenge $\Z_p\subseteq\Q_p$ ist offen und abgeschlossen, denn:
    $$\Z_p = \{x\in\Q_p\mid |x|_p \leq 1\} = \overline{B_1(0)} \text{ und }\Z_p = \{x\in\Q_p\mid |x|_p<p\} = B_p(0)$$

\TODO[rework from here]
    Da $\Z_p\subseteq \Q_p$ offen und $\Q\subseteq \Q_p$ dicht, ist auch $\Z_{(p)} = \Z_p\cap \Q \subseteq \Z_p$ dicht.
    Es ist sogar $\Z$ in $\Z_p$ dicht.
    Sei $x\in \Z_p$ und $\epsilon>0$. Dann gibt es $\frac{a}{b}\in \Z_{(p)}$ mit $\card{x-\frac{a}{b}}<\epsilon$.
    Sei $\overline{b}\in \Z$ so, dass $\overline{b}\cdot b \equiv 1\mod p^n$ für $n\in \N$ mit $p^{-n} < \frac{\epsilon}{|a|_p}$. Dann $|x-\underbrace{a\overline{b}}_{\in \Z}|_p \leq \max \{\card{x-\frac{a}{b}}_p, \card{\frac{a}{b}-a\overline{b}}_p\}$.
    Wir haben $|\frac{a}{b}-a\overline{b}|_p = |a|_p\cdot |\frac{1}{b}-\overline{b}| \leq\footnote{\TODO[Sollte das nicht = sein? $|b|_p$ sollte ja 1 sein]} |a|_p \cdot |b|_p^{-1} \cdot |\frac{1}{b}-\overline{b}|_p = |a|_p\cdot |1-b\overline{b}|_p \leq |a|_p \cdot p^{-n} < \epsilon$\\
    $\Longrightarrow$ $|x-a\overline{b}|_p < \epsilon$

    Für den Restklassenring der Körper $\Q, \Q_p$ gilt:
    $$\F_p\cong \Z_{(p)}/p\Z_{(p)}\overset{\cong}{\longrightarrow} \Z_p/p\cdot \Z_p$$
    Die Surjektivität und damit Bijektivität/Isomorphie folgt aus der Tatsache, dass jede Nebenklasse $x+p\Z_p$ offen ist und somit Elemente aus $\Z_{(p)}$ enthält.\TODO[Injektivität?]

    Reihendarstellung( 4.31):
    Jedes Element $x\in \Q_p$ lässt sich eindeutig schreiben als $$x=\sum_{i=\nu_p(x)}^\infty a_i\cdot p^i\text{, } a_i\in \{0,\dots,p-1\}$$
    Bsp: $-1 = (p-1) + (p-1) \cdot p + (p-1)\cdot +p^2+\dots$
    sieht man eintweder\\ durch $\underbrace{(p-1)+\dots + (p-1) p^n}_{(p-1)(1+\dots+p^n)} = p^{n+1}-1\equiv -1\mod p^{n+1}$ oder\\ durch geometrische Reihe $\frac{-1}{p-1}=\sum_{i=0}^\infty p^i$.

    $\Q_p$ enthält primitive $(p-1)$-ste Einheitswurzeln:
    Sei $\Phi_{p-1}\in \Z[x]\subset \Z_p[x]$ das $(p-1)$-te Kreisteilungspolynom.
    Da $p\nmid (p-1)$ und $\Phi_{p-1}|X^{p-1}-1$ ist $\Phi_{p-1}$ separabel über $\F_p$.
    Weiter besitzt $\F_p$ eine primitive $(p-1)$-te Einheitswurzel.
    => $\Phi_{p-1}$ besitzt eine einfache Nullstelle über $\F_p$.
    Hensels Lemma: $\Phi_{p-1}$ hat eine Nullstelle in $\Z_p\subseteq \Q_p$.
\end{remark}

\begin{theorem} % Hier wurde \nu durch m ersetzt
    Sei $F(x_1,\dots, x_n)\in \Z[x_1,\dots,x_n]$.
    Sei $p$ prim. Die Kongruenz $$F(x_1,\dots,x_n) \equiv 0\mod p^m$$
    ist genau dann für begliebiges $m \in \N$ lösbar, wenn $$F(x_1,\dots,x_n)=0$$ in $\Z_p$ lösbar ist.
\end{theorem}
\begin{proof}$ $
    \begin{itemize}
        \item[$\Leftarrow$]
        Sei $(x_1,\dots,x_n)\in \Z_p^n$ eine Lösung. 
        Betrachte Reihendarstellung $x_i=\sum_{\alpha=0}^\infty a_{i,\alpha} p^\alpha$.
        Dann ist $(\underbrace{\sum_{\alpha =0}^{m-1} a_{1,\alpha} p^\alpha}_{\in\Z}, \dots,\underbrace{\sum_{\alpha = 0}^{m-1} a_{n,\alpha}p^\alpha}_{\in\Z})$
        eine Lösung von $F$ modulo $p^m$.
        \item[$\Rightarrow$]
        Sei umgekehrt $(x_1^m, \dots, x_n^m) \in \Z^n$ eine Lösung modulo $p^m$.
        \obda\footnote{Da Validität des \obda zeigt sich erst im Laufe des Beweises.} sei $n=1$, also $(x^m)_{m\in\N}$ Lösung von $F(x) \equiv 0\mod p^m$.
        Wähle Teilfolge $(y_1^m)_m$ von $(x^m)_m$ und $y_1\in \Z$ mit $$y_1^m \equiv y_1\mod p$$
        $$F(y_1^m)\equiv 0 \mod p$$
        Eine solche Teilfolge muss existieren, da es nur $p$ mögliche Reste für die $x_i$ gibt und so mindestens einen Rest $y_1$ unendlich häufig volkommt und eine Teilfolge gebildet werden kann.
        
        Im nächsten Schritt wähle eine Teilfolge $(y_2^m)_m$ von $(y_1^m)_m$, die $\mod p^2$ konstant ist, d.h. $$y_2^m\equiv y_2\mod p^2$$
        $$F(y_2^m) \equiv 0\mod p^2$$
    
        Dann konvergiert die Folge $y_1,y_2,\dots$ gegen eine Lösung von $F$ in $\Z_p$.
    \end{itemize}
    
\end{proof}

\TODO[Vorlesung vom 26.01 kommt sobald ich sie selber nachgearbeitet habe]
\end{document}