\documentclass[../main.tex]{subfiles}
\graphicspath{{\subfix{../images/}}}
\begin{document}
\begin{flushright}
VL vom 26.1.2024:
\end{flushright}
\subsection{Grundlagen der Modultheorie}
Sei $R$ ein Ring mit Eins (aber nicht zwingend kommutativ)

\begin{definition}
    Ein \emph{Linksmodul} über $R$ ist eine abelsche Gruppe $(M,+)$ zusammen mit einer Skalarmultiplikation
    $$R\times M\rightarrow M,\quad (r,m)\mapsto r * m$$
    mit den Eigenschaften
    \begin{enumerate}[label=(\roman*)]
        \item $r\cdot (x+y) = r\cdot x + r\cdot y$
        \item $(r+s)* x = r * x + s * x$
        \item $(r\cdot s) * x = r * (s*x)$
        \item $1_R * x = x$
    \end{enumerate}
    für alle $r,s\in R$ und $x,y\in M$.\\
    Analog ist ein \emph{Rechtsmodul} definiert mit Skalarprodukt $M\times R\rightarrow M\quad (m,r)\mapsto m * r$
\end{definition}

\begin{remark}
    Auf der $R$ zugrundeliegenden abelschen Grupp erhalten wir eine neue Ringstruktur, indem wir die Multiplikation durch $r\odot s \coloneqq s\cdot r$ ersetzten.
    Man nennt diese den entgegengesetzten Ring $R^{Op}$. Ist $M$ bezüglich $M\times R\rightarrow M$ ein $R$-Rechtsmodul, dann ist $M$ bezüglich $R^{Op}\times M\rightarrow M, r\odot m= m\cdot r$, ein $R^{Op}$-Linksmodul.
\end{remark}
Wir betrachten in Zukunft Linksmoduln und nennen sie einfach Moduln.
\begin{example} $ $
    \begin{enumerate}[label=(\arabic*)]
        \item $R=K$ Körper. Dann sind $K$-Moduln genau die $K$-Vektorräume
        \item $M$ abelsche Gruppe
        $$End(M) = \{f:M\rightarrow M \text{ Homomorphismus}\}$$
        ist ein Ring bezüglich $(f+g)(x) = f(x) + g(x)$ und $(f\cdot g)(x) = f(g(x))$ mit der Skalarmultiplikation $End(M)\times M \rightarrow M\quad (f,m)\mapsto f(m)$ wird $M$ ein $End(M)$-Modul.
        \item Abesche Gruppen sind genau die $\Z$-Moduln:
        Ist $M$ eine abelsche Gruppe, dann ist $M$ ein $\Z$-Modul vermöge
        $$n\cdot x = \underbrace{x+\dots+x}_{n\text{-fach}} \quad (-1)\cdot x = -x$$
    \end{enumerate}
\end{example}

\begin{definition}$ $
    \begin{enumerate}[label=(\alph*)]
        \item Eine Abbildung $f:M\rightarrow N$ zwischen $R$-Moduln ist \emph{$R$-Linear}, wenn für $x,y\in M$, $r\in R$ gilt
        $$f(x+y) = f(x) + f(y) \quad f(r\cdot x) = r\cdot f(x)$$
        $Hom_R(M,N) \coloneqq \{\text{$R$-lineare Abbildung $M\rightarrow N$}\}$
        \item Isomorphismus
        \item Untermodul
    \end{enumerate}
\end{definition}
\begin{example} $ $
    \begin{enumerate}[label=(\alph*)]
        \item Ist $f\in Hom_R(M,N)$, dann sind $\ker(f)$ und $f(M)$ Untermoduln.
        \item $R$ ist ein Links- und Rechtsmodul bzüglich der Ringmultiplikation $R\times R \rightarrow R$. Dann (Links-) Ideale $\hat{=}$ Untermodul (in Linksmodulstruktur). Für Rechts- analog.
    \end{enumerate}
\end{example}

Ilias (angeblich)

\setcounter{theorem}{8}
\begin{definition}
    Sei $(M_i)_{i\in I}$ eine Familie von $R$-Moduln. Das kartesische Produkt $\prod_{i\in I} M_i$ ist ein $R$-Modul bezüglich $(x_i)_{i\in I} + (y_i)_{i\in I} = (x_i + y_i)_{i\in I}$ und $r\cdot (x_i)_{i\in I} = (rx_i)_{i\in I}$.
    Wir sagen $\prod_{i\in I} M_i$ ist das \emph{direkte Produkt der $M_i$}. Für jedes $i\in I$ ist die Projektion $pr_i: \prod_{j\in I} M_i \rightarrow M_i$ $R$-linear.
    Die Zuordnung $\phi\mapsto (pr_i \circ \phi)_{i\in I}$ definiert einen Isomorphismus abelscher Gruppen
    $$Hom_R(M,\prod_{i\in I} M_i)\overset{\cong}{\rightarrow} \prod_{i\in I} Hom_R(M,M_i)$$
    (Ist $R$ kommutativ ist die Struktur wieder ein Modul)
\end{definition}
\begin{remark*}
    \TODO[Timos remark]
\end{remark*}

\begin{definition}[Direkte Summe von $R$-Modul]
    Sei $(M_i)_{i\in I}$ eine Familie von $R$-Moduln. Definiere $\bigoplus_{i\in I} M_i = \{(x_i)_{i\in I}\in \prod_{i\in I} M_i \mid x_i = 0 \text{ bis auf endlich viele $i$}\}$ die \emph{direkte Summe der $M_i$} alse Untermodul von $\prod_{i\in I} M_i$.
    Für jedes $j\in I$ ist $\iota_j : M_j \rightarrow \oplus_{i\in I} M_i$, $\iota_j (x) = (0,\dots, 0,\underbrace{x}_{\text{Position $j$}},0,\dots,0)$ eine injektive $R$-lineare Abbildung.
    Die Zuordnung $\phi \mapsto (\phi \circ \iota_i)_{i\in I}$ definiert einen Isomorphismus abelscher Gruppen
    $$Hom_R \left(\bigoplus_{i\in I} M_i,M\right)\overset{\cong}{\rightarrow} \prod_{i\in I} Hom_R(M_i, M)$$
    Die Umkehrabbildung ist 
    $$(f_i: M_i\rightarrow M)_{i\in I} \mapsto \begin{cases}
        \bigoplus_{i\in I} M_i\rightarrow M,\\
        (x_i)_{i\in I} \mapsto \sum_{i\in I} f_i(x_i)
    \end{cases}$$
    Gewöhnlich schreiben wir $(x_i)_{i\in I} \in \bigoplus_{i\in I} M_i$ auch als $\sum_{i\in I} x_i$.\\
    Notation: Ist $M_i = M$, dann schreiben wir $\bigoplus_{i\in I}M_i = M^{(I)}$ und $\prod_{i\in I} M_i = M^I$
\end{definition}

\begin{definition}
    Sei $S\subseteq M$ eine Teilmenge in einem $R$-Modul $M$. Dann heißt $S$
    \begin{itemize}
        \item \emph{Erzeugendensystem}, wenn $M\coloneqq \langle S\rangle_R = \{\sum_{s\in J} r_s\cdot s \mid J\subseteq S \text{ endlich}, r_s\in R\} = \bigcap\{N\mid N\subseteq M\text{ Untermodul mit }S\subseteq N\}$
        \item \emph{$R$-linear unabhängig}, wenn für alle $s_1,\dots,s_m \in S$ und $r_1,\dots,r_m\in R$ die Implikation $\sum_{i=1}^m r_i\cdot s_i = 0 \Rightarrow r_1 = \dots = r_m = 0$ gilt.
        \item \emph{Basis}, wenn $s$ ein linear unabhängiges Erzeugendensystem ist
        \item $M$ heißt
        \begin{itemize}
            \item \emph{endlich erzeugt}, wenn $M$ ein endliches Erzeugendensystem besitzt
            \item \emph{frei}, wenn $M$ eine Basis besitzt
        \end{itemize}
    \end{itemize}
\end{definition}

\begin{remark}$ $
    \begin{enumerate}[label=\alph*)]
        \item Jeder VR hat eine Basis, ist also frei.
        Aber $\Q$, $\Z/n$ alse $\Z$-Modul ist nicht frei:
        Betrachte $q_1 = \frac{a_1}{b_1},q_2=\frac{a_2}{b_2}\in \Q$. Dann gilt $(b_1a_2)\cdot q_1 - (b_2a_1)\cdot q_2 = 0$ (nicht triviale linearkombination).
        Somit ist $S\subseteq \Q$ mit $|S|\geq 2$ nicht linear unabhängig.
        Weiter kann ein einziges Element $q=\frac{a}{b}$ nicht $\Q$ erzeugen als $\Z$-Modul, denn $\langle q\rangle_\Z = \Z_q = \Z\cdot \frac{a}{b}\subseteq \Z\cdot \frac{1}{b} \subset \Q$
        \item Die Kardinalität einer Basis ist im Allgemeinen keine Invariante feier Moduln.
    \end{enumerate}
\end{remark}

\begin{lemma}
    Ein $R$-Modul $M$ ist frei genau dann, wenn er zu einem Modul $R^{(I)}$ für eine Menge $I$ isomorph ist.
\end{lemma}
\begin{proof} $ $
    \begin{itemize}
        \item[$\Rightarrow$]
        Sei $S$ eine Basis. Betrachte die Abbildung $R^{(S)}\overset{\cong}{\rightarrow} M,\; (r_s)_{s\in S} \mapsto \sum_{s\in S} r_s\cdot s$. Diese ist ein Isomorphismus.
        \item[$\Leftarrow$]
        $R^{(I)}$ ist frei bezüglich der Basis $\{e_i =\iota_i(1_R)\mid i\in I\}$
    \end{itemize}
\end{proof}

\begin{flushright}
    01.02.2023
\end{flushright}
\begin{definition}
    Ein Untermodul $N\subseteq M$ heißt \emph{direkter Summand}, wenn es einen Untermodul $C\subseteq M$ gibt so, dass
    \begin{enumerate}[label=(\roman*)]
        \item $C\cap N = \{0\}$
        \item $C+N=M$
    \end{enumerate}
    In diesem Fall gilt $M\cong C\oplus N$.
    Man schreibt dann $M=C \oplus N$.
\end{definition}
\begin{example*}
    $2\Z \subset \Z$ ist kein direkter Summand
\end{example*}

\begin{definition}[Exakte Folgen]
    Es seien $M,M',M''$ $R$-Moduln.
    Es seien $f:M'\rightarrow M, g:M\rightarrow M''$ $R$-lineare Abbildungen.
    man sagt, dass die Folge $$M'\overset{f}{\rightarrow} M \overset{g}{\rightarrow} M''$$
    \emph{exakt} ist, wenn $Bild(f) = \ker(g)$.

    Eine Folge der Form $$0\rightarrow M'\overset{f}{\rightarrow} M \overset{g}{\rightarrow} M'' \rightarrow 0$$ nennt man \emph{kurz exakte Folge}, wenn sie exakt an jeder Stelle ist, d.h. $f$ injektiv, $g$ surjektiv und $Bild(f) = \ker(g)$.

    Homomorphiesatz liefert, dass
    $$M/f(M') \cong M''$$.
\end{definition}
\begin{example*}
    $0\rightarrow \Z\overset{\cdot 2}{\rightarrow} \Z \rightarrow \Z/2\Z \rightarrow 0$ kurz exakt.
\end{example*}

\begin{theorem}%Satz und Definition
    Für eine kurze exakte Folge $\rightarrow M'\overset{f}{\rightarrow} M \overset{g}{\rightarrow} M'' \rightarrow 0$ sind die folgeneden Aussagen äquivalent:
    \begin{enumerate}[label=(\roman*)]
        \item $f(M')\subseteq M$ ist ein direkter Summand
        \item $\exists s\in Hom_R(M'',M): g\circ s = id_{M''}$
        \item $\exists t\in Hom_R(M,M'): t\circ f = id_{M'}$
    \end{enumerate}
    Sind diese Bedingungen erfüllt, so sagt man, dass die Folge \emph{spaltet}.
\end{theorem}
\begin{proof}$ $
    \begin{itemize}
        \item[(i)$\Rightarrow$ (ii)]
        Angenommen $f(M')\oplus C = M$.
        Dann ist $g_{|C}$ ein Isomorphismus:
        * $\ker(g_{|C}) = \ker{g}\cap C = f(M')\cap C = 0$
        * $Bild(g_{|C}) = g(C) = g(f(M')\oplus C) = g(M) = g(M) = M''$
        Definiere $s(x) = g_{|C}^{-1}(x)\in M$ für $x\in M''$.
        Dann ist $g\circ s = id_{M''}$.
        \item[(ii) $\Rightarrow$ (iii)]
        Für $y\in M$ ist $y-(s\circ g) (y) \in \ker(g)$, weil $$g(y-(s\circ g) (y)) = g(y)- (\underbrace{g\circ s}_{= id}\circ g) (y) = g(y) - g(y) = 0$$.
        Definiere $t(y) = f^{-1}(y- s\circ g (y)$ wobei $f$ als Abbildung aufs Bild batrachtet wird.
        
        Dann $t\circ f = id_{M'}$, wegen
        $t(f(x)) = f^{-1}(f(x) - s\circ \underbrace{g \circ f}_{ = 0} (x)) = x$.
        \item[(iii) $\Rightarrow$ (i)]
        Setze $C\coloneqq \ker(t)$.
        Dann ist $C\cap f(M') = \{0\}$. Sei $x\in C\cap f(M')$. Insbesonderer $x=f(y)$ für ein $y\in M'$. Dann $0 = t(x) = t(f(y)) = y$ $\Longrightarrow$ $x = f(y) = 0$.

        Weiter ist $C+f(M') = M$. Sei $m\in M$.
        Betrachet $t(m-f\circ t(m)) = t(m) - \underbrace{t\circ f}_{=id} \circ t(m) = t(m)-t(m) = 0$, also $m-\underbrace{f\circ t(m)}_{\in Bild(f)} \in C$
    \end{itemize}
\end{proof}
\begin{definition}$ $
    \begin{enumerate}[label=\alph*)]
        \item 
        Ein $R$-Modul heißt \emph{projektiv}, wenn jede kurze exakte Folge von $R$-Moduln der Form $0\rightarrow M' \rightarrow M \rightarrow P\rightarrow 0$ spaltet.
        \item 
        Ein $R$-Modul $I$ heißt \emph{injektiv}, wenn jede kurze exakte Folge von $R$-Moduln der Form $0\rightarrow I\rightarrow M \rightarrow M'' \rightarrow 0$ spaltet.
    \end{enumerate}
\end{definition}

\begin{remark}
    Jede kurze exakte Folge von VR über einem Körper $K$ spaltet, weil jeder Unter-VR ein Komplement besitzt.
    => Alle VR sind projektiv und injektiv.
\end{remark}

\subsection{Bilineare Abbildungen und Tensorprodukte}
In diesem Abschnitt ist $R$ ein kommutativer Ring.
\begin{definition}
    Seien $L,M,N$ drei $R$-Moduln.
    Eine Abbildung $f:M\times N\rightarrow L$ heißt \emph{$R$-bilinear} wenn gelten
    \begin{enumerate}
        \item $f(rx+sy,u) = rf(x,u)+sf(y,u)$
        \item $f(x,ru+sv) = rf(x,u) + sf(x,v)$
    \end{enumerate}
    für $x,y\in M$, $u,v \in N$, $r,s\in R$.
\end{definition}

\begin{definition}
    Ein \emph{Tensorprodukt} zweier $R$-Moduln $M$ und $N$ ist ein Paar $(T,\beta)$ bestehend aus einem $R$-Modul $T$ und einer $R$-bilinearen Abbildung $\beta:M\times N\rightarrow T$ so, dass eine \emph{universelle} Eigenschaft eifüllt ist:
    
    Für jeden $R$ -Modul L und jede $R$-bilneare Abbildung $f:M\times N\rightarrow L$ existiert ein eindeutiger $R$-Homomorphismus $\phi:T\rightarrow L$ mit $\phi\circ\beta = f$
    \TODO[Diagramm]
\end{definition}
\begin{lemma}
    Sind $(T,\beta)$ und $(T',\beta')$ zwei Tensorprodukte von $M\times N$, dann existiert ein eindeutiger Isomorphismus $\phi: T\rightarrow T'$ mit
    \TODO[Bild]
\end{lemma}
\begin{proof}
    \TODO[Viele viele (nicht) bunte Bilder]
\end{proof}

\begin{theorem}[Satz von der Existenz des Tensorprodukts]
    Für je zwei $R$-Moduln existiert ein Tensorprodukt.
\end{theorem}
\begin{proof}
    Sei $M,N$ zwei $R$-Moduln. Sei $I=M\times N$.
    Betrachte den freien $R$-Modul $F\coloneqq R^{(I)} = \bigoplus_{(x,u)\in I} R$.
    Für $(x,u)\in I$ sei $i_{(x,u)}:R\rightarrow F$ die jeweilige Inklusion.
    Schreibe $[x,u] \coloneqq i_{(x,u)} (1)\in F$.
    Die Menge $\{[x,u]\mid x\in M, u\in N\}$ ist eine Basis von $F$.
    Sei $Z$ der Untermodul von $F$, der von allen Elementen der Form $[rx+sy,u] - r[x,u] - s[y,u]$
    $[x, ru+sv] - r\cdot [x,u] - s\cdot [x,v]$ mit $r,s\in R$ ,$x,y\in M$, $u,v\in N$ erzeugt wird.
    Setze $T\coloneqq F/Z.$

    Die Abbildung $\beta:M\times N \rightarrow T, \beta(m,n) = [m,n]+Z$ ist bilinear nach Konstruktion.

    $\beta$ erfüllt die Universelle Eigenschaft:
    Betrachte \TODO[Mehr Bilder]
    Definiere $\overline{\phi}:F\rightarrow L$ durch $\overline{\phi} ([m,n]) = f(m,n)$.
    Z.z. $\overline{\phi}_{|Z} \equiv 0$. Dann induziert $\overline{\phi}$ einen Homomorphismus $\phi: T\rightarrow L$.
    $\overline{\phi}_{|Z}$ folgt sofort aus der Bilinearität von $f$.
\end{proof}
\end{document}