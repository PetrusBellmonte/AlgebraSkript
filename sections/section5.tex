\documentclass[../main.tex]{subfiles}
\graphicspath{{\subfix{../images/}}}
\begin{document}
\begin{flushright}
VL vom 26.1.2024:
\end{flushright}
\subsection{Grundlagen der Modultheorie}
Sei $R$ ein Ring mit Eins (aber nicht zwingend kommutativ)

\begin{definition}
    Ein \imph{Linksmodul} über $R$ ist eine abelsche Gruppe $(M,+)$ zusammen mit einer Skalarmultiplikation
    $$R\times M\rightarrow M,\quad (r,m)\mapsto r * m$$
    mit den Eigenschaften
    \begin{enumerate}[label=(\roman*)]
        \item $r\cdot (x+y) = r\cdot x + r\cdot y$
        \item $(r+s)* x = r * x + s * x$
        \item $(r\cdot s) * x = r * (s*x)$
        \item $1_R * x = x$
    \end{enumerate}
    für alle $r,s\in R$ und $x,y\in M$.\\
    Analog ist ein \imph{Rechtsmodul} definiert mit Skalarprodukt $M\times R\rightarrow M\quad (m,r)\mapsto m * r$
\end{definition}

\begin{remark}
    Auf der $R$ zugrundeliegenden abelschen Grupp erhalten wir eine neue Ringstruktur, indem wir die Multiplikation durch $r\odot s \coloneqq s\cdot r$ ersetzten.
    Man nennt diese den entgegengesetzten Ring $R^{Op}$. Ist $M$ bezüglich $M\times R\rightarrow M$ ein $R$-Rechtsmodul, dann ist $M$ bezüglich $R^{Op}\times M\rightarrow M, r\odot m= m\cdot r$, ein $R^{Op}$-Linksmodul.
\end{remark}
Wir betrachten in Zukunft Linksmoduln und nennen sie einfach Moduln.
\begin{example} $ $
    \begin{enumerate}[label=(\arabic*)]
        \item $R=K$ Körper. Dann sind $K$-Moduln genau die $K$-Vektorräume
        \item $M$ abelsche Gruppe
        $$End(M) = \{f:M\rightarrow M \text{ Homomorphismus}\}$$
        ist ein Ring bezüglich $(f+g)(x) = f(x) + g(x)$ und $(f\cdot g)(x) = f(g(x))$ mit der Skalarmultiplikation $End(M)\times M \rightarrow M\quad (f,m)\mapsto f(m)$ wird $M$ ein $End(M)$-Modul.
        \item Abesche Gruppen sind genau die $\Z$-Moduln:
        Ist $M$ eine abelsche Gruppe, dann ist $M$ ein $\Z$-Modul vermöge
        $$n\cdot x = \underbrace{x+\dots+x}_{n\text{-fach}} \quad (-1)\cdot x = -x$$
    \end{enumerate}
\end{example}

\begin{definition}$ $
    \begin{enumerate}[label=(\alph*)]
        \item Eine Abbildung $f:M\rightarrow N$ zwischen $R$-Moduln ist \imph{$R$-Linear}, wenn für $x,y\in M$, $r\in R$ gilt
        $$f(x+y) = f(x) + f(y) \quad f(r\cdot x) = r\cdot f(x)$$
        $Hom_R(M,N) \coloneqq \{\text{$R$-lineare Abbildung $M\rightarrow N$}\}$
        \item Isomorphismus
        \item Untermodul
    \end{enumerate}
\end{definition}
\begin{example} $ $
    \begin{enumerate}[label=(\alph*)]
        \item Ist $f\in Hom_R(M,N)$, dann sind $\ker(f)$ und $f(M)$ Untermoduln.
        \item $R$ ist ein Links- und Rechtsmodul bzüglich der Ringmultiplikation $R\times R \rightarrow R$. Dann (Links-) Ideale $\hat{=}$ Untermodul (in Linksmodulstruktur). Für Rechts- analog.
    \end{enumerate}
\end{example}

Ilias (angeblich)

\setcounter{theorem}{8}
\begin{definition}
    Sei $(M_i)_{i\in I}$ eine Familie von $R$-Moduln. Das kartesische Produkt $\prod_{i\in I} M_i$ ist ein $R$-Modul bezüglich $(x_i)_{i\in I} + (y_i)_{i\in I} = (x_i + y_i)_{i\in I}$ und $r\cdot (x_i)_{i\in I} = (rx_i)_{i\in I}$.
    Wir sagen $\prod_{i\in I} M_i$ ist das \imph{direkte Produkt der $M_i$}. Für jedes $i\in I$ ist die Projektion $pr_i: \prod_{j\in I} M_i \rightarrow M_i$ $R$-linear.
    Die Zuordnung $\phi\mapsto (pr_i \circ \phi)_{i\in I}$ definiert einen Isomorphismus abelscher Gruppen
    $$Hom_R(M,\prod_{i\in I} M_i)\overset{\cong}{\rightarrow} \prod_{i\in I} Hom_R(M,M_i)$$
    (Ist $R$ kommutativ ist die Struktur wieder ein Modul)
\end{definition}
\begin{remark*}
    \TODO[Timos remark]
\end{remark*}

\begin{definition}[Direkte Summe von $R$-Modul]
    Sei $(M_i)_{i\in I}$ eine Familie von $R$-Moduln. Definiere $\bigoplus_{i\in I} M_i = \{(x_i)_{i\in I}\in \prod_{i\in I} M_i \mid x_i = 0 \text{ bis auf endlich viele $i$}\}$ die \imph{direkte Summe der $M_i$} alse Untermodul von $\prod_{i\in I} M_i$.
    Für jedes $j\in I$ ist $\iota_j : M_j \rightarrow \oplus_{i\in I} M_i$, $\iota_j (x) = (0,\dots, 0,\underbrace{x}_{\text{Position $j$}},0,\dots,0)$ eine injektive $R$-lineare Abbildung.
    Die Zuordnung $\phi \mapsto (\phi \circ \iota_i)_{i\in I}$ definiert einen Isomorphismus abelscher Gruppen
    $$Hom_R \left(\bigoplus_{i\in I} M_i,M\right)\overset{\cong}{\rightarrow} \prod_{i\in I} Hom_R(M_i, M)$$
    Die Umkehrabbildung ist 
    $$(f_i: M_i\rightarrow M)_{i\in I} \mapsto \begin{cases}
        \bigoplus_{i\in I} M_i\rightarrow M,\\
        (x_i)_{i\in I} \mapsto \sum_{i\in I} f_i(x_i)
    \end{cases}$$
    Gewöhnlich schreiben wir $(x_i)_{i\in I} \in \bigoplus_{i\in I} M_i$ auch als $\sum_{i\in I} x_i$.\\
    Notation: Ist $M_i = M$, dann schreiben wir $\bigoplus_{i\in I}M_i = M^{(I)}$ und $\prod_{i\in I} M_i = M^I$
\end{definition}

\begin{definition}
    Sei $S\subseteq M$ eine Teilmenge in einem $R$-Modul $M$. Dann heißt $S$
    \begin{itemize}
        \item \imph{Erzeugendensystem}, wenn $M\coloneqq \langle S\rangle_R = \{\sum_{s\in J} r_s\cdot s \mid J\subseteq S \text{ endlich}, r_s\in R\} = \bigcap\{N\mid N\subseteq M\text{ Untermodul mit }S\subseteq N\}$
        \item \imph{$R$-linear unabhängig}, wenn für alle $s_1,\dots,s_m \in S$ und $r_1,\dots,r_m\in R$ die Implikation $\sum_{i=1}^m r_i\cdot s_i = 0 \Rightarrow r_1 = \dots = r_m = 0$ gilt.
        \item \imph{Basis}, wenn $s$ ein linear unabhängiges Erzeugendensystem ist
        \item $M$ heißt
        \begin{itemize}
            \item \emph{endlich erzeugt}\index{Endlich erzeugter modul}, wenn $M$ ein endliches Erzeugendensystem besitzt
            \item \emph{frei}\index{Freier modul}, wenn $M$ eine Basis besitzt
        \end{itemize}
    \end{itemize}
\end{definition}

\begin{remark}$ $
    \begin{enumerate}[label=\alph*)]
        \item Jeder VR hat eine Basis, ist also frei.
        Aber $\Q$, $\Z/n$ alse $\Z$-Modul ist nicht frei:
        Betrachte $q_1 = \frac{a_1}{b_1},q_2=\frac{a_2}{b_2}\in \Q$. Dann gilt $(b_1a_2)\cdot q_1 - (b_2a_1)\cdot q_2 = 0$ (nicht triviale linearkombination).
        Somit ist $S\subseteq \Q$ mit $|S|\geq 2$ nicht linear unabhängig.
        Weiter kann ein einziges Element $q=\frac{a}{b}$ nicht $\Q$ erzeugen als $\Z$-Modul, denn $\langle q\rangle_\Z = \Z_q = \Z\cdot \frac{a}{b}\subseteq \Z\cdot \frac{1}{b} \subset \Q$
        \item Die Kardinalität einer Basis ist im Allgemeinen keine Invariante feier Moduln.
    \end{enumerate}
\end{remark}

\begin{lemma}
    Ein $R$-Modul $M$ ist frei genau dann, wenn er zu einem Modul $R^{(I)}$ für eine Menge $I$ isomorph ist.
\end{lemma}
\begin{proof} $ $
    \begin{itemize}
        \item[$\Rightarrow$]
        Sei $S$ eine Basis. Betrachte die Abbildung $R^{(S)}\overset{\cong}{\rightarrow} M,\; (r_s)_{s\in S} \mapsto \sum_{s\in S} r_s\cdot s$. Diese ist ein Isomorphismus.
        \item[$\Leftarrow$]
        $R^{(I)}$ ist frei bezüglich der Basis $\{e_i =\iota_i(1_R)\mid i\in I\}$
    \end{itemize}
\end{proof}

\begin{flushright}
    01.02.2023
\end{flushright}
\begin{definition}
    Ein Untermodul $N\subseteq M$ heißt \imph{direkter Summand}, wenn es einen Untermodul $C\subseteq M$ gibt so, dass
    \begin{enumerate}[label=(\roman*)]
        \item $C\cap N = \{0\}$
        \item $C+N=M$
    \end{enumerate}
    In diesem Fall gilt $M\cong C\oplus N$.
    Man schreibt dann $M=C \oplus N$.
\end{definition}
\begin{example*}
    $2\Z \subset \Z$ ist kein direkter Summand
\end{example*}

\begin{definition}[Exakte Folgen]
    Es seien $M,M',M''$ $R$-Moduln.
    Es seien $f:M'\rightarrow M, g:M\rightarrow M''$ $R$-lineare Abbildungen.
    Man sagt, dass die Folge $$M'\overset{f}{\rightarrow} M \overset{g}{\rightarrow} M''$$
    \imph{exakt} ist, wenn $Bild(f) = \ker(g)$.

    Eine Folge der Form $$0\rightarrow M'\overset{f}{\rightarrow} M \overset{g}{\rightarrow} M'' \rightarrow 0$$ nennt man \imph{kurz exakte Folge}, wenn sie exakt an jeder Stelle ist, d.h. $f$ injektiv, $g$ surjektiv und $Bild(f) = \ker(g)$.

    Homomorphiesatz liefert\footnote{Das sollte für alle $M'\overset{f}{\rightarrow} M \overset{g}{\rightarrow} M'' \rightarrow 0$ gelten. $g$ muss surjektiv sein, damit $g(M) = M''$, aber $f$ nicht injektiv, weil das nichts an $Bild(f)=\ker(g)$ ändert}, dass
    $$M/f(M') \cong M''$$.
\end{definition}
\begin{example*}
    $0\rightarrow \Z\overset{\cdot 2}{\rightarrow} \Z \rightarrow \Z/2\Z \rightarrow 0$ kurz exakt.
\end{example*}

\begin{theorem}%Satz und Definition
    Für eine kurze exakte Folge $0\rightarrow M'\overset{f}{\rightarrow} M \overset{g}{\rightarrow} M'' \rightarrow 0$ sind die folgeneden Aussagen äquivalent:
    \begin{enumerate}[label=(\roman*)]
        \item $f(M')\subseteq M$ ist ein direkter Summand
        \item $\exists s\in Hom_R(M'',M): g\circ s = id_{M''}$
        \item $\exists t\in Hom_R(M,M'): t\circ f = id_{M'}$
    \end{enumerate}
    Sind diese Bedingungen erfüllt, so sagt man, dass die Folge \imph{spaltet}.
\end{theorem}
\begin{proof}$ $
    \begin{itemize}
        \item[(i)$\Rightarrow$ (ii)]
        Angenommen $f(M')\oplus C = M$.
        Dann ist $g_{|C}$ ein Isomorphismus:
        \begin{itemize}[noitemsep]
            \item $\ker(g_{|C}) = \ker{(g)}\cap C = f(M')\cap C = 0$
            \item $Bild(g_{|C}) = g(C) = g(f(M')\oplus C) = g(M) = g(M) = M''$
        \end{itemize}
        Definiere $s(x) = g_{|C}^{-1}(x)\in M$ für $x\in M''$.
        Dann ist $g\circ s = id_{M''}$.
        \item[(ii) $\Rightarrow$ (iii)]
        Für $y\in M$ ist $y-(s\circ g) (y) \in Bild(f) = \ker(g)$, weil $$g(y-(s\circ g) (y)) = g(y)- (\underbrace{g\circ s}_{= id_{M''}}\circ g) (y) = g(y) - g(y) = 0$$.
        Definiere $t(y) = f^{-1}(y- s\circ g (y))$ wobei $f$ als Abbildung aufs Bild batrachtet wird.
        
        Dann $t\circ f = id_{M'}$, wegen
        $t(f(x)) = f^{-1}(f(x) - s\circ \underbrace{g \circ f}_{ = 0} (x)) = x$.
        \item[(iii) $\Rightarrow$ (i)]
        Setze $C\coloneqq \ker(t)$
        Dann ist $C\cap f(M') = \{0\}$: Sei $x\in C\cap f(M')$.
        Insbesonderer $x=f(y)$ für ein $y\in M'$. Dann $0 = t(x) = t(f(y)) = y$ $\Longrightarrow$ $x = f(y) = 0$.

        Weiter ist $C+f(M') = M$. Sei $m\in M$.
        Betrachet $t(m-f\circ t(m)) = t(m) - \underbrace{t\circ f}_{=id_{M'}} \circ t(m) = t(m)-t(m) = 0$, also $m-\underbrace{f\circ t(m)}_{\in Bild(f)} \in C$
    \end{itemize}
\end{proof}
\begin{definition}$ $
    \begin{enumerate}[label=\alph*)]
        \item 
        Ein $R$-Modul $P$ heißt \imph{projektiv}, wenn jede kurze exakte Folge von $R$-Moduln der Form $0\rightarrow M' \rightarrow M \rightarrow P\rightarrow 0$ spaltet.
        \item 
        Ein $R$-Modul $I$ heißt \imph{injektiv}, wenn jede kurze exakte Folge von $R$-Moduln der Form $0\rightarrow I\rightarrow M \rightarrow M'' \rightarrow 0$ spaltet.
    \end{enumerate}
\end{definition}

\begin{remark}
    Jede kurze exakte Folge von VR über einem Körper $K$ spaltet, weil jeder Unter-VR ein Komplement besitzt.
    => Alle VR sind projektiv und injektiv.
\end{remark}

\subsection{Bilineare Abbildungen und Tensorprodukte}
In diesem Abschnitt ist $R$ ein kommutativer Ring.
\begin{definition}
    Seien $L,M,N$ drei $R$-Moduln.
    Eine Abbildung $f:M\times N\rightarrow L$ heißt \imph{$R$-bilinear} wenn gelten
    \begin{enumerate}
        \item $f(rx+sy,u) = rf(x,u)+sf(y,u)$
        \item $f(x,ru+sv) = rf(x,u) + sf(x,v)$
    \end{enumerate}
    für $x,y\in M$, $u,v \in N$, $r,s\in R$.
\end{definition}

\begin{definition}
    Ein \imph{Tensorprodukt} zweier $R$-Moduln $M$ und $N$ ist ein Paar $(T,\beta)$ bestehend aus einem $R$-Modul $T$ und einer $R$-bilinearen Abbildung $\beta:M\times N\rightarrow T$ so, dass eine \textit{universelle} Eigenschaft erfüllt ist:
    
    Für jeden $R$ -Modul L und jede $R$-bilneare Abbildung $f:M\times N\rightarrow L$ existiert ein eindeutiger $R$-Homomorphismus $\phi:T\rightarrow L$ mit $\phi\circ\beta = f$
    $$% https://tikzcd.yichuanshen.de/#N4Igdg9gJgpgziAXAbVABwnAlgFyxMJZABgBpiBdUkANwEMAbAVxiRAFkAdTvAW3gAEAORABfUuky58hFAEZyVWoxZsAKmIkgM2PASJk5S+s1aIQAGTFKYUAObwioAGYAnCLyRkQOCEgBM1Caq5s4g1Ax0AEYwDAAKUnqyIK5YdgAWOJou7p6I3r5ICsqmbNwxOHTZIG4eRdSFiIElISDcMAAeWHA4cAIAhNxo6VjhIAxYYGYgUBA4OLZjMWBQSADMxKIUokA
\begin{tikzcd}
M\times N \arrow[d, "f"'] \arrow[r, "\beta"] & T \arrow[ld, "\exists !\phi", dotted, bend left] \\
L                                            &                                                 
\end{tikzcd}$$
\end{definition}
\begin{lemma}
    Sind $(T,\beta)$ und $(T',\beta')$ zwei Tensorprodukte von $M\times N$, dann existiert ein eindeutiger Isomorphismus $\phi: T\rightarrow T'$ mit
$$% https://tikzcd.yichuanshen.de/#N4Igdg9gJgpgziAXAbVABwnAlgFyxMJZABgBpiBdUkANwEMAbAVxiRABUQBfU9TXfIRQAmclVqMWbdgHJuvEBmx4CRAIyk14+s1aIQAWQA6RvAFt4AAgBy3cTCgBzeEVAAzAE4QzSMiBwQSBoSumwmaAAWWPLuXj6Iov6BiH46UvomAEYwOHQxIJ7eSIkBQdRpeiBZOXRy1Ax02QwACvwqQiAeWI4ROHZcQA
\begin{tikzcd}
T \arrow[rr, "\phi", "\cong"'] &                                                     & T' \\
                     & M\times N \arrow[lu, "\beta"] \arrow[ru, "\beta'"'] &   
\end{tikzcd}$$
\end{lemma}
\begin{proof} $ $
    \begin{itemize}
        \item $% https://tikzcd.yichuanshen.de/#N4Igdg9gJgpgziAXAbVABwnAlgFyxMJZARgBpiBdUkANwEMAbAVxiRAFkAdTvAW3gAEAORABfUuky58hFAAZScqrUYs2AFTESQGbHgJEATIuX1mrRCHUByMcphQA5vCKgAZgCcIvJApA4IJDIVczZuACMYHDotdy8fRD8ApGMQtUsIqLpbagY6SIYABSl9WRAPLEcACxxYkE9vIOpkxFSGLDALECgIHBwHEGozdJBuGAAPLDgcOAEAQgFuNCqsO1EgA
\begin{tikzcd}
T \arrow[rr, "\exists ! \phi", dotted] &                                                     & T' \\
                                       & M\times N \arrow[lu, "\beta"] \arrow[ru, "\beta'"'] &   
\end{tikzcd}$ universelle Eigenschaften von $(T,\beta)$
        \item
        $% https://tikzcd.yichuanshen.de/#N4Igdg9gJgpgziAXAbVABwnAlgFyxMJZARgBpiBdUkANwEMAbAVxiRAFkAdTvAW3gAEAORABfUuky58hFAAZScqrUYs2AFTESQGbHgJEATIuX1mrRCHUByMcphQA5vCKgAZgCcIvJApA4IJDIVczZuACMYHDotdy8fRD8ApGMQtUsIqLpbagY6SIYABSl9WRAPLEcACxxYkE9vFOpkxGCGLDALECgIHBwHEGozdJBuGAAPLDgcOAEAQgFuNGxBkDyC4r0ZNgrq2tEKUSA
\begin{tikzcd}
T &                                                     & T' \arrow[ll, "\exists ! \psi"', dotted] \\
  & M\times N \arrow[lu, "\beta"] \arrow[ru, "\beta'"'] &                                         
\end{tikzcd}$ universelle Eigenschaft von $(T',\beta')$
        \item $% https://tikzcd.yichuanshen.de/#N4Igdg9gJgpgziAXAbVABwnAlgFyxMJZARgBoAGAXVJADcBDAGwFcYkQAVAchAF9T0mXPkIpyFanSat2HPgJAZseAkQBMEmgxZtEneYOUiiZYpO0y9AWQA6NvAFt4AAgByfSTCgBzeEVAAZgBOEA5IZCA4EEjiUjrsdmgAFlgGIMGhMTRRSBpxliCJ2GkZYYgROYh5AEYwYFBIAMyxcCkBOEgAtBEWuiBYUAD6cjSM9LWMAApCKqIgjDDtJSFljdnRVVrSfXa1OPQgo+MwUzPGekFY3kkd-IErTevhW-F6uzD7HrxAA
\begin{tikzcd}
T \arrow[r, "\phi"] \arrow[rr, "id_T", bend left, shift left] & T' \arrow[r, "\psi"]                               & T \\
& M\times N \arrow[ru, "\beta"'] \arrow[lu, "\beta"] &  
\end{tikzcd}$ \begin{minipage}{0.5\textwidth}
    univeselle Eigenschaft: $(T,\beta)$ -Eindeutigkeit\\ $\Rightarrow \psi\circ \phi = id_T$
\end{minipage}
        \item $% https://tikzcd.yichuanshen.de/#N4Igdg9gJgpgziAXAbVABwnAlgFyxMJZARgBoAGAXVJADcBDAGwFcYkQAVEAX1PU1z5CKchWp0mrdhwDkPPiAzY8BIgCYxNBizaJOc3v2VCiZYuO1S9AWQA6tvAFt4AAgByPcTCgBzeEVAAMwAnCEckMhAcCCRRCR12ezRseSDQ8MQ46KQNeKsQJIALLFSQELCImmzEXIAjGDAoJABmOLhiwJwkAFpIy10QLCgAfWBZbhAaRnp6xgAFARVhEEYYTtLyjOaqmJqtSQH7epx6OSmZmHnFkz1grB9CrsMy9JadyrzD22PTz24gA
\begin{tikzcd}
T' \arrow[r, "\psi"] \arrow[rr, "id_{T'}", bend left, shift left] & T \arrow[r, "\phi"]                                  & T' \\
& M\times N \arrow[ru, "\beta'"'] \arrow[lu, "\beta'"] &   
\end{tikzcd}$ \begin{minipage}{0.5\textwidth}
    univeselle Eigenschaft: $(T',\beta')$ -Eindeutigkeit\\ $\Rightarrow \phi\circ \psi = id_{T'}$
\end{minipage}
    \end{itemize}
\end{proof}

\begin{theorem}[Satz von der Existenz des Tensorprodukts]
    Für je zwei $R$-Moduln existiert ein Tensorprodukt.
\end{theorem}
\begin{proof}
    Sei $M,N$ zwei $R$-Moduln. Sei $I=M\times N$.
    Betrachte den freien $R$-Modul $F\coloneqq R^{(I)} = \bigoplus_{(x,u)\in I} R$.
    Für $(x,u)\in I$ sei $i_{(x,u)}:R\rightarrow F$ die jeweilige Inklusion.
    Schreibe $[x,u] \coloneqq i_{(x,u)} (1)\in F$.
    Die Menge $\{[x,u]\mid x\in M, u\in N\}$ ist eine Basis von $F$.
    Sei $Z$ der Untermodul von $F$, der von allen Elementen der Form $[rx+sy,u] - r[x,u] - s[y,u]$,
    $[x, ru+sv] - r\cdot [x,u] - s\cdot [x,v]$ mit $r,s\in R$ ,$x,y\in M$, $u,v\in N$ erzeugt wird.
    Setze $T\coloneqq F/Z.$

    Die Abbildung $\beta:M\times N \rightarrow T, \beta(m,n) = [m,n]+Z$ ist bilinear nach Konstruktion.

    $\beta$ erfüllt die Universelle Eigenschaft:
    Betrachte \TODO[Mehr Bilder]
    Definiere $\overline{\phi}:F\rightarrow L$ durch $\overline{\phi} ([m,n]) = f(m,n)$.
    Z.z. $\overline{\phi}_{|Z} \equiv 0$. Dann induziert $\overline{\phi}$ einen Homomorphismus $\phi: T\rightarrow L$.
    $\overline{\phi}_{|Z}$ folgt sofort aus der Bilinearität von $f$.
\end{proof}
\begin{flushright}
    02.02.2024
\end{flushright}
\begin{remark}[Notation]
    Das Tensorprodukt der $R$-Moduln $M$ und $N$ notiert man ${M\otimes_R N}$
    ($M\otimes N$, wenn $R$ aus dem Kontext klar ist).
    Das Bild von $(m,n)\in M\times N$ unter der Strukturabbildung $\beta: M\times N \rightarrow M\otimes N$ notiert man als $m\otimes n\in M\otimes N$.
    Diese Elemente $m\otimes n$, $m\in M$, $n\in N$ nennt man auch \imph{Elementartensor}; diese erzeugen $M\otimes_R N$. (siehe Beweis von 5.22).
    Beachte: Die Bilinearität von $\beta$ übersetzt sich in folgende Gleichungen für Elementartensoren:$\begin{array}[t]{r}
(m_1 + m_2)\otimes n = m_1\ot n + m_2 \ot n \\
m\ot (n_1 + n_2)=m\ot n_1 + m\ot n_2 \\
(r\cdot m)\ot n = r\cdot (m\ot n) = m\ot (r\cdot n)
\end{array}$
\end{remark}
\begin{example}
    $\Q\ot_\Z \Z/n\Z = 0$
    \begin{align*}
        x\ot y &= x\cdot n^{-1}\cdot n \ot y\\
        &= n\cdot (xn^{-1}\ot y)\\
        &= xn^{-1}\ot n\cdot y\\
        &= xn^{-1}\cdot 0\\
        &= 0\in M\ot N
    \end{align*}
\end{example}

\begin{lemma}
    Für alle $R$-Moduln $L,M,N$ gelten:
    \begin{enumerate}[label=(\roman*)]
        \item $R\ot_R N \cong N$
        \item $M\ot_R N \cong N\ot_R M$
        \item $M\ot_R (N\ot_R L) \cong (M\ot_R N)\ot_R L$
        \item $(\bigoplus_{i\in I} M_i)\ot_R N \cong \bigoplus_{i\in I} (M_i\ot_R N)$
    \end{enumerate}
\end{lemma}

\begin{proof}$ $
    \begin{enumerate}[label=(\roman*)]
        \item
        $R\times N \rightarrow N, (r,n)\mapsto r\cdot n$ $R$-bilinear.
        Somit induziert sie einen $R$-Homomorphismus $R\ot_R N\rightarrow N, r\ot n\mapsto r\cdot n$.
        Die Umkehrabbildung ist $N\rightarrow R\ot_R N, n\mapsto 1\cdot n$ $R$-linear.
        \item 
        Die Abbildung $M\times N\rightarrow N\ot_R M, (m,n)\mapsto n\ot m$ ist bilinear. Somit erhält man die induziert Abbildung $M\ot_R N \rightarrow N\ot_R M,\ m\ot n\mapsto n\ot m$.
        Ähnlich erhält man die Umkehrabbildung $N\ot_R M \rightarrow M\ot_R N,\ n\ot m \mapsto m\ot n$.
        \item 
        Sei $z\in L$. Die Abbildung $f_z:M\times N\rightarrow M\ot_R(N\ot_R L),\ (m,n) \mapsto m\ot (n\ot z)$ ist bilinear.
        Daraus lässt sich $f'_z:M\ot_R N \rightarrow M \ot_R(N\ot_R L),\ m\ot n\mapsto m\ot(n\ot z)$ konstruieren.
        Die Abbildung $g:(M\ot_R N)\times L \rightarrow M \ot_R ( N\ot_R L),\ (m\ot n,z) \mapsto f'_z (m\ot n)$ ist bilinear.
        Daraus konstruiere $(M\ot_R N)\ot_R L \rightarrow M \ot_R (N\ot_R L),\ (m\ot n)\ot z \mapsto m\ot (n\ot z)$.
        Die Wohldefiniertheit der offensichtlichen Umkehrabbildung beweist man ähnlich.
        \item 
        Die $R$-Homomorphismen $j_i:M_i\ot_R N \rightarrow\left(\bigoplus_{i\in I} M_i\right) \ot_R N,\ m\ot n\mapsto incl_i(m)\ot n$ sind induziert durch die bilineare Abbildung $M_i\times N \rightarrow (\bigoplus_{i\in I} M_i) \ot_R N,\ (m,n) \mapsto incl_i(m) \ot n$.
        Die Familie $(j_i)_i$ induziert einen $R$-Homomorphismus $\bigoplus_{i\in I} M_i\ot_R N \rightarrow (\bigoplus_{i\in I} M_i) \ot_R N$.
        Die dazugehörige Umkehrhabbildung wird induziert durch die bilineare Abbildung $(\bigoplus_{i\in I} M_i) \times N \rightarrow \bigoplus_{i\in I} M_i \ot_R N,\ (\sum_i m_i,n)\mapsto \sum_i m_i\ot n$ 
        
    \end{enumerate}
\end{proof}
\begin{example*}
    $R^2 \ot R^3 \overset{(iv)}{\cong} R\ot_R R^3 \oplus R\ot R^3 \overset{(iv)+(ii)}{\cong} R\ot_R R \oplus \dots \oplus R\ot_R R \overset{(i)}{\cong}R\oplus\dots \oplus R = R^6$
\end{example*}
\begin{remark}[Tensorprodukt als Funktor]
Seien $f:M\rightarrow M'$, $g:N\rightarrow N'$ zwei $R$-lineare Abbildungen. Dann gibt es genau eine $R$-lineare Abbildung 
\begin{align*}
    M\ot_R N &\overset{f\ot g}{\longrightarrow} M'\ot_R N',\\
    m\ot n &\longmapsto f(m)\ot g(n)
\end{align*}
Kompatibel mit Komposition:
$(f\circ \Tilde{f}) \ot (g\circ \Tilde{g}) = (f\ot g) \circ (\Tilde{f}\ot \Tilde{g})$
\end{remark}

\begin{theorem}[Tensorprodukt ist rechtsexakt]
    Es sei $M$ ein $R$-Modul and $$N'\overset{f}{\longrightarrow} N \overset{g}{\longrightarrow} N'' \rightarrow 0$$ eine exakte Folge von $R$-Moduln.
    Dann ist auch die Folge $$M\ot_R N'\overset{id_M\ot f}{\longrightarrow} M\ot_R N \overset{id_M\ot g}{\longrightarrow} M\ot_R N'' \rightarrow 0$$ exakt.
\end{theorem}
\begin{proof} $ $
    \begin{enumerate}
        \item $id_M\ot g$ ist surjektiv:
        Jeder Elementartensor $x\ot y\in M\ot_R N''$ liegt in $Bild(id_M\ot g)$, denn $x\ot y = x\ot g(\Tilde{y}) = (id_M\ot g) (x\ot \Tilde{y})$ für ein $g$-Urbild $\Tilde{y}$ von $y$.
        Da die Elementartensoren $M\ot_R N''$ erzeugt, ist $Bild(id_M \ot g) = M\ot_R N''$.
        \item $Bild(id_m\ot f) =\ker(id_M \ot g)$:
        \begin{itemize}
            \item[$\subseteq$:] 
            folgt aus $(id_M\ot g)\circ (id_M\ot f) (\underbrace{x\ot y}_{\in M\ot_R N'}) = x\ot g(f(y)) = x\ot 0 = 0$\item[$\supseteq$:]
            Setze $B\coloneqq Bild(id_M\ot f)$.
        Definiere eine bilineare Abbildung $\beta: M\times N'' \rightarrow (M\ot N)/B$ wie folgt:
        Für $x\in M, u'' \in N''$ wähle $u\in N$ mit $g(u) = u''$ und definiere $\beta(x,u'')\coloneqq x\ot u + B$.\\
        Wohldefiniertheit? Sei $g(v) = u''$. Dann ist $u-v \in \ker(g) = Bild(f)$. Sei $u'\in N'$ mit $f(u') = u-v$. Nun folgt $$x\ot v +B = x\ot v + \underbrace{x\ot f(u')}_{\in B} +B = x\ot v + x\ot (u-v) +B = x\ot v + x\ot u - x\ot v + B = x\ot u +B$$
        Das induziret Abbildung\footnote{Ist diese Abbildung nicht $\beta$} $s:M\ot_R N'' \rightarrow (M\ot_R N)/B$ mit der Eigenschaft $m\ot g(u)\mapsto m\ot u +B$.
        % Abschluss Wohldefiniertheit?
        Für $w\in M\ot N$ gilt somit $s\circ(id_M\ot g)(w) = w+B$.
        Somit gilt für $w\in \ker(id_M\ot g)\subseteq M\ot_R N$, dass $w+B=0\in (M\ot_R N)/B$.\\
        $\Longrightarrow$ $w\in B = Bild(id_M\ot f)$.
        \end{itemize}
    \end{enumerate}
\end{proof}
\begin{remark}
    Im Allgemeinen ist der Funktor $M\ot_R -$ nicht exakt, d.h. $M\ot -$ erhält im Allgemeinen nicht kurze exakte Folgen.
    Bsp $0\rightarrow\Z\overset{\cdot 5}{\rightarrow} \Z\rightarrow\Z/5\Z\rightarrow 0$
    $\Z/5/Z\ot_\Z -$ liefert:
    \TODO[bild]
\end{remark}

Ein $R$-Modul $M$ ist \imph{flach}, wenn $M\ot_R -$ exakt ist.
Bsp: 
\begin{itemize}[noitemsep]
    \item $\Q$ ist ein flacher $Z$-Modul.
    \item Jeder freie $R$-Modul flach
    \item Jeder projektive $R$-Modul ist flach.
\end{itemize}

\begin{definition}
    Eine \imph{$R$-Algebra} ist ein Ring $A$ (mit 1, nicht unbedingt kommutativ) mit einer $R$-Modulstruktur $R\times A \rightarrow A$ so, dass die Ringmultiplikation $A\times A\rightarrow A$ $R$-bilinear ist.
    [d.h. $\forall a,b\in A,\ r\in R: r(a._Ab) = (ra)._A b = a._A(rb)$]
\end{definition}
\begin{flushright}
    08.02.2024
\end{flushright}
\begin{minipage}{0.3\textwidth}
    $(ra)\cdot b = a\cdot (rb) = r(a\cdot b)$
\end{minipage}
\begin{example}$ $
    \begin{enumerate}[label=(\roman*)]
        \item $M_n (R)$ ist eine $R$-Algebra
        \item $L|K$ Körpererweiterung. Dann ist $L$ eine $K$-Algebra.
        \item $R$ nullteierfrei. Dann ist $Quot(R)$ eine $R$-Algebra.
        \item Jeder Ring ist eine $\Z$-Algebra
        \item $R[X_1,\dots,X_n]$ ist $R$-Algebra
    \end{enumerate}
\end{example}
\TODO[Ab hier]
\begin{remark}
    Sei $A$ eine $R$-Algebra. Definiere $i:R\rightarrow A, r\mapsto r1_A$.
    $i$ ist ein Ringhomomorphismus.
    Weiter $i(R)\subseteq Z(A)\coloneqq \{a\in A \mid \forall b \in A: ab = ba\}$ ("Zentrum").
    Denn: $b\cdot i(r) = b \cdot (r1_A) = rb$\\
    $i(r)\cdot b = (r1_A)\cdot b = r(1_A\cdot b) = rb$

    Umgekehrt:
    Ist $A$ ein Ring und $i:R\rightarrow Z(A) \subseteq A$ ein Ringhomomorphismus, dann wird $A$ eine $R$-Algebra bezüglich der Skalamultiplikation: $ra\coloneqq i(r)\cdot a$.
\end{remark}
\begin{theorem}[Skalarerweiterung]
    Es seien $A$ eine $R$-Algebra und $M$ ein $R$-Modul.
    Es gibt genau eine Skalarmultiplikation von $A$ auf $A\ot_R M$ mit $a(b\ot x) = ab \ot x$ für $a,b\in A$, $x\in M$, die $A\ot_R M$ zu einem $A$-Modul macht.
    Man sagt: $A\ot_R M$ entsteht durch \imph{Skalarerweiterung}.
    Ist $f:M\rightarrow N$ $R$-linear, dann ist $id_A \ot f: A\ot_R M \rightarrow A\ot_R N$ $A$-linear.
\end{theorem}
\begin{proof}
    Beweisskizze:
    Sei $l_a:A\rightarrow A$ die Linksmultiplikation mit $a\in A$. Dann ist $l_a$ $R$-linear:
    $rl_a(b) = r(a\cdot b) = a \cdot (rb) = l_a(rb).$
    -> induziert die Skalarmultiplikation $l_a\ot id_M:A\ot_R M\rightarrow A\ot_R M, b\ot x\mapsto ab\ot x$.
    Der Rest ist Nachrechnen
\end{proof}
\begin{example}
    Ist $f:V\rightarrow W$ eine $\R$ -lineare Abbildung
    zwischen endlichdimensionalen $\R$-VR, dann ist $id_\C \ot f: \C\ot_\R V \rightarrow \C\ot_R W$ eine $\C$-lineare Abbidlung mit der gleichen Abbildungsmatix\footnote{Wenn $w$ und $V$ die Basen $\{v_1\dots,\}$ und $\{w_1,\dots\}$, dann wäre $\{1\ot v_1,1\ot v_2,\dots\}$ und $\{1\ot w_1,\dots\}$ Basen zu $\C\ot_R V$ und $\C\ot_R W$.}.
\end{example}

\begin{theorem}
    Es seien $M$ und $N$ freie $R$-Moduln mit Basen $\{x_1,\dots,x_m\}$ für $M$ und $\{y_1,\dots,y_n\}$ für $N$. Sei weiter $A$ eine $R$-Algebra.
    \begin{enumerate}[label=(\roman*)]
        \item $M\ot_R N$ ist ein freier $R$-Modul mit Basis $\{x_i\ot y_j\mid i \in \{1,\dots,m\}, j\in \{1,\dots,n\}\}$
        \item $A\ot_R M$ ist ein freier $A$-Modul mit Basis $\{1\ot x_1,\dots,1\ot x_m\}$.
    \end{enumerate}
\end{theorem}
\begin{proof}
i)
    Basen induzieren Isomorphismen $M\cong R^m$ und $N\cong R^n$.
    => $M\ot_R N\cong R^m \ot_R R^n \cong \underbrace{R^m \ot_R R}_{\cong R^m} \oplus \dots \oplus R^m \ot_R R \cong \underbrace{R^m\oplus\dots\oplus R^m}_{n\text{ mal}} \cong R^{n\cdot m}$
    Inspektion des Isomorphismuses zeigt, dass $\{x_i\ot y_i\}$ eine $R$-Basis ist.

    ii)
    $A \ot_R M \cong A \ot_R R^m \cong A\ot_R\oplus \dots \oplus A \ot_R R \cong A\oplus \dots \oplus A \cong = A^m$
    Standardbasis in $A^m$ entspricht $\{1\ot x_i\mid i\in \{1,\dots, n\}\}$ under dem obigen Isomorphismus.
\end{proof}

\begin{corollary}
    Sei $R$ kommutativer Ring, aber nicht der Nullring. Ist $M$ ein freier $R$-Modul mit zwei Basen $\{x_1,\dots,x_m\}$ und $\{y_1,\dots,y_n\}$, dann ist $n=m$.
\end{corollary}
\begin{proof}
    Lemma \ref{theo:2.27}\TODO[Ist 2.27 hier richtig??] liefert uns ein maximales Ideal $\mathcal{m}\subset R$.
    Dann ist $R/\mathcal{m} =: K$ ein Körper. $K$ ist auch eine $R$-Algebra.
    Durch Skalarerweiterung erhält man den $K$-VR $K\ot_R M$ mit den $K$-Basen $\{1\ot x_1, 1\ot x_2,\dots\}$ und $\{1\ot y_1, 1\ot y_2,\dots\}$ (Satz 5.34),
    Lineare Algebra => $n=m$.
\end{proof}

\begin{theorem}
    Seien $A,B$ $R$-Algebren.
    Auf $A\ot_R B$ existiert genau eine Struktur als $R$-Algebra so, dass $(a_1\ot b_1)\cdot (a_2 \ot b_2) = a_1a_2 \ot b_1b_2$ für alle $a_1,a_2\in A$ und $b_1,b_2\in B$.
\end{theorem}
\begin{proof}[Beweisskizze]
    Sei $a\in A$ und $b\in B$.
    Die Abbildungen $r_a:A\rightarrow A, x\mapsto xa$ und $r_b:B\rightarrow B, x\mapsto xb$ sind $R$-linear.
    Wir erhalten somit eine $R$-lineare Abbildung $r_{(a,b)} \coloneqq r_a\ot r_b: A\ot_R B \rightarrow A\ot_R B, x\ot y\mapsto xa\ot yb$.
    Sei $z\in A\ot_R B$. Die Abbildung $f_z: A\times B \rightarrow A\ot_R B, f_z((a,b)) = r_{(a,b)}(z)$ ist $R$-linear.
    -> $f_z:A\ot_R B\rightarrow A\ot_R B$ 
    -> Wohldefinierte Abbildung $A\ot_R B \times A\ot_R B \rightarrow A\ot_R B, (z,w) \mapsto f_z(w)$, die auf Elementartensoren wie folgt aussieht:
    $(x\ot y, a\ot b)\mapsto xa\ot yb$.
    Es verbleibt die Ringaxiome nachzuweisen!
\end{proof}

\begin{remark}
    In der obigen Situation sind $A\rightarrow A\ot_R B,\; a\mapsto a\ot 1$ und
    $A\rightarrow A\ot_R B,\;b\mapsto 1\ot b$ % TODO align
    Homomorphismus von $R$-Algebren.
\end{remark}
\begin{theorem}
    Sei $A,B$ zwei $R$-Algebren, so hat $A\ot_R B$ folgende univeselle Eigenschaft:
    Für jede weitere $R$-Algebra $C$ und Homomorphismen $f:A\rightarrow C$, $g:B\rightarrow C$ mit der Eigenschaft $\forall a\in A, b\in B: f(a)\cdot g(b) = g(b) \cdot f(a)$
    existiert ein eindeutiger Homomorphismus von $R$-Algebren $\phi:A\ot_R B\rightarrow C$ mit $\phi(a\ot b) = f(a)\cdot g(b)$
    $$% https://tikzcd.yichuanshen.de/#N4Igdg9gJgpgziAXAbVABwnAlgFyxMJZABgBpiBdUkANwEMAbAVxiRAEEQBfU9TXfIRRkAjFVqMWbdgB0ZEHAH0ASgAIAQt14gM2PASJkATOPrNWiEJp589goiNJjqZqZYDC3cTCgBzeESgAGYAThAAtkhkIDgQSADMLpIWIEFawWGRiNGxSCI2qZlIRtS5iPnaoRHFpXGIiRLmbL4g1Ax0AEYwDAAK-PpCICFYvgAWOOmF1eW1CUlNlnIwAB5YcDhwqgCEcmijWK0gDFhgKVB0cKM+XlxAA
\begin{tikzcd}
A \arrow[rd, "f"] \arrow[d]                 &   \\
A\ot_R B \arrow[r, "\exists !\phi", dashed] & C \\
B \arrow[u] \arrow[ru, "g"']                &  
\end{tikzcd}$$ mit $f(a)\cdot g(b) = g(b)\cdot f(a)$

$\phi(a\ot b) = \phi(a\ot 1 \cdot 1\ot b) = \phi(a\ot 1)\cdot \phi(1\ot b) = f(a)\cdot g(b)$
\end{theorem}
\begin{proof}
    Die Abbildung $A\times B\rightarrow C, (a,b) \mapsto f(a)\cdot g(b)$ ist $R$-bilinear.
    Somit erhält man $\phi: A\ot_R B \rightarrow C$ mit $\phi(a\ot b) = f(a)\cdot g(b)$ $R$-linear.

    Beh: $\phi$ ist Ringhomomorphismus:
    $\phi(a_1\ot b_1 \cdot a_2 \ot b_2) =\phi(a_1a_2\ot b_1b_2) = f(a_1a_2) \cdot g(b_1b_2)$\\
    $\phi(a_1\ot b_1) \cdot \phi(a_2\ot b_2) = f(a_1)\cdot g(b_1)\cdot f(a_2)\cdot g(b_2) =f(a_1)f(a_2)g(b_1)g(b_2) = f(a_1a_2)\cdot g(b_1b_2)$
    $f,g$ Ringhomomorphismen.
\end{proof}
\begin{example}
    Sei $A$ eine $R$-Algebra mit Skalarabbildung $i:R\rightarrow Z(A)\subseteq A, r\mapsto r1_A$.
    Beh: $A\ot_R M_n(R)\cong M_n(A)$ als $R$-Algebren.

    Betrachte $f:A\rightarrow M_n(A), a\mapsto (\text{Dialgonalmatrix mit a auf der diag})$, $g:M_n(R)\rightarrow M_n(A), (r_{ij}\mapsto (i(r_{ij}))\in M_n(Z(A)) \subseteq M(A)$.
    Dann gilt $f(a)g((r_{ij})) = g((r_{ij})) f(a)$.
    Nach der universellen Eigenschaft existiert $A\ot_R M_n(R) \rightarrow M_n(A), a\ot (r_{ij}) \mapsto (diag-matix mit a in diag) \cdot (r_1)$ Homomorphismus.
    Warum $\phi$ Iso? $A\ot_R \underbrace{M_n(R)}_{=R^{n^2}}\cong A^{n^2}$ und $\phi$ überführt die "Standardbasen".
    $M_n(A)\cong A^{n^2}$
\end{example}

\begin{flushright}
    09.02.2024
\end{flushright}
\subsection{Neothersche Moduln und Ring}
[$R$ möglicherweise nicht kommutativer Ring]
\begin{example}\label{theo:5.40}
    $R=\C[X_1, X_2,\dots]$ Polynomring in abzählbar unendlich vielen Variablen.
    $M=R$ $R$-Modul ist endlich erzeugt: $M=\langle 1_R\rangle$.
    Sei $I=\langle X_1, X_2,\dots \rangle\subseteq M$ der von allen Varialben erzeugte Untermodul.
    Dann ist $I$ nicht endlich erzeugt.
    
    Seien $f_1, \dots, f_m\in I$. Sei $X_n$ eine Variable, die nicht in den Polynomen $f_1,\dots, f_m$  vorkommt.
    Dann $X_n\notin \langle f_1,\dots,f_m\rangle = \{\sum_{i=1}^m g_i f_i \mid g_1,\dots, g_m\in R\}$.
    Somit sind Untermoduln von endlich erzeugten $R$-Moduln i.A. nicht endlich erzeugt.
\end{example}

\begin{definition}
    Ein $R$-Modul $M$ heißt \imph{noethersch}, wenn jeder Untermodul von $M$ endlich erzeugt ist.
\end{definition}

\begin{theorem}\label{theo:5.42}
    Sei $M$ ein $R$-Modul. Dann sind folgende Aussagen äquivalent:
    \begin{enumerate}[label=(\roman*)]
        \item $M$ ist noethersch
        \item Ist $M_1\subseteq M_2\subseteq \dots$ eine aufsteigende Kette von Untermoduln von $M$, dann gibt es ein $n_0\in \N$ mit $M_{n_0} = M_{n_0+1} = M_{n_0+2}=\dots$.
        \item Jede nicht leere Menge von Untermoduln von $M$ besitzt ein maximales Element\footnote{Eine Menge $N$ ist maximal in $\mathcal{X}$, wenn für jedes Element $N'\in \mathcal{X}$ gilt $N'\supseteq N \Rightarrow N = N'$}.
    \end{enumerate}
\end{theorem}
\begin{proof} $ $
    \begin{itemize}
        \item[(i) $\Rightarrow$ (ii)]
        Sei $M_1\subseteq M_2\subseteq \dots$ eine aufsteigende Kette und $N\coloneqq \bigcup_{i=1}^\infty M_i$.
        $M$ noethersch => $N$ endlich erzeugt, d.h. $N=\langle m_{i_1},\dots, m_{i_n}\rangle$ mit $m_{i_k} \in M_{i_k}$
        Setze $n_0 = \max\{i_1,i_n\}$.
        Dann ist $N=M_{n_0}$
        \item[(ii) $\Rightarrow$ (iii)]
        Durch Wiederspruch:\\
        Sei $\mathcal{X}$ eine nicht leere Menge von Untermoduln, die kein maximales Element hat.
        Wähle $M_1\in \mathcal{X}$. $M_1$ nicht maximal
        =>\footnote{Gäbe es keine echte Obermenge gäbe, wäre es maximal. Wenn es eine gibt, dann ist sie nach Definition von Maximal ungleich.} es gibt $M_2\in \mathcal{X}$ mit $M_1\subset M_2$.
        $M_2$ nicht maximal => es gibt $M_3\in \mathcal{X}$ mit $M_2\subset M_3$...
        Wir erhalten induktiv eine strikt aufsteigende Kette von Untermoduln $M_1\subset M_2\subset M_3\subset \dots$ \Lightning
        \item[(iii) $\Rightarrow$ (i)] Sei $N\subset M$ ein Untermodul.
        Betrachte $\mathcal{X}=\{N'\subseteq N\mid N' \text{ endlich erzeugt}\} \neq\emptyset$, wegen $\{0\}\in \mathcal{X}$.
        Sei $N_0\in \mathcal{X}$ ein maximales Element.
        Beh. $N_0 = N$. Angenommen $x\in N\setminus N_0$.
        Dann $\langle N_0\cup \{x\}\rangle\in \mathcal{X}$ im Widerspruch zur Maximalität.
    \end{itemize}
\end{proof}

\begin{lemma}\label{theo:5.43}
    Es sei $0\rightarrow M'\rightarrow M\overset{p}{\rightarrow }M''\rightarrow 0$ eine kurze exalkte Folge von $R$-Moduln.
    Dann ist $M$ genau dann noethersch, wenn $M'$ und $M''$ noethersch sind.
\end{lemma}
\begin{proof} $ $
    \begin{itemize}
        \item[$\Rightarrow$]
        Sei $M$ noethersch.
        Dann ist jeder Untermodul von $M'$ isomorph zu einen Untermodul von $M$ und somit endlich erzeugt.
        Also ist $M'$ noethersch.

        Sei $N\subseteq M''$ ein Untermodul.
        Dann ist $p^{-1}(N)\subseteq M$ endlich erzeugt.
        Somit ist $p(p^{-1}(N)) = N$ auch endlich erzeugt. Also ist $M''$ noethersch.

        \item[$\Leftarrow$]
        Sei $W\subseteq M$ ein Untermodul.
        Dann ist
        $$0\rightarrow \underbrace{j^{-1}(w)}_{\subseteq M'}\overset{j}{\rightarrow} W \overset{p}{\rightarrow} \underbrace{p(W)}_{\subseteq M''}\rightarrow 0$$
        exakt und $j^{-1}(W)$ und $p(W)$ sind endlich erzeugt.

        Die Aussage folgt aus der folgenden Implikation:
        $0\rightarrow M_1 \overset{j}{\rightarrow} M_2 \overset{p}{\rightarrow} M_3 \rightarrow 0$ kurze exakte Folge und $M_1,M_3$ endlich erzeugt => $M_2$ endlich erzeugt:
        
        Seien $M_1 = \langle x_1,\dots, x_n\rangle$ und $M_3=\langle y_1, \dots, y_m\rangle$.
        Dann ist $M_2 = \langle \Tilde{y}_1,\dots, \Tilde{y}_m, j(x_1),\dots,j(x_n)\rangle$ für $\Tilde{y}_i\in p^{-1}(\{y_i\})$:

        Sei $z\in M_2$. Dann gibt es $r_1,\dots,r_m\in R$ mit $p(z) = r_1\cdot y_1+ \dots + r_m\cdot y_m$.
        Somit ist $z-(r_1\Tilde{y}_1 + \dots + r_m \Tilde{y}_m)\in\ker (p) = Bild(j)$.
        => es gibt $s_1,\dots, s_n\in R$ mit $z-(r_1\Tilde{y}_1 + \dots +r_m\cdot \Tilde{y}_m) = s_1 j(x_1) + \dots + s_n j(x_n)$.
    \end{itemize}
\end{proof}

\begin{corollary}\label{theo:5.44} $ $
    \begin{enumerate}[label=(\roman*)]
        \item Jeder Untermodul und jeder Faktormodul eines noethersch Moduls ist noethersch.
        \item Jede endliche direkte Summe von noetherschen Moduln ist noethersch.
    \end{enumerate}
\end{corollary} $ $
\begin{proof}$ $
    \begin{enumerate}[label=(\roman*)]
        \item direkte Konsequenz von \ref{theo:5.43}
        \item Es genügt zu zeigen, dass $M\oplus N$ noethersch ist, falls $M,N$ noethersch sind. Folgt aus \ref{theo:5.43} und der kurzen exakten Sequenz $0\rightarrow M \rightarrow M\oplus N \rightarrow N \rightarrow 0$
    \end{enumerate}
\end{proof}

\begin{definition}
    Ein Ring heißt \imph{linksnoethersch}, wenn er als Linksmodul über sich selbst noethersch ist.

    Analog definiert man \imph{rechtsnoethersch}.
    Ein Ring heißt \imph{noethersch}, wenn er links- und rechtsnoethersch ist.
\end{definition}
\begin{example*}
    für noethersche Ringe:
    \begin{itemize}
        \item $R=K$ Körper
        \item Hauptidealringe (folgt aus dem Sturktursatz für Hauptidealringe)
        \item $R=M_n(K)$, $K$ Körper. Jedes (Links-) Ideal in $R$ ist ein $K$-Untervektorraum von $M_n(K) = K^{n^2}$, somit endlich erzeugt als $K$-VR. Somit erst recht endlich erzeugt als $R$-Ideal
    \end{itemize}
    Bsp \ref{theo:5.40}: $\C[X_1,X_2,\dots]$ nicht noethersch.
\end{example*}
\begin{lemma}\label{theo:5.46}
    Jeder endlich erzeugte Modul über einem linksnoetherschen Ring ist noethersch
\end{lemma}
\begin{proof}
    \ref{theo:5.44} (ii) => $R^n$ ist noethersch für $n\in \N$.
    Sei $M=\langle x_1, \dots, x_n\rangle$ ein endlich erzeugter Modul. Dann ist $R^n \twoheadrightarrow M, e_i\mapsto x_i$ ein Epimorphismus.
    \ref{theo:5.44} (i) => $M$ noethersch.
\end{proof}
\begin{lemma}\label{theo:5.47}
    Sei $M$ ein noetherscher $R$-Modul. Falls $M\cong M\oplus N$ für einen $R$-Modul $N$, so gilt $N=0$.
\end{lemma}
\begin{theorem}
    Sei $R$ ein linksnoetherscher Ring, aber nicht der Nullring.
    Sei $M$ ein freier $R$-Modul mit Basen $\{x_1,\dots,x_n\}$ und $\{y_1,\dots,y_m\}$. Dann gilt $n=m$.
\end{theorem}

\begin{proof}
    Ang. $m\leq n$. Dann gilt $R^m\cong M\cong R^n = R^m \oplus R^{n-m}$. Weiter ist $R^m$ noethersch, weil $R$ linksnoethersch ist.
    Lemma \ref{theo:5.47} => $R^{n-m} = 0$  = R nicht Nullring => n-m = 0 
\end{proof}

\begin{flushright}
    15.02.2024
\end{flushright}

\begin{proof}[Beweis zu 5.47]
    Sei $X=\{\text{durch $M$ komplementiebare Untermoduln von $M$}\} = \{N_0\subseteq M\mid\text{$N_0$ Untermodul, es existiert ein Untermodul $C$ s.d. $N_0\oplus C = M$ und $C\cong M$} \}$.
     Dann ist $\{0\}\in X$, also $X\neq \emptyset$. Sei $L\in X$ ein maximales Element, dass nach \ref{theo:5.42} existiert.
     Dann ist $M=L\oplus C$ mit $C\cong M$.
     Sei $f:M\oplus N\rightarrow M$ ein Isomorphismus.
     Dann betrachte 
     $$% https://tikzcd.yichuanshen.de/#N4Igdg9gJgpgziAXAbVABwnAlgFyxMJZABgBpiBdUkANwEMAbAVxiRAFkAdTiNZuAAQA5EAF9S6TLnyEUARnJVajFmy6cAxgQDmAgMJiJIDNjwEiAJkXV6zVog5ilMKNvhFQAMwBOEALZIZCA4EEgKynZs3Fpg2oZevgGI4SFIVsF0WAxsABYQEADW8SA+-oHUqYjpAEYwYFBIALQAzEG2qg5x1Ax0tQwAClJmsiAMMJ44TqJAA
    \begin{tikzcd}
    M\oplus N \arrow[r, "\cong", "f"'] \arrow[rr, "g", bend right] & M\cong C \arrow[r, hook] & M
    \end{tikzcd}$$
    Setze $L_1\coloneqq L + g(N) = L\oplus g(N)\subseteq M$.
    Dann gilt $M=L\oplus C = L\oplus g(N)\oplus g(M) = L\oplus \overbrace{g(M)}^{\cong M}$ somit $L_1\in X$.
    Wegen Maximaxtät und $L_1\supseteq L$ ist $g(N) = 0$ und somit $N=0$.
\end{proof}

\begin{theorem}[Hilbertscher Basissatz]
    Ist $R$ ein linksnoetherscher Ring, dann ist der Polynomring $R[X]$ auch linksnoethersch.
\end{theorem}
\begin{proof}
    Sei $I\subseteq R[X]$ ein Linksideal.
    Für $n\in \N$ sei $I_n = \{f\in I\mid \deg(f)\leq n\}$.
    Dann ist $I_{\leq n}$ ein $R$-Untermodul von $I$.
    Für $f=\sum a_i x^i$ sei $koef_n(f)\colon a_n$.
    Dann ist $koef_n: I_{\leq n} \rightarrow R$ $R$-linear.
    Sei $J_{\leq n} \coloneqq Koef_n(I_{\leq n}) \subseteq R$.
    Dann ist $J_{\leq n}$ ein Linksideal von $R$.
    Falls $f\in I_{\leq n}$ dann ist $x.f\in I_{\leq n+1}$ und $Koef_{n+1}(x\cdot f) = Koef_n(f)$.
    Somit $J_{\leq n}\subseteq J_{\leq n+1}$.
    da $R$ linksnoethersch ist, existiert ein $k\in \N$ mit $J_{\leq m} = J_{\leq k}$ für alle $m\geq k$.

    Beh: $I = \langle I_{\leq k}\rangle_{R[X]} = R[X]\cdot I_{\leq K}$.
    Klar ist $\supseteq$.
    Umgekehrt sei $f\in I$ mit $\deg(f)=m$.
    mit Induktion über den Grad $m$ zeigen wir, dass $f\in \langle I_{\leq k}\rangle_{R[X]}$.

    I-Anfang: $m\leq k$ \checkmark
    I-Schritt: $Koef_m(f)\in J_{\leq m} = J_{\leq k}$.
    Also gibt es $g\in I_{\leq k}$ mit $koef_k(g)=koef_m(f)$.
    Betrachte $\Tilde{f} = f- X^{m-k}\cdot g$.
    Dann ist $\deg \Tilde{f}  < m$.
    I-Vorrausetzung => $\Tilde{f}\in \langle I_{\leq k} \rangle_{R[X]}$.
    => $f = \Tilde{f} + X^{m-k} \cdot \underbrace{g}_{I_{\leq k}} \in \langle I_{\leq k}\rangle_{R[X]}$.
    Nun ist $I_{\leq k}$ ein $R$-Untermodul von $R x^k \oplus Rx^{k-1}\oplus \dots \oplus Rx \oplus R\cong R^{k+1}$.
    Da $R$ linksnoethersch ist, folgt mit \ref{theo:5.46}, dass $I_{\leq k}$ als $R$-Modul endlich erzeugt ist.
    Sei $g_1,\dots, g_l\in I_{\leq k}$ ein $R$-Erzeugendensystem.
    Dann gilt: $I=\langle I_{\leq k}\rangle_{R[X]}$ auch $I = \langle\{g_1,\dots,g_l\}\rangle_{R[X]}$
    % Zeige dass I von einem I_{\leq n} erzeugt wird. Betrachte dafür die Menge der höchsten koeffizienten, die noethersche Untermoduln sind und damit iwann enden. Zeige dann dass auch an dei I_{\leq n} front ab diesem Punkt nichts relevanets mehr hinzukommen kann. Dann ist verbleibt noch zu zeigen, dass endlich erzeugt
\end{proof}

\begin{corollary}
    Sei $K$ ein Körper\footnote{linksnoetherscher Ring reicht aus}. Der Polynomring $K[X_1,\dots,x_n]$ ist noethersch.
\end{corollary}
\begin{proof}
    Man hat $K[X_1,\dots,X_n] = K[X_1,\dots,X_{n-1}][X_n]$ und wendet den Hilbertschen Basissatz  wiederholht an.
\end{proof}
\begin{corollary} % F mehr wie ein I machen
    Sei $K$ ein Körper und sei $F\subseteq K[X_1, \dots, X_n]$ eine Menge von Polynomen. Betrachte die Verschwindungsmenge $$V(F) = \{(z_1,\dots,z_n)\in K^n\mid \forall f\in F: f(z_1,\dots,z_n) = 0\}$$
    Dann existieren endlich viele $f_1,\dots,f_r\in K[X_1,\dots,X_n]$ mit $V(F) = V(\{f_1,\dots, f_r\})$
\end{corollary}


\end{document}