\documentclass[../main.tex]{subfiles}
\graphicspath{{\subfix{../images/}}}

\begin{document}
\begin{definition}
    Eine algebraische Körpererweiterung $L|K$ heißt \emph{Galoiserweiterung} (oder galois'sch), wenn $L|K$ normal und separabel ist.
    Man nennt dann $Gal(L|K) = Aut_K(L)$ die \emph{Galoisgruppe} von $L|K$.

    Sei $F$ ein Körper und $H\leq Aut(F)$ eine Untergruppe.
    dann ist $F^H = \{x\in F\mid \forall\sigma\in H:\sigma(x) = x\}$ ein Teilkörper von F, genannt \emph{Fixkörper} von H.
\end{definition}
\subsection{Hauptsatz der Galoistheorie}
\begin{lemma}
    Sei $L|K$ Galoiserweiterung. Dann ist $L^{Gal(L|K)} = K$.
\end{lemma}
\begin{proof}$ $
    \begin{enumerate}
        \item["'$\supseteq$"'] klar
        \item["'$\subseteq$"'] (Durch Kontraposition) Sei $a\in L\setminus K$. Das Minimalpolynom $m_{a,K}$ hat Grad $\geq2$.
        \begin{itemize}
            \item $L|k$ normal => $m_{a,K}$ zerfällt in $L[X]$ in Linearfaktoren.
            \item $L|K$ separabel => $m_{a,k}$ hat keine mehrfach Nullstelle.
        \end{itemize}
    
        Also gibt es eine weitere Nullstelle $b$ von $m_{a,K}$ mit $b\neq a$.
        Nach Satz \cref{theo:2.37} existiert eine $\sigma \in Gal(L|K)$ mit $\sigma(a)=b$.
        => $a\notin L^{Gal(L|K)}$
    \end{enumerate}
\end{proof}
\begin{theorem}
    Seien $L$ ein Körper und $H\leq Aut(L)$ eine endliche Untergruppe.
    Dann ist
    \begin{enumerate}
        \item $L|L^H$ galois'sch
        \item $[L:L^H] = \card{H}$
        \item $Gal(L|L^H)\cong H$
    \end{enumerate}
\end{theorem}
\begin{proof}
    Sei $a\in L$. Betrachte die $H$-Bahn von $a$.
    $$H\cdot a = \{\sigma(a)\mid \sigma\in H\} = \{a_1,\dots, a_n\} \subseteq L$$
    Betrachte Polynom $f_a = \Pi_{i=1}^n (X-a_i)$ hat Koeffizienten in $L^H$.
    Für $\sigma\in H$ ist $\sigma_*(f) = \Pi_{i=1}^n(X-\sigma(a_i)) = \Pi_{i=1}^n(X-a_i)$.
    Weil alle Nullstellen von $f_a$ auch Nullstelle von $m_{a,L^H}$, muss schon $f_a=m_{a,L^H}$ sein.\footnote{$\sigma$ erhält Minimalpolynom-Nullstellen oder so}
    Nach Konstruktion hat $f_a$ nur einfache Nullstellen, ist also separabel. => $a$ separabel. $\overset{=>}{a\ beliebig}$ $L|L^H$separabel.
    Weiter zerfällt $f_a=m_{a,L^H}$ in Linearfaktoren -> $L|L^H$ normal and damit Galois.

    Es gilt$\card{H}\leq \card{Gal(L|L^H)} = [L:L^H]_S = [L:L^H]$.
    Angenommen $\card{H} < [L:L^H]\leq \infty$. 
    Dann findin wir ein $L_0$ mit $L^H\subseteq L_0 \subseteq L$ mit $\card{H} \leq [L_0:L^H] <\infty$.
    Der Satz vom primitiven Element liefert ein $a\in L_0$ mit $L_0 = L^H(a)$.
    Aber $f_a=m_{a,L^H}$ hat grad $leq \card{H}$.
    => $[L_0:L^H] = \deg(m_{a,L^K})\leq \card{H}$ \Lightning

    Also ist $[L:L^H] = \card{H} = \card{Gal(L|L^H)}$
    => $H=Gal(L|L^H)$.
\end{proof}
\end{document}