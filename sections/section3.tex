\documentclass[../main.tex]{subfiles}
\graphicspath{{\subfix{../images/}}}

\begin{document}
\begin{definition}
    Eine algebraische Körpererweiterung $L|K$ heißt \emph{Galoiserweiterung} (oder galois'sch), wenn $L|K$ normal und separabel ist.
    Man nennt dann $Gal(L|K) = Aut_K(L)$ die \emph{Galoisgruppe} von $L|K$.

    Sei $F$ ein Körper und $H\leq Aut(F)$ eine Untergruppe.
    dann ist $F^H = \{x\in F\mid \forall\sigma\in H:\sigma(x) = x\}$ ein Teilkörper von F, genannt \emph{Fixkörper} von H.
\end{definition}
\subsection{Hauptsatz der Galoistheorie}
\begin{lemma} \label{theo:3.2}
    Sei $L|K$ Galoiserweiterung. Dann ist $L^{Gal(L|K)} = K$.
\end{lemma}
\begin{proof}$ $
    \begin{enumerate}
        \item["'$\supseteq$"'] klar
        \item["'$\subseteq$"'] (Durch Kontraposition) Sei $a\in L\setminus K$. Das Minimalpolynom $m_{a,K}$ hat Grad $\geq2$.
        \begin{itemize}
            \item $L|k$ normal => $m_{a,K}$ zerfällt in $L[X]$ in Linearfaktoren.
            \item $L|K$ separabel => $m_{a,k}$ hat keine mehrfach Nullstelle.
        \end{itemize}
    
        Also gibt es eine weitere Nullstelle $b$ von $m_{a,K}$ mit $b\neq a$.
        Nach Satz \cref{theo:2.37} existiert eine $\sigma \in Gal(L|K)$ mit $\sigma(a)=b$.
        => $a\notin L^{Gal(L|K)}$
    \end{enumerate}
\end{proof}
\begin{theorem} \label{theo:3.3}
    Seien $L$ ein Körper und $H\leq Aut(L)$ eine endliche Untergruppe.
    Dann ist
    \begin{enumerate}
        \item $L|L^H$ galois'sch
        \item $[L:L^H] = \card{H}$
        \item $Gal(L|L^H) = H$
    \end{enumerate}
\end{theorem}
\begin{proof}
    Sei $a\in L$. Betrachte die $H$-Bahn von $a$.
    $$H\cdot a = \{\sigma(a)\mid \sigma\in H\} = \{a_1,\dots, a_n\} \subseteq L$$
    Betrachte Polynom $f_a = \prod_{i=1}^n (X-a_i)$.
    Für $\sigma\in H$ ist $\sigma_*(f) = \prod_{i=1}^n\left(X-\sigma(a_i)\right) = \prod_{i=1}^n(X-a_i)$.\footnote{Zu jedem $a_i$ existiert ein $\tau\in H$ mit $\tau(a) = a_i$. Dann ist $\sigma(a_i) = \sigma(\tau(a)) = \underbrace{(\sigma\circ \tau)}_{\in H}(a)\in H\cdot a$.
    $\sigma$ is bijektiv, weshalb jedes $a_i$ auf ein anderes Element in $H\cdot a\subseteq L$ abgebildet.}
    Da $f_a$ also fix unter $\sigma\in H$ ist, müssen die Koeffizienten von $f_a$ in $L^H$ liegen und damit $f_a\in L^H[X]$.
    Weil alle Nullstellen von $f_a$ auch Nullstelle von $m_{a,L^H}$ sind, muss schon $f_a=m_{a,L^H}$ sein.\footnote{$m_{a,L^H}$ teilt $f_a\in L^H[X]$, weil $a$ Nullstelle von $f_a$}
    Nach Konstruktion hat $f_a$ nur einfache Nullstellen, ist also separabel. $\Rightarrow$ $a$ separabel. $\overset{a\text{ beliebig}}{\Rightarrow}$ $L|L^H$separabel.
    Weiter zerfällt $f_a=m_{a,L^H}$ in Linearfaktoren $\Rightarrow$ $L|L^H$ normal and damit galois'sch.

    Wegen $H\subseteq Gal(L|L^H)$ gilt $\card{H}\leq \card{Gal(L|L^H)} = [L:L^H]_S \overset{\cref{theo:2.47}}{=} [L:L^H]$.
    Angenommen $\card{H} < [L:L^H]\leq \infty$. 
    Dann finden wir ein $L_0$ mit $L^H\subseteq L_0 \subseteq L$ mit $\card{H} \leq [L_0:L^H] <\infty$.
    Der Satz vom primitiven Element liefert ein $a\in L_0$ mit $L_0 = L^H(a)$.
    Aber $f_a=m_{a,L^H}$ hat Grad $\leq \card{H}$.
    => $[L_0:L^H] = \deg(m_{a,L^K})\leq \card{H}$ \Lightning

    Also ist $[L:L^H] = \card{H} = \card{Gal(L|L^H)}$
    => $H=Gal(L|L^H)$.
\end{proof}
\begin{flushright}
VL vom 30.11.2023:
\end{flushright}
\begin{theorem}[Hauptsatz]\label{theo:3.4}
    $L|K$ endliche Galoiserweiterung.
    $$U \coloneqq \{H\leq Gal(L|K) \mid H \text{ Untergruppe}\}$$
    $$Z\coloneqq \{E\subseteq L \mid K\subseteq E \subseteq L \text{ Zwischenkörper}\}$$
    \begin{enumerate}[label=\alph*)]
        \item Die Zuordnungen $Fix: U \rightarrow Z, H\mapsto L^H$ und $\Gamma: Z\rightarrow U, E \mapsto Gal(L|E)$
        sind zueinander inverse Bijektionen:
        \begin{align*}
            U &\longleftrightarrow Z\\
            H &\longmapsto L^H = Fix(H)\\
            \Gamma(E) = Gal(L|E) &\longmapsfrom  E 
        \end{align*}
        $Fix$ und $\Gamma$ sind enthaltungsumkehrend:
        \begin{alignat*}{3}
            H_1\leq H_2\;&\Rightarrow\; Fix(H_1) &&\supseteq Fix(H_2)\\
            E_1\subset E_2\;&\Rightarrow\; \Gamma(E_1) &&\geq \Gamma(E_2)
        \end{alignat*}
        \item Sei $E \in Z$, dann ist $E|K$ normal g.d.w. $Gal(L|E)$ ein Normalteiler von $Gal(L|K)$ ist.
        $$ Gal(E|K) \cong Gal(L|K)/Gal(L|E)$$
    \end{enumerate}
\end{theorem}
\begin{proof} $ $
\begin{itemize}[align= left]
    \item[Zu a)]
    Beachte: $E\in Z$ , dann $L|E$ nach \cref{theo:2.37} a) normal und nach \cref{theo:2.44} separabel und damit galoissch
    
    \begin{itemize}[align=left]
        \item[]$Fix$ und $\Gamma$ sind inverse Bijektionen:
        \item[$Fix\circ \Gamma = id$] Für jedes $E\in Z$ gilt $Fix(\Gamma(E))=Fix(Gal(L|E)) = L^{Gal(L|E)}\overset{\ref{theo:3.2}}{=} E$
        \item[$\Gamma\circ Fix = id$] Für jedes $H\in U$ gilt $\Gamma(Fix(H)) = \Gamma(L^H) = Gal(L|L^H)\overset{\ref{theo:3.3}}{=}H$
    \end{itemize}
    
    \begin{itemize}[label=•,align=left]
        \item[$Fix$ und $\Gamma$ sind enthaltungsumkehrend] 
        \item Sei $H_1\leq H_2$.
        Für jedes $x\in L^{H_2}$  gilt per Definition $\forall\sigma\in H_2\colon \sigma(x)=x$, was damit auch insbesondere für jedes $\sigma\in H_1\subseteq H_2$ gilt. $\Rightarrow x\in L^{H_1}$\\
        $\Rightarrow Fix(H_2) = L^{H_2}\subseteq L^{H_1} = Fix(H_1)$
        \item 
        Sei $E_1\subseteq E_2$.
        $\sigma \in \Gamma(E_2) = Gal(L|E_2)\leq Gal(L|E_1)=\Gamma(E_1)$ 
    \end{itemize}

    
    \begin{remark*}
        Für $\sigma\in Gal(L|K)$ und $H\in U$, dann $\sigma(L^H) = L^{\sigma H \sigma^{-1}}$, denn für $x\in L^{\sigma H \sigma^{-1}} \Leftrightarrow \forall \tau \in H\colon \sigma \tau \sigma^{-1}(x)=x \Leftrightarrow \forall \tau \in H\colon \tau \sigma^{-1}(x) = \sigma^{-1}(x) \Leftrightarrow \sigma^{-1}x\in L^H \Leftrightarrow x\in \sigma(L^H)$
    \end{remark*}

    \item[Zu b)]$ $
    \begin{itemize}
        \item["'$\Rightarrow$"']
        Ist $E|K$ normal, so gilt $\sigma(E) = E$ für alle $\sigma \in Gal(L|K)$ nach \cref{theo:2.35} b).
        $E=L^{Gal(L|E)} \overset{\sigma\Gamma(\cdot)\sigma^{-1}}{\Longrightarrow} \sigma Gal(L|E) \sigma^{-1} = \sigma \Gamma(E) \sigma^{-1} = Gal(L|E)$ für alle $\sigma \in Gal(L|K)$.
        Damit ist $Gal(L|E)=\Gamma(E)$ normal, also $Gal(L|E)\nteq Gal(L|K)$.
        \item["'$\Leftarrow$"']
        $Gal(L|E) \nteq Gal(L|K)$, d.h. $\forall \sigma\in Gal(L|K)$ gilt $\sigma(E)=\sigma(L^{Gal(L|E)}) = L^{\sigma Gal(L|E) \sigma^{-1}} = L^{Gal(L|E)} = E$.
    Nach \cref{theo:2.35} (ii) ist $E|K$ normal.
    \end{itemize}

    Sei $E|K$ normal.
    Die Restriktionsabbildung $r_E: Gal(L|K) \rightarrow Gal(E|K), \sigma \mapsto \sigma_{|E}$ ist Gruppenhomomorphismus mit $\ker(r_E) = Gal(L|E)$.
    $r_E$ ist surjektiv:
    Für $\tau\in Gal(E|K)$ findet man dank Fortsetzungssatz (2.31) ein $\sigma \in Gal(L|K)$ sodass $\sigma_{|E}=\tau$.
    Mit Homomorphiesatz folgt die Behauptung.
\end{itemize}
\end{proof}
\begin{remark}
    $L|K$ endliche Galoiserweiterung $H\leq Gal(L|K)$
    \begin{itemize}
        \item $[L:L^H] = \card{H}$
        \item $[L^H:K] = [L:K]/[L:L^H] = \frac{\card{Gal(L|K)}}{\card{H}} = [Gal(L|K):H]$
    \end{itemize}
\end{remark}
\begin{example}
    Wie bei (\cref{theo:2.36} a)
    $L$ Zerfällungskörper von $X^4-2$ über $\Q$.
    $L = \Q(a, ia, -a, -ia) = \Q(a,i)$ mit $a=\sqrt[4]{2}$.
    Damit ist $[L:\Q] = 8$.
    $L|\Q$ ist Galois Gruppe.
    $\sigma\in Gal(L|\Q)$ eindeutig bestimmt durch $\sigma(a)\in \{a,ia,-a,-ia\}$, $\sigma(i)\in \{i,-i\}$.
    Da $\card{Gal(L|\Q)} = 8$ sind alle Kombinationen möglich.
    \TODO[Graphik]
    $Gal(L|\Q) = \{id,\rho,\rho^2,\rho^3,\tau,\rho\tau,\rho^2\tau,\rho^3\tau\}$ % In 2 Zeilen!
    $\tau\rho^{-1}=\rho \tau = \rho\tau^3$ % Bei erstem fakor klammer??
    Isomorph zu der Diedergruppe:
    $D_4 = <x,y\mid x^4=1, y^2=1,xy=yx^3>$
    \TODO[Andere Graphik und mehr...]
\end{example}

\begin{flushright}
VL vom 1.12.2023:
\end{flushright}
\begin{theorem}[Produktsatz]
Sei $K$ Körper
    $L_1,L_2\subset \overline{K}$ zwei Teilkörper sodass $(L_1|K)$ und $(L_2|K)$ endlich und galois'sch.
    Dann ist das Kompositum $L_1L_2$ eine Galois-Erweiterung von $K$.
    Die Zuordnung 
    \begin{align*}
        \Phi\colon Gal(L_1L_2|K))&\rightarrow Gal(L_1|K)\times Gal(L_2|K)\\
        \sigma &\mapsto (\sigma_{|L_1},\sigma_{|L_2})
    \end{align*}
    ist ein injektiver Gruppenhomomorphismus.
    Falls $L_1\cap L_2 = K$, so ist $\Phi$ ein Isomorphismus.
\end{theorem}
\begin{proof}
    Damit $L_1L_2$ galois ist, muss es normal und separabel sein:
    \begin{itemize}
        \item[normal] $L_i$ ist Zerfällungskörper von $w_i\subset K[X]\setminus K$ ($i\in\{1,2\}$)=> $L_1L_2$ ist Zerfällungskörper von $w_1\cup w_2$ => $(L_1L_2|K)$ ist normal.
        \item[separabel] Nach Satz \ref{theo:2.50} gilt $L_1 = K(a_1)$, $L_2=K(a_2)$ mit $a_i\in L_i$.
        Sie sind separabel über $K$ => $L_1L_2 = K(a_1,a_2)$ ist separabel über K (nach \ref{theo:2.47})
    \end{itemize}

    \begin{description}
        \item[$\Phi$ ist injektiv], denn für $\sigma\in \ker(\Phi)$ gilt $\sigma_{|L_1} =id$,$\sigma_{|L_2} = id$ und damit $\sigma = id$, da $L_1L_2$ erzeugt wird von $L_1\cup L_2$.
        \item[Surjektivität von $\Phi$ im Fall $K = L_1\cap L_2$]:
        
        $(L_1L_2|L_1), (L_1L_2|L_2)$ sind Galoiserweiterungen.
        \begin{align*}
            \Phi_1: Gal(L_1L_2|L_2)&\rightarrow Gal(L_1|K)\\
            \sigma &\mapsto \sigma_{|L_1}
        \end{align*}
         ist injektiv, denn für $\sigma \in \ker(\Phi_1)$ gilt $\sigma_{|L_1} = id$.
        Außerdem ist $\sigma_{|L_2} = id$, da $\sigma$ $L_2$-Homomorphismus. $\Rightarrow$ $\sigma = id$
    
        $\Phi_1$ ist surjektiv (falls $K = L_1 \cap L_2$):
        Sei $H=Bild(\Phi_1)\leq Gal(L_1|K)$.
        \begin{align*}
            L_1^H&=\{x\in L_1\mid\forall \sigma\in Gal(L_1L_2|L_2)\colon \sigma_{|L_1}(x) = x\}\\
            &= L_1\cap (L_1L_2)^{Gal(L_1L_2|L_2)}\\
            &= L_1\cap L_2\\
            &= L_1^{Gal(L_1|L_1\cap L_2)}
        \end{align*}
         $\Rightarrow H= Gal(L_1|L_1\cap L_2)$
    
        Analog mit $Phi_2: Gal(L_1L_2|L_1)\rightarrow Gal(L_2|K), \sigma \mapsto \sigma_{|L_2}$
    
        $Gal(L_1L_2|K)\geq Gal(L_1L_2|L_i)$ => $\Phi$ ist surjektiv
    \end{description}
\end{proof}
\begin{example}
    $L_1=\Q(\sqrt[4]{2},i)$
    $L_2=\Q(\sqrt{11})$ Zerfällungskörper von $X^2-11$.
    $L_1\cdot L_2 = \Q(\sqrt[4]{2},i,\sqrt{11})$
    $L_1\cap L_2 = \Q$
    Damit gilt $Gal(L|\Q)\cong \underbrace{Gal(L_1|\Q)}_{=D_4}\times \underbrace{Gal(L_2|\Q)}_{=\Z/2\Q}$
\end{example}

\subsection{Kreisteilungskörper und Einheitswurzeln}
\begin{definition}
    Sei $K$ Körper und $n\in \N$.
    Ein Element $\xi\in K^\times$ mit $\xi^n=1$ heißt \emph{$n$-te Einheitswulzel (EW)}. Hat $\xi$ die Ordnung $n$, so nennt man $\xi$ \emph{primitive $n$-te EW}.
    $\mu_n(K)\coloneqq\{\text{n-te EW in K}\}\leq K^\times$ zyklische Untergruppe.
    $\card{\mu_n(K)}\leq n$
    $\mu_n^*(K) \coloneqq \{\text{primitive n-te EW in K}\}$
\end{definition}
\begin{example*}
    $\mu_n(\C) = \{exp(2\pi i\frac{k}{n})\mid k=0,\dots,n-1\}$
    $exp(2\pi i \frac{2}{4} = exp(2\pi i \frac{1}{2})$
    $\xi\in \mu_n(\C)$ ist primitiv $\Leftrightarrow ggT(k,n)=1$

    $theta\in \mu_n(K)\setminus\trivGO \Rightarrow (X^n-1)/(X-1) = X^{n-1}+X^{n-2}+\dots+X+1$ hat $\xi$ als Nullstelle
    \TODO[bild]
\end{example*}
Definition:
$K$ Körper, $n\in \N$
$K_n$ ist definiert als Zerfällungskörper von $X^n-1$ über $K$.
\begin{theorem}\label{theo:3.10}
    $K$ Körper, $n\in \N$, $char(K)\nmid n$
    \begin{enumerate}[label=\alph*)]
        \item $(K_n|K)$ ist Galoiserweiterung, $\card{\mu_n(K_n)}=n$, $\phi_n\coloneqq \card{\mu_n^*(K_n)}= \card{(\Z/n\Z)^\times}$ und
        $K_n = K(\xi)$ für $\xi \in \mu_n^*(K_n)$
        \item $Gal(K_n|K)$ ist isomorph zu einer Untergruppe von $(\Z/n\Z)^\times$
    \end{enumerate}
\end{theorem}
\begin{proof}
    Zu a)
    $(K_n|K)$ ist normal als Zerfällungskörper.
    Es ist $D(X^n-1) = nX^{n-1}$, d.h. $X^n-1$ und $D(X^n-1)$ haben keine gemeinsamen NS d.h. $X^n-1$ ist separabel $\overset{2.48}{\Rightarrow}$ $K_n|K$ ist separabel.
    Insbesondere hat $X^n-1$ $n$ verschiedene NS, d.h. $\card{\mu_n(K_n)} = n$.
    Gruppe der $n$-ten EW ist zyklisch (mit Argument wie in \ref{theo:2.49}), also $\card{\mu_n^*(K_n)} = \card{\Z/n\Z}$

    Zu b)
    $\sigma \in Gal(K_n|K)$, $theta \in \mu_n^*(K_n)$
    $\sigma(\xi) = \xi^m$ für m teilerfremd zu $n$.
    $\sigma$ ist durch $m$ eindeutig bestimmt.
    $\Phi: Gal(K_n|K) \rightarrow (\Z/n\Z)^\times, \sigma \mapsto m+n\Z$ ist ein injektiver Gruppenhomomorphismus.
\end{proof}

\begin{definition}
    Der Körper $\Q(\xi_n)$ für $\xi_n\in\mu_n^*(\C)$ heißt $n$-ter \emph{Kreisteilungskörper}.
    Das Polynom $\phi_n=\prod_{\xi\in\mu_n^*(\C)} (X-\xi)$ heißt $n$-tes \emph{Kreisteilungspolynom}
\end{definition}

\begin{lemma}
    $\phi_n\in\Z[X]$
\end{lemma}
\begin{proof}
    Beweis mit Induktion über $n$:
    IA: $\phi_1 = X-1$ \checkmark
    $n\geq 2$: Wir verwenden, dass für $d<n$ $\phi_d\in \Z[X]$.
    Es ist $\mu_n(\C) = \bigcup_{d|n}\mu_d^*(\C)$ (disjukt).
    $X^n-1 = \prod_{\xi\in\mu_n(\C)} (X-\xi) = \prod_{d|n}\underbrace{\prod_{\xi\in\mu_d^*(\C)}(X-\xi)}_{=\phi_d}$
    Setze $f = \prod_{d|n,d<n}\phi_d \overset{(IV)}{\in}\Z[X]$, f ist normiert.
    $X^n-1=qf+r$ mit $q,r\in \Z[X]$, $\deg(r)<deg(f)$.
    In $\C[X]$ $X^n-1=\phi_n f$, d.h. $r = (q-\phi_n) f$.
    Da $\deg(r)<\deg(f)$ gilt $\phi_n=q\in \Z[X]$
\end{proof}
\begin{remark}
    Rekursive Bestimmung der Kreisteilungspolynome mittels $X^n-1 = \prod_{d|n}\phi_d$.
    Wenn $n =p$ prim: $\phi_p \cdot \phi_1 = X^p-1$ -> $\phi_p = \sum_{k=0}^{p-1} X^k$.

    $\phi_2 = X+1$, $\phi_3 = X^2+X+1$
    $\phi_4 \cdot \phi_2 \cdot \phi_1 = X^4 -1$ -> $\phi_4 =X^2+1$.
    $\phi_6\cdot\phi_3\cdot\phi_2\cdot\phi_1$ -> $\phi_6 = X^2-X+1$.

    Für $p\in \N$ Primzahl und $\alpha\in\N$ gilt
    $$X^{p^\alpha}-1 = \prod_{d|p^\alpha} \phi_d = \phi_{p^\alpha} \underbrace{\prod_{d|p^{\alpha-1}} \phi_d}_{X^{p^{\alpha-1}}-1}=\phi_{p^\alpha} (X^{p^{\alpha-1}})$$
    $$\phi_{p^\alpha} = \phi_p(X^{p^{\alpha-1}}) = \sum_{k=0}^{p-1} (X^{p^{\alpha -1}})^k$$
\end{remark}

\begin{flushright}
VL vom 7.12.2023:
\end{flushright}

\begin{theorem}\label{theo:3.14}
    Das $n$-te Kreisteilungspolynom $\phi_n$ ist irreduzibel in $\Q[X]$, d.h.
    $\phi_n=m_{\xi,\Q}$ für $\xi\in \mu_n^*(\C)$
\end{theorem}
\begin{proof}
    Sei $\xi \in \mu_n^*(\C)$. Sei $f\coloneqq m_{\xi,\Q} \in \Q[X]$.
    Wir zeigen, dass jede primitive $n$-te EW eine Nullstelle von $f$ ist.
    Dies imliziert $\phi_n |f$ und somit $\phi_n =f$ irreduzibel.
    
    Da $\xi$ Nullstelle von $X^n-1$ ist, existiert ein $h\in\Q[X]$ mit $X^n-1=f\cdot h$.
    Weiter gilt, dass $f,h\in \Z[X]$ aus folgendem Grund:
    \begin{reminder*}[Gauß-Lemma 2.5 b]
        $$c(f)\cdot c(h) = c(f\cdot h) = c(X^n-1) = 1$$
        
    \end{reminder*}
    Weiter ist $\underbrace{c(f)}_{ =\Tilde{c}(a\cdot f) a^{-1}},c(h)\in \{\frac{1}{k}\mid k\in \N\}$.
    Also folgt $c(f) = c(h)=1$. => $f,g\in \Z[X]$

    Sei $p$ eine Primzahl, die $n$ nicht teilt.
    Dann ist $\xi^p$ auch eine primimtive $n$-te EW.
    Wir behaupten, dass $\xi^p$ eine Nullstelle von $f$ ist.

    Ist das nicht der Fall, dann ist $h(\xi^p)=0$, also ist $xi$ eine Nullstelle von $h(X^p)$.
    Somit $f|h(X^p)$. Es existiert $g\in \Q[X]$ mit $h(X^p) = f\cdot g$.
    Ähnlich wie oben ist sogar $g\in \Z[X]$.
    Betrachte die Reduktion $\mod p$:
    $$\Z[X] \rightarrow \F_p[X], \sum_i c_i X^i \mapsto \sum_i \overline{c}_i X^i$$.
    Dann ist $\overline{h}^p = \overline{h}^p(X^p)= \overline{f}\cdot \overline{g}$, weil $p$-te Potenz Homomorphismus in $char=p$ und nach Euler?? $c^p\equiv c \mod p$.
    Somit sind $f$ und $h$ nicht teilerfremd $\mod p$.
    Dann ist $X^n-1 = \overline{f}\cdot\overline{h}\in \F_p[X]$ nicht separabel im Widerspruch zu $D(X^n-1) = nX^{n-1} \neq 0$.
    Somit ist $\xi^p$ Nullstelle von $f$.

    Ist $\xi'$ eine andere primitive $n$-te EW, dann ist $\xi' = \xi^m$ mit $(m,n)=1$ und man erhält $\xi'$ durch wiederholtes Bilden von Primpotenzen von $\xi$, wobei die Primexponenten zu $n$ teilerfremd sind.
    Durch Wiederholhung des obigen Alguments bekommt man $f(\xi)=0$.
\end{proof}
\begin{corollary}
    Sei $\xi\in \mu_n^*(\C)$.
     Dann ist $[\Q(\xi):\Q] = \phi(n)$ und $Gal(\Q(\xi)|\Q)\cong (\Z/n\Z)^\times$
\end{corollary}
\begin{proof}
    Kombination von \ref{theo:3.10} und \ref{theo:3.14}
\end{proof}
\begin{remark*}[Satz von Kronecker-Weber ('Kroneckers Jugendtraum')]
      Jede endliche abelsche Erweiterung\footnote{Erweiterung mit abelscher Galoisgruppe} von $\Q$ in einem Kreisteilungskörper enthalten.
      \TODO[Optional: Ausblick über dessen Konsequenzen (inc Graphik)]
\end{remark*}
\subsection{Charaktere und Normalbasen}
\begin{definition}
    Sei $\Gamma$ eine Gruppe, $K$ ein Körper.
    Ein \emph{Charakter} (von $\Gamma$ nach $K$) ist ein Homomorphismus $\Gamma\rightarrow K^\times$.
\end{definition}
Für jede Menge $M$ ist die Menge $Abb(M,K)$ der Abbildungen von $M$ nach $K$ ein $K$-Vektorraum bezüglich punktweiser Addition und skalarer Multiplikation.

\begin{lemma}\label{theo:3.17}
    $K, \Gamma$ wie zuvor.
    Paarweise verschiedene Charakterer $\chi_1,\dots,\chi_n$ von $\Gamma$ nach $K$ sind in $Abb(\Gamma,K)$ linear unabhängig.
\end{lemma}
\begin{proof}
    Durch Induktion über $n$.
    \begin{itemize}
        \item[IA ($n=1$)] $\chi = \chi_1\neq 0$, da $\chi$ Werte in $K^\times$ annimmt
        \item[IS $n\geq2$] 
        Annahme: Je $n-1$ verschiedene Charaktere sind linear unabhängig.
        Angenommen $\sum_{i=1}^n c_i \chi_i = 0$, wobei nicht alle $c_i$ Null sind (Sei \obda insbesondere $c_2 \neq 0$).
        Sei $\mu\in \Gamma$ mit $\chi_1(\mu) \neq \chi_2(\mu)$.
        Dann gilt für alle $\gamma \in \Gamma$:
        \begin{align*}
            (1):\quad 0 &= \sum_{i=1}^n c_i\chi_i(\mu\gamma) = \sum_{i=1} ^n c_i \chi_i(\mu) \chi_i(\gamma)\\
            (2):\quad 0 &= \chi_1(\mu)\sum_{i=1}^n c_i \chi_i(\gamma) = \sum_{i=1}^n c_i \chi_1(\mu)\chi_i(\gamma)\\
            (1) - (2):\quad 0&= \sum_{i=1}^n c_i (\underbrace{\chi_i(\mu) - \chi_1(\mu)}_{= 0\text{ für }i=1}) \cdot \chi_i(\gamma)\\
            &= \sum_{i=2}^n c_i (\underbrace{\chi_i(\mu) - \chi_1(\mu)}_{\neq 0\text{ für }i=2}) \cdot \chi_i(\gamma)
        \end{align*}
        Damit sind $\chi_2,\dots,\chi_n$ linear abhängig.
        \Lightning Widerspruch zu I-Annahme.
    \end{itemize}
\end{proof}
\begin{corollary} \label{theo:3.18}
    Paarweise verschiedene Automorphismen eines Körpers $K$ sind linear unabhängig in $Abb(K,K)$
\end{corollary}
\begin{proof}
    Sei $\sigma_1,\dots,\sigma_n\in Aut(K)$.
    Dann wende \cref{theo:3.17} auf ${\sigma_1}_{|K^\times},\dots, {\sigma_n}_{|K^\times}$ an.
\end{proof}
\begin{lemma}\label{theo:3.19}
    Sei $L|K$ endliche und separabel. Sei $n= [L:K]$.
    Seien $\sigma_1,\dots,\sigma_n:L\rightarrow \overline{K}$ paarweise verschiedene $K$-Homomorphismen.
    Dann sind äquivalent:
    \begin{enumerate}[label=(\roman*)]
        \item $v_1,\dots,v_n\in L$ bilden eine $K$-Basis von $L$
        \item $\det\left((\sigma_i(v_j))_{ij}\right)\neq 0$
    \end{enumerate}
\end{lemma}
\begin{proof}$ $
    \begin{itemize}
        \item[(ii)$\Rightarrow$(i)] Durch Kontraposition:
        Sei $v_1,\dots,v_n$ also keine Basis.
        Wenn sie kein Erzeugendensystem sind, ist auch die lineare Unabhängigkeit verletzt, da $n=[L:K]$.
        $v_1,\dots,v_n$ ist also nicht linear unabhängig.
        Sei dann $\sum_{j=1}^n \lambda_i v_i = 0$, wobei nicht alle $\lambda_j\in K$ Null sind.
        Dann ist $0= \sigma(0)= \sigma(\sum_{j=1}^n \lambda_i v_i) = \sum_{j=1}^n \lambda_j \sigma_i(v_j)$ für alle $i$.
        D.h. die Spalte von $(\sigma_i(v_j))_{ij}$ sind linear abhängig.
        \item[(i)$\Rightarrow$(ii)] Durch Kontraposition: Sei $\det = 0$. Dann gibt es $\mu_1,\dots,\mu_n\in \overline{K}$ mit $\sum_{i=1}^n\mu_i \sigma_i(v_j) = 0$ für alle $j$ (Zeilen lin. abh).
        Nach \cref{theo:3.17} sind ${\sigma_1}_{|L^\times},\dots,{\sigma_n}_{|L^\times}$ linear unabhängig.
        Falls $\langle\{v_1,\dots,v_n\}\rangle = L$, dann gilt $\sum_{i=1}^n \mu_i\sigma_i = 0$ im Widerspruch zur linearen Unabhängigkeit.
        Also ist $v_1,\dots v_n$ keine Basis.
    \end{itemize}
\end{proof}

\begin{flushright}
VL vom 8.12.2023:
\end{flushright}

\begin{theorem}[Satz von der Normalbasis]
    Sei $L|K$ eine endliche Galois-Erweiterung. Sei $n = [L:K]$ und $Gal(L|K) = \{\sigma_1,\dots,\sigma_n\}$.
    Es existiert ein $a\in L$, sodass $\equalto{\sigma_1(a)}{id},\dots,\sigma_n(a)$ eine $K$-Basis von $L$ bilden.
    Eine solche Basis nennen wir \emph{Normalbasis} von $L|K$.
\end{theorem}
\begin{proof} [Beweis für unendliche Körper]
    Gemäß \ref{theo:3.19} reicht es ein Element $a\in L$ zu finden, mit $\det(\sigma_i(\sigma_j(a)))_{ij}\neq 0$.
    Nach dem Satz vom primitiven Element (\ref{theo:2.56}) gibt es ein $b\in L$ mit $K(b)=L$.
    Dann sind $b_i\coloneqq\sigma_i(b)$ die paarweise verschiedenen Nullstellen von $f\coloneqq m_{b,K}=\prod_{i=1}^n (X-\sigma_i(b))\in K[X]$.
    Es ist $b_1\coloneqq b$ und $b_i \coloneqq \sigma_i(b)$.
    Setze $$g_j = \prod_{i\neq j}\frac{X-b_i}{b_j-b_i}\in L[X]\quad\text{ womit }g_j(b_k) = \delta_{jk} = \begin{cases}
        1 & \text{, wenn} j=k\\
        0 & \text{, wenn} j\neq k
    \end{cases}$$
    Weiter ist 
    $$(1)\quad g_1+\dots+g_n=1$$, weil die linke Seite Grad $\leq n-1$ hat und beide Seiten für $b_1,\dots,b_n$ übereinstimmen.
    $$(2)\quad {\sigma_{j}}_*(g_1) = {\sigma_{j}}_*\left(\prod_{i\neq 1}\frac{X-b_i}{b_1-b_i}\right) = \prod_{i\neq 1}\frac{X-{\sigma_{j}}_*(b_i)}{b_j-{\sigma_{j}}_*(b_i)}=\prod_{i\neq j}\frac{X-b_i}{b_j-b_i} = g_j$$
    Betrachte die Matrix $$A=({\sigma_{i}}_*(g_j))_{ij} \in M_n(L[X])$$
    Beh: $\det(A) \neq 0\in L[X]$
    \begin{itemize}
        \item Für jedes $b_k$ ist $(g_i\cdot g_j)(b_k) = \delta_{ik}*\delta_{jk} = 0$ falls $i\neq j$.
        $\Rightarrow$ Für $i\neq j$ gilt $f= m_{b,K}|(g_i\cdot g_j)$. (smile)
        \item Multipliziere $(1)$ mit $g_i$. Dann ergibt sich, dass $g_i^2 \equiv g_i \mod f\cdot L[X]$. (*)
    \end{itemize}
    Sei $B=AA^T = (\beta_{ij})_{ij} \in M_n(L[X])$.
    Dann gilt 
    \begin{align*}
        \beta_{ij} &= \sum_{k=1}^n{\sigma_{i}}_*(g_k){\sigma_{j}}_* (g_k)\\
        &\overset{(2)}{=} \sum_{k=1}^n{\sigma_{i}}_*({\sigma_k}_*(g_1)){\sigma_{j}}_* ({\sigma_k}_*(g_1))\\
        &= \sum_{k=1}^n (\underbrace{\sigma_i\circ \sigma_k}_{= \sigma_{m(i,k)}})_* (g_1) \cdot (\underbrace{\sigma_j\circ\sigma_k}_{\sigma_{m(j,k)}})_* (g_1)
    \end{align*}
    mit einer passenden Abbildung $m\colon [n]\times [n] \rightarrow [n]$.
    \begin{itemize}
        \item[$i=j$]
        $\beta_{ii} = \sum_{k=1}^n \left(\sigma_{m(i,k)}(g_1)\right)^2 =\sum_{k=1}^n g_k^2\overset{(*)}{\equiv} \sum_{k=1}^n g_k \overset{(1)}{=} 1 \mod f\cdot L[X] $
        \item[$i\neq j$]
        $m(i,k) \neq m(j,k)$ für alle $k$.
        Wegen (smile) also $\beta_{ij}\equiv 0\mod f\cdot L[X]$.
    \end{itemize}

    $\Rightarrow$ $B\equiv I_n\mod f\cdot L[X]$
    $\Rightarrow$ $\det(A)^2 = \det(B)\equiv 1 \mod f\cdot L[X]$.
    Insbesondere ist $\det(A)\neq 0$.

    Weil $K$ unendlich ist existiert ein $u\in K$ so, dass $p(u)\neq 0$, wobei $p\coloneqq \det(A)\in L[X]$.
    Definiere $a\coloneqq g_1(u)$.

    Dann folgt:
    \begin{align*}
        0\neq p(u) &= \det\left(\left({\sigma_i}_*\left(g_j\right)(u)\right)_{ij}\right)\\
        &= \det\left(\left((\sigma_i\circ \sigma_j)_* g_j(u)\right)_{ij}\right)\\
        &= \det\left(\left((\sigma_i\circ \sigma_j)(a)\right)_{ij}\right)\\
        &= \det\left(\left(\sigma_i(\sigma_j(a))\right)_{ij}\right)
    \end{align*}
\end{proof}

\subsection{Auflösbarkeit von Gleichungen}
\begin{definition}
    Sei $K$ eine Körper, $a\in K$.
    Eine Nullstelle von $X^n-a$ nennt man \emph{Radikal} von $a$.
\end{definition}
Die Nullstellen von $X^n-a$ in $\overline{K}$ sind $$\{\xi\cdot \sqrt[n]{a}\mid \xi \in \mu_n(\overline{K})\}$$.
Der Zerfällungskörper ist $K(\xi,\sqrt[n]{a})$, wobei $\xi$ eine primitive $n$-te EW ist.

\begin{theorem} \label{theo:3.22}
    Sei $K$ ein Körper, $n\in \N$ mit $char(K)\nmid n$.\footnote{0 teilt keine Zahl}
    Es enthalte $K$ eine primitive $n$-te EW $\xi$
    \begin{enumerate}[label=(\alph*)]
        \item $K(\sqrt[n]{a})|K$ eine endliche Galois-Erweiterung.
        Die Galoisgruppe $Gal(K(\sqrt[n]{a})|K)$ ist zyklisch und ihre Ordnung teit n.
        \item Ist $L|K$ eine endliche Galois-Erweiterung mit $[L:K] = n$ und zyklischer Galoisgruppe,
        dann ist $L=K(\sqrt[n]{a})$ für ein $a\in K$.
    \end{enumerate}
\end{theorem}
\begin{proof}$ $
    \begin{itemize}
        \item[a)] $K(\sqrt[n]{a})|K$ ist galois'sch, weil $K(\sqrt[n]{a})$ Zerfällungskörper des separablen Polynoms $X^n-a$ ist.
        Sei $\sigma\in Gal(K(\sqrt[n]{a})|K)$.
        Dann ist $\sigma(\sqrt[n]{a}) =w_\sigma \cdot \sqrt[n]{a}$, wobei $\omega_\sigma = \xi^{m_\sigma}$ eine $n$-te EW ist.
        Die Abbildung 
         $\psi: Gal(K(\sqrt[n]{a})|K)\rightarrow \mu_n(K), \sigma \mapsto \omega_\sigma = \frac{\sigma(\sqrt[n]{a})}{\sqrt[n]{a}}$ ist eine injektiver Gruppenhomomorphismus:
         \begin{itemize}[align=left]
             \item[Gruppenhomomorphismus]
             $$\psi(\sigma\tau) = \frac{\sigma(\tau(\sqrt[n]{a}))}{\sqrt[n]{a}} = \frac{\sigma(\omega_\tau \cdot \sqrt[n]{a})}{\sqrt[n]{a}} = \omega_\tau \cdot \frac{\sigma(\sqrt[n]{a})}{\sqrt[n]{a}} = \omega_\tau \cdot \omega_\sigma = \psi(\tau)\psi(\sigma)$$
             \item[injektiv]
             , weil $\sigma$ durch $\omega_\sigma$ eindeutig feistgelegt wird, da $\sqrt[n]{a}$ prim Element von $K(\sqrt[n]{a})$.
         \end{itemize}

        Wegen $\mu_n(K) \cong (\Z/n\Z)^\times$ nach \ref{theo:3.10} a) ist $Gal(K(\sqrt[n]{a}|K))$ auch zyklisch
        
        \item[b)] $L|K$ Galois mit $Gal(L|K) = \langle\sigma\rangle$.
        Langrange-Resolvente: $\phi = \sum_{i=0}^{n-1} \xi^{-i}\cdot \sigma^i\in Abb(L,L)$
        Nach \ref{theo:3.18} sind $\sigma^0,...,\sigma^{n-1}$ linear unabhängig in $Abb(L,L)$.
        Somit ist $\phi\neq 0$. Also existiert ein $c\in L$ mit $b\coloneqq\phi(c) \neq 0$.
        Es gilt $\sigma(b) = \sum_{i=0}^{n-1} \xi^{-i}\cdot \sigma^{i+1}(c) =\xi \sum_{i=0}^{n-1} \xi^{-(i+1)}\cdot \sigma^{i+1} = \xi \cdot b$, weil $\xi^{-((n-1)+1)}=\xi^{-n}=\xi^{0}$ und $\sigma^{(n-1)+1} = \sigma^0$.
        Das Minimalpolynom von $b$ hat die Nullstellen $b, \xi\cdot b, \xi^2\cdot b,\dots,\xi^{n-1} \cdot b$ und Grad $\leq n$.
        Also $m_{b,K} = \prod_{i=0}^{n-1} (X-\xi^i\cdot b) = X^n-b^n\in K[X]$.
        Wähle $a=b^n$.
    \end{itemize}
\end{proof}

\begin{definition}$ $
    \begin{enumerate}[label=(\roman*)]
        \item man sagt eine $K$-Erweiterung $L|K$ sei \emph{durch Radikale auflösbar}, wenn ein Turm von endlichen $K$-Erweiterungen gibt $K=K_0\subseteq K_1\subseteq K_2\dots\subseteq K_r$, sodass
        \begin{itemize}
            \item Für alle $i$ ist $K_i=K_{i-1}(u_i)$ mit $u_i^{m_i}\in K_{i-1}$ für $m_i\in\N$
            \item $L\subseteq K_r$
        \end{itemize}
        \item Wir nennen ein Polynom $f\in K[X]\setminus K$ durch Radikale auflösbar, wenn der Zerfällungskörper von $f$ über $K$ durch Radikale auflösbar ist.
    \end{enumerate}
\end{definition}
Bedeutung:
Die Nullstellen von $f$ lassen sich mittels Körperoperationen (+ and *) und (iterierte) Wurzeln schreiben.

\begin{flushright}
VL vom 15.12.2023:
\end{flushright}
\begin{lemma}\label{theo:3.24}
    $K$ Körper mit $char(K)=0$
    \begin{enumerate}[label=\alph*)]
        \item Wenn $L|K$ durch Radikale auflösbar, dann existiert ein Radikalturm $K\subseteq K_1\subseteq\dots\subseteq K_r$ mit $L\subseteq K_r$ und $K_r|K$ galois'sch.
        \item Sei $\xi \in \overline{K}$  eine EW, dann ist $L|K$ durch Radikale auflösbar g.d.w. $L(\xi)|K(\xi)$ durch Radikale auflösbar ist.
    \end{enumerate}
\end{lemma}
\begin{proof}
    \begin{enumerate}[label=\alph*)]
        \item Induktion über die Länge des Turms:
        Der Induktionsanfang mit Länge $0$ ist gegeben \checkmark.
        Für den Induktionsschritt betrachte einen Turm $K\subseteq K_1\subseteq\dots\subseteq K_{n-1}\subseteq K_n$, wobei $K_n = K_{n-1}(u)$ mit $u^m \in K_{n-1}$ und $K_{n-1}|K$ galoisch ist.
        Nach Induktionsvorraussetzung muss ein solcher Turm existieren.
        Betrachte $$f = \prod_{\sigma\in Gal(K_{n-1}|K)} \left(X^m-\sigma(u^m)\right)\in K[X]$$.
        $f\in K[X]$, weil $\sigma\circ f=f$ (die Linearfaktoren werden nur permutiert) und somit alle Koeffizienten in $K_{n-1}^{Gal(K_{n-1}|K)}=K$ liegen müssen.
        Seien $u_1,\dots,u_t$ die Nullstellen von $f$ in $M$, dem Zerfällungskörper von $f$:\footnote{Wenn $K_{n-1}$ Zerfällungskörper von $W\subseteq K[X]$, dann ist $M$ Zerfällungskörper von $W\cup \{f\}$}
        $$K\subseteq K_1\subseteq\dots\subseteq K_{n-1}\subseteq K_{n-1}(u_1)\subseteq K_{n-1}(u_1,u_2)\subseteq\dots\subseteq K_{n-1}(u_1,\dots,u_t)=M$$

        Die zu beweisende Aussage gilt mit $K_r\coloneqq M$:
        $M|K$ ist normal (weil Zerfällungskörper) und separarabel (da $char(K)=0)$ und damit galois'sch.
        Offensichtlich ist auch $L\subseteq M$ und $M$ lässt sich durch einen Radikalturm darstellen.
        

        \item $ $
        \begin{itemize}
            \item[$\Leftarrow$] Sei $K(\xi) \subseteq K_1\subseteq K_r\supseteq L(\xi)$ ein Radikalturm für $L(\xi)|K(\xi)$.
            Erhalte neuen Turm mit $K\subseteq K(\xi)\subseteq K_1\subseteq\dots\subseteq K_r \supseteq L(\xi)\supseteq L$ einen neuen Radikalturm für $L|K$.
            
            \item[$\Rightarrow$]
            Sei $K\subseteq K_1\subseteq \dots\subseteq K_r\supset L$ ein Radikalturm für $L|K$.
            Dann ist $K(\xi)\subseteq K_1(\xi)\subseteq\dots\subseteq K_r(\xi)\supseteq L(\xi)$.
        \end{itemize}
    \end{enumerate}
\end{proof}

\begin{theorem}\label{theo:3.25}
    $K$ Körper, $char(K)=0$, $L|K$ endliche Erweiterung. Dann sind äquivalent:
    \begin{enumerate}[label=(\roman*)]
        \item $L|K$ ist durch Radikale auflösbar
        \item Es existiert eine endliche galois'sche Erweiterung $M|K$ mit $L\subseteq M$ und $Gal(M|K)$ auflösbar.
    \end{enumerate}
\end{theorem}
\begin{reminder*}
    $G$ Gruppe auflösbar, wenn eine Normalreihe $\trivGO=G_0\nteq G_1 \nteq\dots\nteq G_r = G$ und $G_{i+1}/G_i$ abelsch existiert bzw. $G' = [G,G]$ terminiert in $\trivGO$ nach endlich vielen Schritten.
\end{reminder*}
\begin{proof} $ $
    \begin{itemize}
        \item[(ii) $\Rightarrow$ (i)]
        Sei $n=[M:K]$ und $\xi\in\overline{K}$ eine primitive $n$-te Einheitswurzel.
        Nach \ref{theo:3.24} genügt es zu zeigen, dass $M(\xi)|K(\xi)$ durch Radikale auflösbar ist.
        Bemerke $M(\xi)|K(\xi)$ galois'sch, da $M|K$ bereits galois'sch nach Vorraussetzung.
        $$Gal(M(\xi)|K(\xi)) \leq Gal(M|K)$$
        gilt, da sich Automorphismen aus $G(M(\xi)|K(\xi))$ auf $Gal(M|K)$ eingeschränkt werden kann.
        Die ist möglich, da die evtl.\footnote{$\xi$ könnte bereits in $M$ sein. In dem Fall ist $Gal(M(\xi)|K(\xi))$ die Menge der Automorphismen in $Gal(M|K)$ ist, die $xi$ fix halten.} zusätzliche Nullstellen $\xi^i$ (als Elemente in $K(\xi)$ ) fix gehalten werden und die Automorphismen in $Gal(M(\xi)|K(\xi))$ höchstens auf den verbleibenden Nullstellen/Elementen in $M$ nicht-fix agieren können.
        
        Nach \cref{theo:1.12} sind die Untergruppen auflösbarer Gruppen auflösbar. Damit ist auch $Gal(M(\xi)|K(\xi))$ auflösbar, da $Gal(M|K)$ nach Vorraussetzung auflösbar.

        Sei $\trivGO = G_0\nteq G_1\nteq \dots\nteq G_r = Gal(M(\xi)|K(\xi))$ mit $G_i/G_{i-1}$ abelsch, nach \ref{theo:1.13} sogar zyklisch.
        Definiere $K_i=M(\xi)^{G_{r-i}}$.
        $$K(\xi) = K_0\subseteq K_1\subseteq \dots\subseteq K_r = M(\xi)$$
        Nach Haupsatz der Galoistheorie ist $M(\xi)|K_i$ galois'sch mit $Gal(M(\xi)|K_i) = G_{r-i}$.
        Nach \ref{theo:3.4} und wegen $G_{i-1}\nteq G_i$ folgt $Gal(K_{i+1}|K_i) = G_{r-i}/G_{r-i-1}$, was zyklisch ist (siehe oben).
        Sei $m=[K_{i+1}:K_i]|n$ und $K$ enthält eine primitive $n$-te EW.
        Dann gilt nach 3.22b) $K_{i+1}=K_i(\sqrt[n]{a})$ für ein $a\in K_{i-1}$
        \item[(i) $\Rightarrow$ (ii)]
        Beweisidee: Gruppenturm aus Körperturm umsetzen.
        Mit \cref{theo:3.24} finde Radikalturm $K=K_0\subseteq\dots\subseteq K_r$ mit $K_r|K$ galios'sch und $K_i=K_{i-1}(u_i)$ wobei $u_i^{m_i}\in K_{i-1}$ für geeignete $m_i\in \N$.
        Sei $n$ eine Vielfaches von $m_1,\dots,m_r$ und $\xi$ eine primitive $n$-te EW und $M=K_r(\xi)|K$ galois'sch.
        Setze $G_i = Gal(M|K_{r-i}(\xi))\leq Gal(M|K)$.
        $K_{r-i}(\xi)$ ist Radikalerweiterung von $K_{r-i-1}$.
        Nach \ref{theo:3.22} gilt/gibt es? $\trivGO=G_0\nteq G_1\nteq \dots \nteq G_r = Gal(M|K(\xi))\leq Gal(M|K)$
        Nach \ref{theo:3.22}a) ist $G_i/G_{i-1}$ für alle $i$ zyklisch.
        Da $K(\xi)|K$ galois'sch, folg aus \ref{theo:3.4}, dass normal Untergruppe und nach \ref{theo:3.10} ist die Faktorgruppe ablsch.
    \end{itemize}
\end{proof}

\begin{corollary}
    $K$ Körper mit $char(K)=0$. Dann $f\in K[X]\setminus K$ auflösbar durch Radikale gdw. Galoisgruppe des Zerfällungskörper auflösbar ist.
\end{corollary}
\begin{proof} $ $
    \begin{itemize}
        \item[$\Leftarrow$]
        $L$ Zerfällungskörper, $Gal(L|K)$ auflösbar, setze $M=L$ in \ref{theo:3.25}
        \item[$\Rightarrow$]
        Sei $L|K$ durch Radikale auflösbar.
        Nach \ref{theo:3.25} ist $K\subseteq L \subseteq M$.
        Da $L|K$ galois'sch, ist $Gal(M|L)\nteq Gal(M|K)$ und $Gal(M|K)/Gal(M|K) = Gal(L|K)$.
        Nach \ref{theo:3.25} ist $Gal(M|K)$ auflösbar und damit ist $Gal(L|K)$ als dessen Faktor auch auflösbar.
    \end{itemize}
\end{proof}

\begin{example}
    \begin{enumerate}
        \item $S_5$ nicht auflösbar ($A_5 \leq S_5$ einfach, also nicht auflösbar).
        Wenn Polynom $f$ so geartet ist, dass für dessen Zerfällungskörper $M$  gilt $Gal(M|K) \cong S_5$, dann lassen sich die Nullstellen von $f$ nicht durch iterierte Wurzeln sind.
        Z.B $f=X^5-25X+5\in \Q[X]$.
        \item Alle Untergruppen von $S_4$, $S_3$ sind auflösbar.
    \end{enumerate}
\end{example}

\subsection{Spur und Norm}
\begin{definition}
    Sei $L|K$ endlich.
    Sei $a\in L$. Die \emph{Spur} $Sp_{L/K}(a)\in K$ ist die Spur des $K$-lin. Endomorphismus $\phi_a:L\rightarrow L, \phi_a(x) = a\cdot x$.
    Ähnlich definiert man die \emph{Norm} $N_{L/K}(a) \in K$ als $\det(\phi_a)$.
    Das char. Polynom von $\phi_a$ bezeichnen wir mit $X_{a,L
    /K}\in K[X]$.
    
    Offensichtlich ist $Sp_{L/K}:L\rightarrow K$ $K$-linear.
    Weiter ist $N_{L/K}(a\cdot b) = N_{L/K}(a)\cdot N_{L/K}(b)$.
\end{definition}
\begin{example}
    $L=\Q(\sqrt{3})$, $K=\Q$.
    Sei $a = a_1+\sqrt{3}a_2\in L$, $a_i \in \Q$.
    Die darstellende Matrix von $\phi_a$ bez. der $\Q$ -Basis $1,\sqrt{3}$ ist.
    \begin{equation}
        a_1 3a_2
        a_2 a_1
    \end{equation}

    Also gilt $Sp_{L/K}(a)=2a_1$ und $N_{L/K}(a) = a_1^2-3a_2^2$
\end{example}
\begin{lemma}
    Sei $M$ ein Zwischenkörper der endlichen Körperweiterungen $L|K$. dann ist
    $\chi_{a,L/K} = \chi_{a,L/K}^{[L:M]}$ für $a\in M$.
\end{lemma}
\begin{proof}
    Wähle VR-Basen
    $v_1,\dots,v_n$ von $M$ über $K$,
    $w_1,\dots,w_m$ von $L$ über $M$
    bilden die Produkte $w_k\cdot v_i$ eine vR-Basis von $L$ über $K$. Sei $A$ die darstellende Matrix von $\phi_a$ bez. $(v_1,\dots,v_n)$. Dann gilt
    $$\chi_{a,M/L} = \det(X\cdot I_n-A)$$
    ´Die darstellende Martix von $\phi_a:L\rightarrow L$ bez der Basis $w_k\cdot v_i$ ($k=1,\dots,m$, $i=1,\dots,n$) ist.
    \TODO[Matrix-bild]
    $\Rightarrow \chi_{a,L/K} =\chi_{M/L}^m$, $m=[L:M]$
\end{proof}
\begin{remark}
    Unter der Vorraussetzung von 3.30 erhält man für $a\in M$, dass $Sp_{L/K}(a) = [L:M]\cdot Sp_{M/K}(a)$
    $N_{L/K}(a) = N_{M/K}(a)$
\end{remark}
\begin{theorem}
    Sei $L|K$ endlich und $a\in L$.
    Es sei $m_{a,K} = X^n+\alpha_{n-1}X^{n-1}+\dots +\alpha_0$.
    Dann gelten
    $$Sp_{L/K}(a) = -[L:K(a)]\alpha_{n-1}$$
    $$N_{L/K} (a) = ((-1)^n\cdot \alpha_0)^{[L:K(a)]}$$
\end{theorem}
\begin{proof}
    Nach Lemma 3.30/3.31 reicht es den Fall $L=K(a)$ zu betrachten.
    In diesem Fall hat $L$ die Basis $1,a,\dots,a^{n-1}$ über $K$.
    Die dazugehörige Matrix von $\phi_a$ lautet
    \begin{equation}
        0 0 ... 0       -\alpha_0
        1 0     \vdots
        0 1     \vdots
        \vdots  \vdots
        0 \dots 0
        0\dots  1       -\alpha_{n-1}
    \end{equation}
    $\Rightarrow$ $Sp_{L/K}(a) = -\alpha_{n-1}$
    $\Rightarrow$ $N_{L/K} (a) = (-1)^n \alpha_0$
\end{proof}
\begin{remark}
    Es gilt sogar $\chi_{a,K(a)/K} = m_{a,K}$.
    Cayley Hamilton:
    $\chi_{a,K(a)/K}(\phi_a) = 0$.
    Andererseits ist $0=\chi_{a,K(a)/K}(\phi_a)(1) = \chi_{a,K(a)/K}(\underbrace{\phi_a(1)}_{=a}) $. Da $\deg(\chi_{a,K(a)/K})=\deg(m_{a,K})$ ist $m_{a,K} = \chi_{a,K(a)/K}$.
\end{remark}

\begin{theorem}
    Sei $L|K$ endlich und separabel.
    Sei $n=[L:K]$ und $\{\sigma_1,\dots,\sigma_n\} = Hom_K(L,\overline{K})$.
    Dann gilt $Sp_{L/K} = \sum_{i=1}^n \sigma_i(a)$
    $N_{L/K} (a) = \prod_{i=1}^n \sigma_i(a)$.
\end{theorem}
\begin{proof}
    $m_{a,K} = X^r+\alpha_{r-1}X^{r-1}+\dots + \alpha_0$
    das Minimalpolynom.
    Nach lemma 2.44 $K(a)|K$ isti auch separabel.
    Folglich gibt es $r$ verschiedene $K$-Homomorphismen $\tau_1,\dots,\tau_r$ von $K(a)$ nach $\overline{K}$.
    Weiter ist $m_{a,K} = \prod_{i=1}^r (X-\tau_i(a))$.
    => $\sum_{i=1}^r\tau_i(a) = -\alpha_{r-1}$ (\TODO[why?]), $\prod_{i=1}^r \tau_i(a) = (-1)^r \cdot \alpha_0$.

    Es ist $\{{\sigma_1}_{|K(a)}, \dots, {\sigma_n}_{|K(a)}\} = \{\tau_1,\dots,\tau_r\}$.
    Für jedes $i$ ist die Anzahl der $j$'s mit ${\sigma_j}_{|K(a)} = \tau_i$ gleich $[L:K(a)]_S = [L:K(a)]$
    Daher folgt $\sum_{j=1}^n\sigma_j(a) = [L:K(a)]\cdot \sum_{i=1}^r \tau_i(a) = -[L:K(a)]\alpha_{r-1} \overset{Satz...}{ =} Sp_{L|K}(a)$.
\end{proof}
\begin{theorem}
    Sei $M$ ein Zwischenkörper der endlichen Erweiterung $L|K$.
    Dann gelten $Sp_{L|K} = Sp_{M|K}\circ Sp_{L|M}$
    $N_{L|K} = N_{M|K}\circ N_{L|M}$
\end{theorem}
\begin{proof}nur für separable Erweiterungen:
    Seien $hom_M(L,\overline{K}) = \{\tau_1,\dots,\tau_m\}$, $m=[L:M]$
    $hom_K(M,\overline{K}) = \{\sigma_1,\dots,\sigma_l\}$, $l = [M:K]$.
    Dann ist $\{\overline{\sigma}_j\circ\tau_i\mid i \in \{1,\dots,m\}, j\in \{1,\dots,l\}\} = hom_K(L,\overline{K})$, wobei $\overline{\sigma}_j$ Erweiterung vorn $\sigma_j$ zu $\overline{K}\rightarrow \overline{K}$ (siehe Beweis von 2.46)
    \begin{align*}
        Sp_{L|K}(a) &= \sum_{i,j} \overline{\sigma}_j(\tau_i(a))\\
        &=\sum_j\overline{\sigma}_j(\underbrace{\sum_i \tau_i (a)}_{Sp_{L|M}(a)\in M})\\
        &= \sum_j \sigma_j (Sp_{L|M} (a))//
        &= Sp_{M|K}(Sp_{L|M}(a))
    \end{align*}
\end{proof}
\begin{theorem}
    Sei $L|K$ endlich und separabel.
    \begin{enumerate}%a)
        \item Es gibt ein $a\in L$ mit $Sp_{L|K}(a)\neq 0$
        \item Durch $(v,w)\coloneqq Sp_{L|K}(v\cdot w)$ wird eine symmetirsche Bilienarform des $K$-VR $L$ definiert, die nicht ausgeartet ist.
    \end{enumerate}
\end{theorem}
\begin{proof}
Zu a)
    Die Homom. $\{\tau_1,\dots,\tau_n\} = hom_K(L|\overline{K})$ sind lin. unabhöngig als Elemente von $Abb(L^\times,\overline{K})$ und somit  von $Hom_K(L,\overline{K})$.
    Da $Sp_{L|K} = \sum_{i=1}^n \tau_i$ kann $Sp_{L|K}$ nicht die Nullabbildung sein.

    zu b)
    Sei $a\in L$ mit $Sp_{L|K}(a)\neq 0$. Dann gilt $(v, av^{-1}) = Sp_{L|K}(a) \neq 0$ => nicht ausgeartet
\end{proof}
\subsection{Anwendungen der Galoistheorie}
\begin{reminder*}[Fundamentalsatz der Algebra]
    $\C$ ist algebraisch abgeschlossen.
\end{reminder*}
\begin{proof}
    Sei $\R\subseteq\C\subseteq L$ eine Kette von endlichen Erweiterungen.
    Zu zeigen: $L= \C$.
    Durch Vergrößern von $L$ können wir annehmen, dass $L|\R$ eine Galoiserweiterung ist.
    Sei $G\coloneqq Gal(L|\R)$.
    Es ist $[L:\R] = \card{G} = 2^k\cdot m$ mit $2\nmid m$.
    Es sei $H\leq G$ eine $2$-Sylowuntergruppe von $G$.
    \TODO[Bild]
    Satz vom primitiven Element: $L^H= \R(\alpha)$.
    Da $m_{\alpha,\R}$ eine Nullstelle in $\R$ besitzt (Zwischenwertsatz!), ist $m_{\alpha,\R}$ linear, also $m=1$ und $L^H= \R$.
    $\Rightarrow$ $[L:\R] = 2^k$, $[L:\C] = 2^{k-1}$

    Ang. $k\geq 2$. Dann existiert $H'\leq G' = Gal(L|\C)$ mit $\card{H'} = 2^{k-2}$ (allg. Aussage über $p$-Gruppen)
    \TODO[bild]
    Da jedes quadratische Polynom in $C[X]$ zerfällt in $\C$, folgt ähnlich wie oben ein Widerspruch (bet $m_{a,\C}$ für die $L^H=\C(a)$)
\end{proof}

\end{document}