\documentclass[../main.tex]{subfiles}
\graphicspath{{\subfix{../images/}}}

\begin{document}
\begin{definition}
    Eine algebraische Körpererweiterung $L|K$ heißt \emph{Galoiserweiterung} (oder galois'sch), wenn $L|K$ normal und separabel ist.
    Man nennt dann $Gal(L|K) = Aut_K(L)$ die \emph{Galoisgruppe} von $L|K$.

    Sei $F$ ein Körper und $H\leq Aut(F)$ eine Untergruppe.
    dann ist $F^H = \{x\in F\mid \forall\sigma\in H:\sigma(x) = x\}$ ein Teilkörper von F, genannt \emph{Fixkörper} von H.
\end{definition}
\subsection{Hauptsatz der Galoistheorie}
\begin{lemma}
    Sei $L|K$ Galoiserweiterung. Dann ist $L^{Gal(L|K)} = K$.
\end{lemma}
\begin{proof}$ $
    \begin{enumerate}
        \item["'$\supseteq$"'] klar
        \item["'$\subseteq$"'] (Durch Kontraposition) Sei $a\in L\setminus K$. Das Minimalpolynom $m_{a,K}$ hat Grad $\geq2$.
        \begin{itemize}
            \item $L|k$ normal => $m_{a,K}$ zerfällt in $L[X]$ in Linearfaktoren.
            \item $L|K$ separabel => $m_{a,k}$ hat keine mehrfach Nullstelle.
        \end{itemize}
    
        Also gibt es eine weitere Nullstelle $b$ von $m_{a,K}$ mit $b\neq a$.
        Nach Satz \cref{theo:2.37} existiert eine $\sigma \in Gal(L|K)$ mit $\sigma(a)=b$.
        => $a\notin L^{Gal(L|K)}$
    \end{enumerate}
\end{proof}
\begin{theorem}
    Seien $L$ ein Körper und $H\leq Aut(L)$ eine endliche Untergruppe.
    Dann ist
    \begin{enumerate}
        \item $L|L^H$ galois'sch
        \item $[L:L^H] = \card{H}$
        \item $Gal(L|L^H)\cong H$
    \end{enumerate}
\end{theorem}
\begin{proof}
    Sei $a\in L$. Betrachte die $H$-Bahn von $a$.
    $$H\cdot a = \{\sigma(a)\mid \sigma\in H\} = \{a_1,\dots, a_n\} \subseteq L$$
    Betrachte Polynom $f_a = \Pi_{i=1}^n (X-a_i)$ hat Koeffizienten in $L^H$.
    Für $\sigma\in H$ ist $\sigma_*(f) = \Pi_{i=1}^n(X-\sigma(a_i)) = \Pi_{i=1}^n(X-a_i)$.
    Weil alle Nullstellen von $f_a$ auch Nullstelle von $m_{a,L^H}$, muss schon $f_a=m_{a,L^H}$ sein.\footnote{$\sigma$ erhält Minimalpolynom-Nullstellen oder so}
    Nach Konstruktion hat $f_a$ nur einfache Nullstellen, ist also separabel. => $a$ separabel. $\overset{=>}{a\ beliebig}$ $L|L^H$separabel.
    Weiter zerfällt $f_a=m_{a,L^H}$ in Linearfaktoren -> $L|L^H$ normal and damit Galois.

    Es gilt$\card{H}\leq \card{Gal(L|L^H)} = [L:L^H]_S = [L:L^H]$.
    Angenommen $\card{H} < [L:L^H]\leq \infty$. 
    Dann findin wir ein $L_0$ mit $L^H\subseteq L_0 \subseteq L$ mit $\card{H} \leq [L_0:L^H] <\infty$.
    Der Satz vom primitiven Element liefert ein $a\in L_0$ mit $L_0 = L^H(a)$.
    Aber $f_a=m_{a,L^H}$ hat grad $leq \card{H}$.
    => $[L_0:L^H] = \deg(m_{a,L^K})\leq \card{H}$ \Lightning

    Also ist $[L:L^H] = \card{H} = \card{Gal(L|L^H)}$
    => $H=Gal(L|L^H)$.
\end{proof}
\begin{flushright}
VL vom 30.11.2023:
\end{flushright}
\TODO[Die Stunde muss korrigiert und formatiert werden!!!]
\begin{theorem}[Hauptsatz]
    $L|K$ endliche Galoiserweiterung.
    $$U=\{H\leq Gal(L|K) \mid H \text{untergruppe}\}$$
    $$Z:= \{E\subseteq L \mid K\subseteq E \subseteq L \text{Zwischenkörper}\}$$
    \begin{enumerate}%a)
        \item Die Zuordnungen $Fix: U \rightarrow Z, H\mapsto L^H$ und $\tau: Z\rightarrow U, E \mapsto Gal(L|E)$
        sind zueinander inverse Bijektionen.
        $Fix$ und $\tau$ sind enthaltungsumkehrend
        $H_1\leq H_2 \Rightarrow Fix(H_1) \subseteq Fix(H_2)$
        $E\subset E_2 \Rightarrow \tau(E_1) \leq \tau(E_2)$
        \item Sei $E \in Z$, dann ist $E|K$ normal genau dann wenn $Gal(L|E)$ ein Normalteiler von $Gal(L|K)$ ist.
        $$ Gal(E|K) \cong Gal(L|K)/Gal(L|E)$$
    \end{enumerate}
\end{theorem}
\begin{proof}
    Zu a)
    Beachte: $E\in Z$ , dann $L|E$ galoissch (nach 2.37 a, nach 2.44)
    * $E\in Z$: $Fix(Gal(L|E)) = L^{Gal(L|E)}\overset{3.2}{=} E$
    * $H\in U$: $\tau(Fix(H)) = \tau(L^H) = Gal(L|L^H)\overset{3.3}{=}H$
    * $H_1\leq H_2$: $x\in L^{H_2}$ D.h. $\forall\sigma\in H_2\colon \sigma(x)=x$ also insbesondere für $\sigma\in H_1 \Rightarrow x\in L^{H_1}$

    $E_1\subseteq E_2$: $\sigma \in \tau(E_1) = Gal(L|E_2)\leq Gal(L|E_1)=\tau(E_1)$
    \begin{remark*}
        Für $\sigma\in Gal(L|K)$ und $H\in U$, dann $\sigma(L^H) = L^{\sigma H \sigma^{-1}}$, denn für $x\in L^{\sigma H \sigma^{-1}} \Leftrightarrow \forall tau \in H\colon \sigma tau \sigma^{-1}(x)=x \Leftrightarrow \forall tau \in H\colon tau \sigma^{-1}(x) = \sigma^{-1}(x) \Leftrightarrow \sigma^{-1}x\in L^H \Leftrightarrow x\in \sigma(L^H)$
    \end{remark*}

    Zu b)
    Ist $E|K$ normal, so gilt $\sigma(E) = E$ für alle $\sigma \in Gal(L|K)$ nach Satz 2.35 (ii).
    $E=L^{Gal(L|E)} \Rightarrow \sigma Gal(L|E) \sigma^{-1} = \sigma \tau(E) \sigma^{-1} = Gal(L|E)$ für alle $\sigma \in Gal(L|K)$.
    Damit ist $H:=\tau(E)$ normal.
    Damit ist $Gal(L|E)\nteq Gal(L|K)$.

    $Gal(L|E) \nteq Gal(L|K)$, d.h. $\forall \sigma\in Gal(L|K)$ gilt $\sigma(E)=\sigma(L^{Gal(L|E)}) = L^{\sigma Gal(L|E) \sigma^{-1}} = L^{Gal(L|E)} = E$.
    Nach Satz 2.35 (ii) ist $E|K$ normal.

    * Sei $E|K$ normal. Die Restirktionsabbildung $r_E: Gal(L|K) \rightarrow Gal(E|K), \sigma \mapsto \sigma_{|E}$ ist Gruppenhomomorphismus mit $Kern(r_E) = Gal(L|E)$.
    $r_E$ ist surjektiv:
    Für $tau\in Gal(E|K)$ findet man dank Fortsetzungssatz (2.31) ein $\sigma \in Gal(L|K)$ sodass $\sigma_{|E}=tau$.
    Mit Homomorphiesatz folgt die Behauptung.
\end{proof}
\begin{remark}
    $L|K$ endliche Galoiserweiterung $H\leq Gal(L|K)$
    \begin{itemize}
        \item $[L:L^H] = \card{H}$
        \item $[L^H:K] = [L:K]/[L:L^H] = \frac{\card{Gal(L|K)}}{\card{H}} = [Gal(L|K):H]$
    \end{itemize}
\end{remark}
\begin{example}
    Wie bei (2.36 a)
    $L$ Zerfällungskörper von $X^4-2$ über $\Q$.
    $L = \Q(a, ia, -a, -ia) = \Q(a,i)$ mit $a=\sqrt[4]{2}$.
    Damit ist $[L:\Q] = 8$.
    $L|\Q$ ist Galois Gruppe.
    $\sigma\in Gal(L|\Q)$ eindeutig bestimmt durch $\sigma(a)\in \{a,ia,-a,-ia\}$, $\sigma(i)\in \{i,-i\}$.
    Da $\card{Gal(L|\Q)} = 8$ sind alle Kombinationen möglich.
    \TODO[Graphik]
    $Gal(L|\Q) = \{id,\rho,\rho^2,\rho^3,\tau,\rho\tau,\rho^2\tau,\rho^3\tau\}$ % In 2 Zeilen!
    $\tau\rho^{-1}=\rho \tau = \rho\tau^3$ % Bei erstem fakor klammer??
    Isomorph zu der Diedergruppe:
    $D_4 = <x,y\mid x^4=1, y^2=1,xy=yx^3>$
    \TODO[Andere Graphik und mehr...]
\end{example}


This is not
\begin{theorem}[Translationssatz] % No number
    $L_1$,$L_2$ Körper und $L_1|K$, $L|K$ gemeinsame Erweiterung ungskörper
    $L_1|K$ endl. Galoiserweiterung => $L_1\cdot L_2|L_2$ ist Galois-erw.
    
    Durchschnitt: $Gal(L_1L_2|L_2)\rightarrow Gal(L_1|K)$
    $\sigma \mapsto \sigma_{|L_1}$ def. Isom.
    $Gal(L_1L_2|L_2) \cong Gal(L_1|L_1\cap L_2)$

    $H\in $
\end{theorem}
\end{document}